\documentclass[10pt]{article}
\usepackage[utf8]{inputenc}
\usepackage{url}
\usepackage{hyperref}
\usepackage{amsmath}
\usepackage{amsfonts}
\usepackage{amssymb}
\usepackage{graphicx}
\usepackage{float}
\usepackage{lipsum}
\usepackage{multicol}
\usepackage{xcolor}
\usepackage{natbib}
\usepackage[font=small]{caption}
\usepackage{graphicx,float}
\addtolength{\abovecaptionskip}{-3mm}
\addtolength{\textfloatsep}{-5mm}
\setlength\columnsep{20pt}

\usepackage{wrapfig}

% \documentclass{article}
\usepackage{amsmath}
\usepackage{esint}
\usepackage[LGR, T1]{fontenc}
\usepackage[utf8]{inputenc}
\usepackage[greek]{babel}
\usepackage{alphabeta}
\usepackage{graphicx}
\usepackage{tikz}

\title{ \selectlanguage{english} Project 1 - \selectlanguage{greek} Τεχνητή Νοημοσύνη}

\usepackage[a4paper,left=1.50cm, right=1.50cm, top=1.50cm, bottom=1.50cm]{geometry}


\begin{document}

   
   \begin{center}
        {\Large \textbf{\selectlanguage{english} Project 2 \ - \ \selectlanguage{greek} Τεχνητή Νοημοσύνη}}\\
        \vspace{1em}
        {\large Παναγιώτα Γύφτου ,   A.M.: 1115201900318  } \\
        \vspace{1em}
        {\large Δεκέμβριος 2022}
    \end{center}
    
    
    \begin{center}
        \rule{150mm}{0.2mm}
    \end{center}

    \begin{abstract}
    Θέμα εργασίας: Αναζήτηση με Αντιπαλότητα. \\ 
    
    Απαντήσεις θεωρητικών ασκήσεων, εργασίας 2.
    

    \end{abstract}

    \begin{center}
        \rule{150mm}{0.2mm}
    \end{center}

    \vspace{5mm}

\section*{Πρόβλημα 1}
\vspace{5mm}
\ \ \ \ \ \ \ \ \ \ \ \ \ \ \ \ 
\ \ \ \ \ \ \ \ \ \ \ \ \ \ \ \
\ \ \ \ \ \ \ \ \ \ \ \ \ - \ - \bf   Πρώτο σκέλος προβήματος \normalfont - \ - \\

Στην περίπτωση όπου ο $MIN$ επιλέξει να παίξει μη βέλτιστα, αυτός ο τρόπος παιχνιδιού  για τον $MAX$ δεν έχει αρνητικό αντίκτυπο. Ο $MAX$ μπορεί να χειριστεί την κατάσταση εξίσου καλά.Ωστόσο η αξία κίνησης που επιλέγει ο  $MAX$ μπορεί να είναι διαφορετική από αυτή της υπό κανονικής συνθήκης, δηλαδή στην κατάσταση όπου και οι δύο πλευρές παίζουν βέλτιστα. \\

Οι εκδοχές είναι δύο: \\

α. Ο  $MIN$  να επιλέξει να παίξει μη βέλτιστα, και ο $MAX$ να διαλέξει μια μεγαλύτερη αξίας κίνηση σε σχέση με αυτή

\ \ \ \ που θα επέλεγε με το βέλτιστο τρόπο παιχνιδιού. \\ \\

β. Ο  $MIN$  να επιλέξει να παίξει τυχαία βέλτιστα και ο $MAX$ να διαλέξει την σωστή κίνηση. \\ \\ \\

Θα δώσουμε ένα παράδειγμα για να γίνουν πιο κατανοητά τα παραπάνω .Έστω για παράδειγμα το δέντρο παιχνιδιού που βρίσκεται στο βιβλίο Σελ. 176 και στις διαφάνειες (Αναζήτηση με Αντιπαλότητα) δ.11

\begin{figure}[H]
    \includegraphics[width=\linewidth, height=.25\textheight,keepaspectratio=true]{1a.png}\\
    \caption{Ένα δέντρο παιχνιδιού δύο στρώσεων.}
\end{figure} 

Έστω ότι ο $MIN$ δεν παίζει βέλτιστα, και υποθέτουμε ότι διαλέγει τα φύλλα με την μέγιστη χρησιμότητα (μπορεί να επιλέξει εννοείται και οποιονδήποτε άλλο φύλλο μεγαλύτερο από το φύλλο με την ελάχιστη χρησιμότητα, απλά εδώ επιλέγω να εφαρμόσω για αυτόν το τρόπo παιχνιδιού την ακραία περίπτωση, δηλαδή ότι ο παίχτης $MIN$ παίζει \bf αυστηρά μη βέλτιστα\normalfont).Επομένως έχουμε:
\begin{figure}[H]
    \includegraphics[width=\linewidth, height=.25\textheight,keepaspectratio=true]{1a_1.png}\\
    \caption{Ο παίχτης $MIN$ παίζει αυστηρά μη βέλτιστα. Η τιμή του κόμβου ρίζα $MAX$ έχει τιμή 14.}
\end{figure} 

Συμπεραίνουμε ότι η χρησιμότητα για τον $MAX$ σε ένα παιχνίδι όπου ο $MIN$ παίζει μη βέλτιστα θα είναι τουλάχιστον ίση με αυτή του βέλτιστου παιχνιδιού, δηλαδή μεγαλύτερη ή ίση.Η χρησιμότητα του $MAX$ δεν θα γίνει ποτέ μικρότερη σε σχέση με αυτή του βέλτιστου τρόπου παιχνιδιού. \\ \\ \\ 

\ \ \ \ \ \ \ \ \ \ \ \ \ \ \ \ 
\ \ \ \ \ \ \ \ \ \ \ \ \ \ \ \
\ \ \ \ \ \ \ \ \ \ \ \ \ - \ - \bf   Δεύτερο σκέλος προβήματος \normalfont - \ - \\

Εάν ο $MAX$ γνωρίζει εξαρχής ότι ο $MIN$ στερείται υπολογιστικής ισχύος, τότε ο $MAX$  παγιδεύει τον $MIN$ επιλέγοντας μια κίνηση με λιγοστές πιθανότητες να κερδίσει, η οποία σε ένα βέλτιστο παιχνίδι θα οδηγούσε τον $MAX$ σε σίγουρη \bf ήττα.\normalfont Παίρνοντας αυτό το ρίσκο ο $MAX$ ξεγελάει τον $MIN$,ο οποίος διαλέγει το φύλλο με την μεγαλύτερη χρησιμότητα από αυτή που θα έπρεπε βέλτιστα να επιλέξει.\ \\ \\

Θα δώσουμε ένα παράδειγμα για να γίνουν πιο κατανοητά τα παραπάνω .Έστω για παράδειγμα το παρακάτω δέντρο, ο βέλτιστος τρόπος παιχνιδιου θα είχε σαν αποτέλεσμα ο ΜΑΧ να επιλέξει την κίνηση α3 και ο ΜΙΝ την $d3$.

\begin{figure}[H]
    \includegraphics[width=\linewidth, height=.25\textheight,keepaspectratio=true]{1a _2.png}\\
    \caption{Ένα δέντρο παιχνιδιού δύο στρώσεων, όπου και οι δύο πλευρές παίζουν βέλτιστα.}
\end{figure} 

Γνωρίζοντας ο ΜΑΧ την αδυναμία του ΜΙΝ θα διαλέξει να ακολουθήσει την κίνηση α1, έτσι ο ΜΙΝ μη γνωρίζοντας ότι ο ΜΑΧ ξέρει ότι υπάρχει πρόβλημα θα επιλέξει εν αγνοία του την κίνηση $b3$ με το αντίστοιχο  φύλλο να έχει τιμή χρησιμότητας 8.Έτσι ο ΜΑΧ κερδίζει και τα πάει ακόμα πιο καλύτερα σε σύγκριση με τα αποτελέσματα του βέλτιστου παιχνιδιού.Η κατάσταση αυτή φαίνεται στο παρακάτω σχήμα.

\begin{figure}[H]
    \includegraphics[width=\linewidth, height=.25\textheight,keepaspectratio=true]{1a _3.png}\\
    \caption{Ένα δέντρο παιχνιδιού δύο στρώσεων, όπου και οι δύο πλευρές παίζουν μη βέλτιστα.}
\end{figure} 

\section*{Πρόβλημα 2}
\vspace{1em}
Για την επίλυση του προβλήματος 2, μελετήθηκαν οι διαφάνειες της Ενότητα 3, Αναζήτηση σε γράφους (διαφ.: 11 έως 95) και οι σελίδες του βιβλίου 175,176,177,179,180. \\ \\  
\bf α. \ \normalfont Στην παρακάτω εικόνα φαίνεται συμπληρωμένο το δέντρο παιχνιδιού με τις τιμές $minimax$ των κόμβων.  \\ \\
\includegraphics[width=\linewidth,
height=.50\textheight,keepaspectratio=true]{a.png}\\\\\
\bf β. \ \normalfont H $minimax$ απόφαση στη ρίζα του δέντρου παιχνιδιού είναι \bf 8. 
  \\ \\
\includegraphics[width=\linewidth,
height=.20\textheight,keepaspectratio=true]{b.png} 
\begin{figure}[H]
\bf γ. \ \normalfont \\ \\ \\ \\ 
    \includegraphics[width=\linewidth,
    height=.25\textheight,keepaspectratio=true]{1.png}\\ 
    \caption{ Το δέντρο παιχνιδιού πριν την εφαρμογή της τεχνικής κλάδεμα άλφα-βήτα.}
    
    \begin{center}
        \rule{120mm}{0.2mm}
    \end{center}
 
    \vspace{20mm}

    \includegraphics[width=\linewidth,
    height=.25\textheight,keepaspectratio=true]{2.png}\\ 
    \caption{Το πρώτο φύλλο (το πρώτο από τα αριστερά) έχει τιμή 4. Οπότε ο κόμβος $MAX$ έχει τιμή τουλάχιστον 4.  }
    
    \begin{center}
        \rule{120mm}{0.2mm}
    \end{center}
    
\end{figure}

\begin{figure}[H]

    \includegraphics[width=\linewidth,
    height=.25\textheight,keepaspectratio=true]{3.png}\\
    \caption{Το 2ο φύλλο (από τα αριστερά) έχει τιμή 8. Επομένως ο κόμβος $MAX$ έχει τιμή ακριβώς 8, διότι επιλέγει τελικά την τιμή 8 που έχει μεγαλύτερη αξία από το φύλο με τιμή 4 και αυτό το τρέχων φύλλο που επιλέχθηκε είναι και ο τελευταίος διάδοχος του $MAX$.Άρα ο πατέρας του $MAX$, δηλαδή ο κόμβος $MIN$,έχει την τιμή το πολύ 8.  }
    \begin{center}
        \rule{120mm}{0.2mm}
    \end{center}
    \includegraphics[width=\linewidth,
    height=.25\textheight,keepaspectratio=true]{4.png}\\
    \caption{Το 3o φύλλο (από τα αριστερά) έχει τιμή 9. Οπότε ο κόμβος $MAX$ έχει τιμή τουλάχιστον 9.}
    \begin{center}
        \rule{120mm}{0.2mm}
    \end{center}
    
    \vspace{7mm}
    \includegraphics[width=\linewidth,height=.25\textheight,keepaspectratio=true]{5.png}\\
    \caption{Ο πρώτος διάδοχος του 1ου κόμβου (από τα αριστερά) $MIN$ έχει αξία 8, έτσι ο κόμβος πατέρας $MIN$ δεν θα διαλέξει ποτέ τον κόμβο διάδοχο με αξία 9 τουλάχιστον, διότι έχει μεγαλύτερη αξία από τον πρώτο διάδοχο.Επομένως ο πατέρας $MIN$ έχοντας επιλέξει και ελέγξει όλους τους διαδόχους, έχει τιμή ακριβώς 8.}
    
\end{figure}

\begin{figure}[H]
    
    \includegraphics[width=\linewidth, height=.25\textheight,keepaspectratio=true]{6.png}\\
    \caption{Έτσι η τιμή της ρίζας έχει μία επιλογή και επομένως η τρέχουσα τιμή της είναι τουλάχιστον 8.}
    \includegraphics[width=\linewidth,
    height=.25\textheight,keepaspectratio=true]{7.png}\\
    \caption{Το 5ο φύλλο (από τα αριστερά) έχει τιμή 2. Οπότε ο κόμβος $MAX$ έχει τιμή τουλάχιστον 2.  }
    
    \begin{center}
        \rule{120mm}{0.2mm}
    \end{center}
    
    \vspace{1mm}
    \includegraphics[width=\linewidth,height=.25\textheight,keepaspectratio=true]{8.png}\\
    \caption{Το 6ο φύλλο (από τα αριστερά) έχει τιμή -2. Επομένως ο κόμβος $MAX$ έχει τιμή ακριβώς 2, διότι επιλέγει τελικά την τιμή 2 που έχει μεγαλύτερη αξία από το φύλο με τιμή -2 και αυτό το τρέχων φύλλο με αξία -2 είναι και ο τελευταίος διάδοχος του $MAX$.Άρα ο πατέρας του $MAX$, δηλαδή ο κόμβος $MIN$,έχει την τιμή το πολύ 2.  }
    
    \begin{center}
        \rule{120mm}{0.2mm}
    \end{center}
    
\end{figure}
\begin{figure}[H]    \includegraphics[width=\linewidth,
    height=.25\textheight,keepaspectratio=true]{9.png}\\
    \caption{Ο κόμβος ρίζα έχει αξία τουλάχιστον 8, γι' αυτό τον λόγο δεν θα διαλέξει ποτέ τον κόμβο διάδοχο με αξία το πολύ 2, διότι έχει μικρότερη αξία.Επομένως δεν έχει νόημα να εξετάσουμε τους υπόλοιπους θυγατρικούς κόμβους του τρέχοντος διαδόχου.Συνεχίζουμε με το 11ο φύλλο (από τα αριστερά) έχει τιμή 3. Επομένως ο κόμβος $MAX$ έχει τιμή τουλάχιστον 3.}

     \begin{center}
        \rule{120mm}{0.2mm}
    \end{center}
    \includegraphics[width=\linewidth,
    height=.25\textheight,keepaspectratio=true]{10.png}\\
    \caption{Έχοντας ελέγξει όλους του θυγατρικούς κόμβους καταλήγουμε ότι η καλύτερη μέγιστη τιμή που επιλέγει ο $MAX$ (ανάμεσα στα φύλλα με τιμές 3,6,5) είναι το φύλλο με αξία 6.Άρα ο $MAX$ έχει τιμή ακριβώς 6. Οπότε ο πατέρας του $MAX$, δηλαδή ο κόμβος $MIN$,έχει τιμή το πολύ 6.Ο κόμβος ρίζα έχει αξία τουλάχιστον 8, γι' αυτό τον λόγο δεν θα διαλέξει ποτέ τον κόμβο διάδοχο με αξία το πολύ 6, διότι έχει μικρότερη αξία.Επομένως δεν έχει νόημα να εξετάσουμε τους υπόλοιπους θυγατρικούς κόμβους του τρέχοντος διαδόχου.}
   
    \begin{center}
        \rule{120mm}{0.2mm}
    \end{center}

    \includegraphics[width=\linewidth,
    height=.25\textheight,keepaspectratio=true]{11.png}\\
    \caption{Τέλος έχοντας ελέγξει όλους τους διαδόχους της ρίζας κόμβου καταλήγουμε ότι η τιμή της ρίζας κόμβου είναι ακριβώς 8.}
    
\end{figure}
\begin{figure}
    \includegraphics[width=\linewidth,
    height=.25\textheight,keepaspectratio=true]{12.png}\\
    \caption{Οι εξερευνημένοι κόμβοι του δέντρου παιχνιδιού με την εφαρμογή της τεχνικής κλάδεμα άλφα-βήτα.}
    \vspace{10mm}
\end{figure} 
\section*{Πρόβλημα 3}
\vspace{1em}
\bf α. \ \normalfont Στην παρακάτω εικόνα φαίνεται συμπληρωμένο το δέντρο παιχνιδιού με τις τιμές των κόμβων αλλά και την ένδειξη της καλύτερης κίνησης με ένα βέλος, η οποία είναι η $α_{1}$. \\ \\
\includegraphics[width=\linewidth,
height=.20\textheight,keepaspectratio=true]{3a.png}\\\\\
\bf β. \\ \\ 
β.1) \ \normalfont Η αξιολόγηση του 7ου και του 8ου φύλλου είναι αναγκαία, διότι ο αλγόριθμος συμπεριφέρεται διαφορετικά για τις διαφορετικές τιμές των φύλλων.Γνωρίζουμε ότι ο κόμβος τύχης, ο πρώτος από τα αριστερά, έχει τιμή 1.5 (= 2*0.5+1*0.5), άρα η τιμή της ρίζας είναι τουλάχιστον 1.5.Ο πρώτος διάδοχος (από τα αριστερά) του δεύτερου κόμβος τύχης (από τα αριστερά) έχει την τιμή 0.Επομένως ο τελευταίος διάδοχος του δεύτερου κόμβου τύχης (που περιέχει το 7ο και το 8ο φύλλο) είναι αυτός που θα παίξει καθοριστικό ρόλο για την εύρεση της καλύτερης κίνησης.\\ \\ Έχουμε τις εξής περιπτώσης: \\ \\
(\textbf{1}) \\

Αν οι τιμές του 7ου και του 8ου φύλλου είναι μεγαλύτερες ή ίσες με 3, δηλαδή το εύρος αξίας των δύων φύλλων είναι τουλάχιστον 3 (\textit{[3,+$\infty$)}), τότε η καλύτερη κίνηση είναι αυτή  της επιλογής του δεύτερου από τα αριστερά κόμβου τύχης, δηλαδή η δεξιά κίνηση $α_{2}$. \\

Για παράδειγμα αν ο κόμβος $MIN$ που περιέχει το 7ο και 8ο φύλλο είχε τιμή 3 τότε ο κόμβος τύχης θα ήταν 1.5 και έτσι η καλύτερη κίνηση θα ήταν η $α_{2}$. \\ \\
(\textbf{2}) \\

Αν οι τιμές του 7ου ή του 8ου φύλλου είναι μικρότερες από 3, δηλαδή το εύρος αξίας των δύων φύλλων είναι μικρότερο του 3 (\textit{(-$\infty$, 3)}), τότε η καλύτερη κίνηση είναι αυτή της επιλογής του πρώτου από τα αριστερά κόμβου τύχης, δηλαδή η αριστερή κίνηση $α_{1}$. \\

Για παράδειγμα αν ο κόμβος $MIN$ που περιέχει το 7ο και 8ο φύλλο είχε τιμή - 1, όπως στο παράδειγμα μας, τότε ο κόμβος τύχης  έχει τιμή -0.5 και έτσι η καλύτερη κίνηση είναι  η $α_{1}$, αφού η τιμή 1.5 $>$ -0.5 . \\ \\ \\
\bf β.2) \ \normalfont Ο υπολογισμός των τιμών του 8ου φύλλου δεν είναι αναγκαίος, διότι ο αλγόριθμος συμπεριφέρεται εξίσου το ίδιο για όλες τις διαφορετικές τιμές των φύλλων.Γνωρίζουμε ότι ο κόμβος τύχης, ο πρώτος από τα αριστερά, έχει τιμή 1.5 (= 2*0.5+1*0.5), άρα η τιμή της ρίζας είναι τουλάχιστον 1.5.Ο πρώτος διάδοχος (από τα αριστερά) του δεύτερου κόμβος τύχης (από τα αριστερά) έχει την τιμή 0.Επομένως ο τελευταίος διάδοχος του δεύτερου κόμβου τύχης (που περιέχει το 7ο και το 8ο φύλλο) είναι αυτός που θα παίξει καθοριστικό ρόλο για την εύρεση της καλύτερης κίνησης.Ωστώσο το 7ο φύλλο έχει τιμή -1 όπου είναι χαμηλή τιμή, και κατ’ επέκταση ο κόμβος πατέρας  $MIN$ έχει τιμή το πολύ -1.Εάν οι τιμές του 8ου φύλλου ήταν μεγαλύτερες του -1, δηλαδή το εύρος τιμών ήταν το σύνολο \textit{(-1, +$\infty$)}, τότε ο κόμβος πατέρας $MIN$ δεν επίλεγε ποτέ αυτή την κίνηση, και έτσι ο κόμβος πατέρας θα είχε τιμή ακριβώς -1, με αποτέλεσμα ο κόμβος ρίζα να επιλέξει την κίνηση $α_{1}$ (αριστερά).Όμοια θα συμπεριφερόταν το πρόγραμμα και στην περίπτωση που το εύρος τιμών του 8ου φύλλου έπαιρνε μικρότερες τιμές από το φύλλο -1,γιατί πάλι ο κόμβος τύχης (ο δεύτερος από τα αριστερά) θα κατέληγε να είχε μικρότερη τιμή από τον κόμβο ρίζας.Επομένως για όλους τους συνδυασμούς το αποτέλεσμα θα ήταν ο κόμβος ρίζας να επιλέγει την αριστερή κίνηση $α_{1}$.  \\ \\ 
\bf γ) \ \normalfont 
Αρχικά γνωρίζουμε ότι ο πρώτος από τα αριστερά διάδοχος του αριστερού κόμβου τύχης έχει τιμή ακριβώς 2, άρα ο κόμβος τύχης προς το παρόν έχει τιμή 0.5*2+0.5*(τιμή δεξιού διαδόχου) = 1+0.5*(τιμή δεξιού διαδόχου) . Για την εύρεση των δυνατών τιμών του αριστερού κόμβου τύχης θα εξετάσουμε τις ακραίες περιπτώσεις των φύλλων του συνόλου [-2,2].Έχουμε: \\ \\ \\
(\textbf{1}) \ \ Εάν ο δεύτερος διάδοχος επιλέξει την κίνηση με αξίας -2 τότε ο κόμβος τύχης θα έχει τιμή: \\

 \hspace{30mm} 1+0.5*(τιμή δεξιού διαδόχου) = 1+0.5*(-2) = 1-1 = 0  \\ \\ \\
(\textbf{2}) \ \
Εάν ο δεύτερος διάδοχος επιλέξει την κίνηση με αξίας 2 τότε ο κόμβος τύχης θα έχει τιμή:\\

 \hspace{30mm}   1+0.5*(τιμή δεξιού διαδόχου) = 1+0.5*(+2) = 1+1 = 2
\\ \\ \\
Επομένως οι δυνατές τιμές του αριστερού κόμβου τύχης βρίσκονται στο διάστημα [0,2].\\ \\ \\ \\
\bf δ) \ \normalfont  \ \ 
Είναι γνωστό ότι ο αριστερός διάδοχος του κόμβου ρίζα έχει τιμή 1.5.Άρα ο κόμβος ρίζα έχει τιμή τουλάχιστον 1.5.Το 5ο φύλλο από αριστερά έχει χρησιμότητα 0.Επομένως το 6ο φύλλο και κατ’ επέκταση και ο δεξιός διάδοχος του δεξιού κόμβου τύχης δεν είναι αναγκαίο να ελεγχθούν, διότι η τιμή του δεξιού κόμβου τύχης δεν θα υπερβεί ποτέ την τιμή 1.5, αφού το εύρος τιμών των φύλλων είναι [-2,2].
Πιο συγκεκριμένα ο 3ος κόμβος ΜΙΝ από τα αριστερά δεν θα επιλέξει ποτέ φύλλο μεγαλύτερο του 0 και το σύνολο των δυνατών τιμών του 4ου κόμβου ΜΙΝ από τα αριστερά θα είναι [-2,2]
Άρα το σύνολο των δυνατών τιμών του αριστερού διαδόχου του δεξιου κόμβου τύχης θα είναι το [-2,0] και ο δεξιός όπως μόλις είπαμε θα έχει δυνατές τιμές στο σύνολο [-2,2].Επομένως το σύνολο των δυνατών τιμών του δεξιού κόμβου τύχης είναι [-2,1]. \\ \\
Οι κόμβοι που δεν χρειάζεται να αποτιμηθούν είναι σημειωμένοι με κύκλο:
\begin{figure}[H]
    \includegraphics[width=\linewidth, height=.17\textheight,keepaspectratio=true]{3a_2.png}\\
    \caption{Οι κόμβοι που δεν χρειάζεται να αποτιμηθούν είναι σημειωμένοι με κύκλο.}
\end{figure} 
\section*{Πρόβλημα 4}
\vspace{10mm}
\bf α) \ \normalfont  Για την καλύτερη ανάγνωση, κατανόηση του δέντρου παιχνιδιού, έχω φτιάξει δύο δέντρα.Στο πρώτο σχήμα απεικονίζονται αναλυτικά όλοι οι κόμβοι με τις δυνατές κινήσεις κάθε παίκτη ( τα σπίρτα στο πρώτο σχήμα είναι καλύτερα διακριτά με $zoom$), ενώ το δεύτερο σχήμα είναι πιο απλουστευμένο από το πρώτο.Πιο συγκεκριμένα στο 2ο έχουν αφαιρεθεί οι κόμβοι που ειναι συμμετρικοι, καθώς το αποτέλεσμα είναι ίδιο όπως παρατηρούμε στο πρώτο σχήμα. 
\vspace{10mm}

% ...
\begin{figure}[H]
    \centering
    \includegraphics[width=\linewidth, height=.25\textheight,keepaspectratio=true]{4a_1.png} \\
    \caption{Πρώτο δέντρο: Αναλυτική απεικόνιση όλων των κόμβων με τις δυνατές κινήσεις κάθε παίκτη.}

    \begin{center}
        \rule{120mm}{0.2mm}
    \end{center}
    \vspace{10mm}
\end{figure}
\begin{figure}[H]
    \centering
    \includegraphics[width=\linewidth, height=.45\textheight,keepaspectratio=true]{4a.png} \\
    \caption{Δεύτερο δέντρο: Αφαίρεση συμμετρικών κόμβων. }

    \begin{center}
        \rule{120mm}{0.2mm}
    \end{center}
\end{figure}
\bf β) \ \normalfont
\begin{figure} [H] 
    \vspace{10mm}
    \centering
    \includegraphics[width=\linewidth, height=.35\textheight,keepaspectratio=true]{4b_2.png} \\
    \caption{Το δέντρο παιχνιδιού μετά την εφαρμογή της τεχνικής κλάδεμα άλφα-βήτα. }

    \begin{center}
        \rule{120mm}{0.2mm}
    \end{center}
\end{figure}
\begin{figure}[H]
    \centering
    \includegraphics[width = .8\textwidth]{4b_3.png} \\
    \caption{Οι εξερευνημένοι κόμβοι του δέντρου παιχνιδιού με την εφαρμογή της τεχνικής κλάδεμα άλφα-βήτα. }

    \begin{center}
        \rule{120mm}{0.2mm}
    \end{center}
    
\end{figure} 

\bf γ) \ \normalfont 
Γνωρίζουμε ότι όταν δύο παίχτες παίζουν αλάνθαστα, ο τρόπος παιχνιδιού είναι βέλτιστος.Παρατηρούμε στο συγκεκριμένο παιχνίδι με δεδομένα, 2 στοίβες και 2 όμοια αντικείμενα,ότι και οι δύο πλευρές παίζουν αλάνθαστα, με αποτέλεσμα την νίκη του $MIN$.Συμπεραίνουμε ότι οι παίχτες ακολουθώντας την ίδια στρατιγική (δηλ. βέλτιστων κινήσεων) και με το κόσμο παιχνιδιού να είναι στημένος με τα ίδια δεδομένα (αριθμό στοίβων, αντικειμένων,ίδια σειρά παιξίματος παιχτών) μοναδικός νικητής θα είναι πάντα ο $MIN$ με την απαραίτητη προυπόθεση ότι θα παίζει πάντα, δεύτερος.Επομένως νικητής είναι πάντα ο δεύτερος.
\end{document}

\section{...}

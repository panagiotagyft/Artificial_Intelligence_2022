\documentclass[10pt]{article}
\usepackage[utf8]{inputenc}
\usepackage{url}
\usepackage{hyperref}
\usepackage{amsmath}
\usepackage{amsfonts}
\usepackage{amssymb}
\usepackage{graphicx}
\usepackage{float}
\usepackage{lipsum}
\usepackage{multicol}
\usepackage{xcolor}
\usepackage{natbib}
\usepackage[font=small]{caption}
\usepackage{graphicx,float}
\addtolength{\abovecaptionskip}{-3mm}
\addtolength{\textfloatsep}{-5mm}
\setlength\columnsep{20pt}

\usepackage{wrapfig}

% \documentclass{article}
\usepackage{amsmath}
\usepackage{esint}
\usepackage[LGR, T1]{fontenc}
\usepackage[utf8]{inputenc}
\usepackage[greek]{babel}
\usepackage{alphabeta}
\usepackage{graphicx}
\usepackage{tikz}

\title{ \selectlanguage{english} Project 1 - \selectlanguage{greek} Τεχνητή Νοημοσύνη}

\usepackage[a4paper,left=1.50cm, right=1.50cm, top=1.50cm, bottom=1.50cm]{geometry}


\begin{document}

   
   \begin{center}
        {\Large \textbf{\selectlanguage{english} Project 3 \ - \ \selectlanguage{greek} Τεχνητή Νοημοσύνη}}\\
        \vspace{1em}
        {\large Παναγιώτα Γύφτου ,   A.M.: 1115201900318  } \\
        \vspace{1em}
        {\large Ιανουάριος 2023}
    \end{center}
    
    
    \begin{center}
        \rule{150mm}{0.2mm}
    \end{center}

    \hspace{69mm}
    \textbf{Θέμα εργασίας} \\
    
\hspace{15mm}1. Προβλήματα Ικανοποίησης Περιορισμών. \\ \\

\hspace{15mm}2. Πράκτορες βασισμένοι στη γνώση και Προτασιακή Λογική.


    \begin{center}
        \rule{150mm}{0.2mm}
    \end{center}

    \vspace{5mm}
    
\section*{Δημιουργημένο Αρχείο \& Εκτέλεση}

   \vspace{5mm}

Το πρόβλημα 1, έχει υλοποιηθεί στο δημιουργημένο αρχείο $kenken.py$ \\

$\circ \ $ Για να εκτελέστε το αρχείο, πληκτρολογήστε: \\

 \ \hspace{60mm}  $ \bf python \ kenken.py$ \normalfont \\ \\ 
 Κατά την εκτέλεση θα σας ρωτήσει να επιλέξετε από το $menu$, το μέγεθος του $puzzle$, με το οποίο θέλετε να ξεκινήσετε την προσομοίωση.Έπειτα θα σας ρωτήσει ποιον αλγόριθμο επιθυμείτε. Για παράδειγμα : \\ \\
Αρχικά επιλέγουμε το $puzzle \ \ 3x3$: πρέπει να πληκτρολογήσετε $3x3$\\ \\
Και έπειτα επιλέγουμε τον αλγόριθμο $FC+MRV$, πρέπει να πληκτρολογήσετε: $FC+MRV$ \\ \\
Τέλος έχω ένα αρχείο $examples.zip$, το οποίο περιέχει τις φωτογραφίες με τις λύσεις των $puzzles$. 

\section*{Πρόβλημα 1}
\vspace{5mm}
1.

\hspace{5mm}Το $puzzle \ Kenken$ μπορεί να θεωρηθεί ως ένα πρόβλημα ικανοποίησης περιορισμών, με $n \ x \ n$ μεταβλητές, όπου $n \ x \ n$ οι διαστάσεις του πλέγματος.Το ζητούμενο σε αυτό το πρόβλημα είναι να συμπληρωθούν όλα τα κουτάκια του πλέγματος $Kenken$, με τέτοιο τρόπο, ώστε πρώτον να μην εμφανίζονται στην ίδια σειρά και ίδια γραμμή ο ίδιος αριθμός δύο φορές και δεύτερον το αποτέλεσμα της πράξης κάθε κλίκας να είναι ίσο με την  αναγραφόμενη τιμή της κλίκας. \\ \\

Χ το σύνολο των μεταβλητών, που αποτελείται από όλα τα τετράγωνα του πλέγματος $Kenken$:
\[
X = ( X_1,X_2,...,X_n)
\] \\

$D$ το σύνολο των πεδίων ένα για κάθε μεταβλητή.Όλες οι μεταβλητές έχουν τα ίδια πεδία, ίδιου μεγέθους.

\[
D_i = {1,2,...,n}
\] \\

$C$ το σύνολο των περιορισμών.Κάθε περιορισμός $C_i$ αποτελείται από ένα ζεύγος $<scope,rel>$ (βιβλίο σελ. 210) 

Έστω $X_{1}$ και $X_2$ δύο τετράγωνα του πλέγματος. \\ \\

$\circ \ $ Κάθε αριθμός μπορεί να εμφανιστεί μόνο μία φορά στην ίδια γραμμή και στην ίδια στήλη.

\[ 
\left<   (X_1, X_2) , 
\begin{array}{ll}
\\  \ \ X_1 \neq X_2 \\ \\
     
     \end{array} 
\right > \] 
\\ 

$\circ \ $ Τα αποτελέσματα της πράξης κάθε κλίκας να είναι ίσο με τον αριθμό στόχο. \\ \\ \\ \\ \\
2. Η υλοποίηση του $Kenken \ puzzle$ βρίσκεται στο δημιουργημένο αρχείο $kenken.py$ \\

Η κωδικοποιημένη αναπαράσταση του προβλήματος είναι η εξής:

\[ 
"l_{i}c_{j} \_\text{κλίκα},c_{j+1} \_\text{κλίκα},c_{j+2}\_\text{κλίκα}.l_{i+1}c_{j}\_\text{κλίκα},c_{j+1}\_\text{κλίκα},c_{j+2}\_\text{κλίκα}.l_{i+2}c_{j}\_\text{κλίκα},c_{j+1}\_\text{κλίκα},c_{j+2}\_\text{κλίκα}...\text{κλίκα1:(αριθμός} \]
\[ \text{στόχος και πράξη),κλίκα2:(αριθμός στόχος και πράξη)}"
\] \\
Με το $l$ συμβολίζεται η γραμμή και ο δείκτης $i$, δείχνει τον αριθμό της γραμμής, ακολουθούν τα $c$ που συμβολίζουν τις στήλες και ο δείκτης $j$, δείχνει τον αριθμό κάθε στήλης.Μετά από κάθε  $c$ έπεται (η κάτω παύλα λειτουργεί ως διαχωριστικό) η κλίκα που αντιστοιχεί στο συγκεκριμένο τετράγωνο αυτής της στήλης $c$ και γραμμής $l$.Προς στο τέλος με έναν διαχωρισμό τριών κουκίδων ακολουθούν τα ονόματα της κλίκας με το αποτέλεσμα στόχο και την πράξη που αντιστοιχεί. Η καταγραφή της συγκεκριμένης κωδικοποιημένης αναπαράστασης ξεκινά από πάνω και την πρώτη στήλη από τα αριστερά προς τα δεξιά. \\ \\

-Περιγραφή Προγράμματος:

$\bullet \ $ Διεπαφή προσομοίωσης παιχνιδιού. \\ \\

Η κλάση $kenken$ είναι η διεπαφή μέσω της οποίας, ένας χρήστης μπορεί να ξεκινήσει  να τρέχει την προσομοίωση.Συγκεκριμένα, για να ξεκινήσει η προσομοίωση, ο χρήστης πρέπει να δημιουργήσει ένα αντικείμενο τύπου $kenken$ και να επικαλεστεί τη μέθοδο $backtracking\_search$ της κλάσης $csp$.  \\

Για την δημιουργία του αντικειμένου καλείται η συνάρτηση αρχικοποίησης της κλάσης $kenken$, η $\_\_init\_\_$  με όρισμα την κωδικοποιημένη αναπαράσταση.Αρχικά στο σώμα της έχουν οριστεί κάποιες σημαντικές δομές: \\

$\circ \ $ Η λίστα $Variables$, η οποία περιέχει τις μεταβλητές του προβλήματος, οι οποίες είναι τα τετράγωνα και αναπαρίστανται 

\hspace{3mm} από τις συντεταγμένες της θέσης στην οποία βρίσκονται, δηλαδή (γραμμή στήλη). \\

$\circ \ $ Το λεξικό $Domains$ περιέχει το σύνολο των πεδίων ένα για κάθε μεταβλητή, τα κλειδιά του λεξικού είναι οι 

\hspace{3mm} μεταβλητές, και αντίστοιχα τιμές είναι τα πεδία τιμών (μεταβλητή: πεδίο τιμών μεταβλητής) \\

$\circ \ $ Το λεξικό $Neighbors$ περιέχει το σύνολο των γειτόνων κάθε μεταβλητής και εδώ τα κλειδιά είναι οι μεταβλητές 

\hspace{3mm} (μεταβλητή: γείτονες μεταβλητής)\\

$\circ \ $ Το λεξικο $Cliques$ περιέχει το σύνολο των τετραγώνων κάθε κλίκας, τα κλειδιά του λεξικού είναι τα ονόματα των 

\hspace{3mm} κλικών και οι τιμές τα τετράγωνα που ανήκουν σε κάθε κλίκα (όνομα κλίκας : τετράγωνα κλίκας) \\

$\circ \ $ Το λεξικό $CliqueOperations $ περιέχει την αριθμητική πράξη κάθε κλίκας (όνομα κλίκας : αριθμητική πράξη).\\

$\circ \ $  Και τέλος ορίζεται και αρχικοποιείται η μεταβλητή $kenken\_size$, όπου θα κρατάει την πληροφορία του μεγέθους του 

\hspace{3mm} πλέγματος. \\ \\

Στην συνέχεια με την βοήθεια της μεθόδου $extractProblemData $, εξάγωνται οι πληροφορίες της κωδικοποιημένης αναπαράστασης.Αφού αρχικοποιηθούν όλες οι δομές και οι μεταβλητές (αναφερόμαστε στα μέλη της κλάσης), καλείται η αρχικοποίηση του προβλήματος ως $csp$ με δεδομένα τις μεταβλητές, τα πεδία τιμών, τους γείτονες και μία συνάρτηση περιορισμών ($ Variables, Domains, Neighbors, kenken\_constraint$).  \\

Η συνάρτηση περιορισμών $kenken\_constraint$ είναι μία μέθοδος της κλάσης $kenken$. Τα ορίσματα που καλείται να ελέγξει είναι πιθανές αναθέσεις τιμών σε δύο συγκεκριμένα τετράγωνα τη φορά. Αρχικά ελέγχεται, αν ικανοποιείται ο περιορισμός της μη εμφάνισης του ίδιου αριθμού, στην ίδια γραμμή και  στήλη, κάνοντας τον έλεγχο, αν είναι γειτονικά μεταξύ τους.Εάν είναι σε δεύτερο βήμα εξετάζεται αν και οι δύο τιμές που επρόκειτο να δοκιμαστούν προς ανάθεση είναι ίδιες, αν ισχύει αυτό, τότε σημαίνει ότι παραβιάζεται ο περιορισμός.Στην συνέχεια γίνεται ο έλεγχος του περιορισμού του αριθμητικού αποτελέσματος της κλίκας.Αρχικά εξετάζεται σε ποιές κλίκες ανήκουν τα δύο τετράγωνα.Εάν είναι στην ίδια κλίκα τότε καλούμε την μέθοδο $ clique\_constraint
$ δίνοντας της ως όρισματα τα τετράγωνα με τις τιμές τους και την κλίκα στην οποία ανήκουν.Διαφορετικά αν τα δύο τετράγωνα ανήκουν σε διαφορετικά, καλείται για το καθένα ξεχωριστά πάλι η μέθοδος $clique\_constraint$ , με την διαφορά ότι σαν ορίσματα περνάμε δύο φορές το τετράγωνο με την τιμή του και την κλίκα στην οποία ανήκει .Τέλος να βρίσκονται στην ίδια κλίκα, τότε επιστρέφουμε κατευθείαν το αποτέλεσμα παραβίασης της μεθόδου, διαφορετικά συγκρίνουμε τα αποτελέσματα των δύων κλήσεων της μεθόδου των τετραγώνων.Αν και οι δύο δεν παραβιάζουν τον κανόνα τότε επιστρέφεται $True$, διαφορετικά σε κάθε άλλη περίπτωση $False$. \\

Η  $clique\_constraint$ αρχικά βρίσκει όλα τα τετράγωνα της κλίκας στα οποία έχουν ανατεθεί τιμή, όπου αποθηκεύονται σε ένα βοηθητικό λεξικό, σε αυτό επίσης αποθηκεύονται αν δεν υπάρχουν τα τετράγωνα Α και Β με τις τιμές τους. Στην συνέχεια εξάγουμε την πληροφορία που αφορά το αριθμητικό αποτέλεσμα και το σύμβολο της αριθμητικής πράξης.Τέλος ανάλογα με την αριθμητική πράξη έχουμε τις εξής περιπτώσεις: \\

$\bullet \ $ \bf + \normalfont: Υπολογίζει το άθροισμα όλων των τετραγώνων που είναι αποθηκευμένα στο βοηθητικό λεξικό, αν το αποτέλεσμα, είναι μικρότερο ή ίσου του αριθμού στόχου της κλίκας, τότε επιστρέφει $True$ . \\

$\bullet \ $ \bf - \normalfont: Βρίσκει το αποτέλεσμα της αφαίρεσης της κλίκας, ανάμεσα σε δύο τετράγωνα.Εάν είμαστε στην περίπτωση που η μέθοδος $clique\_constraint$ έχει κληθεί για χάρη ενός τετραγώνου, τότε ελέγχεται ποιό είναι το δεύτερο τετράγωνο της κλίκας, αν του έχει ανατεθεί τιμή τότε ελέγχουμε την εγκυρότητα του αποτελέσματος της πράξης και ανάλογα επιστρέφουμε την αντίστοιχη τιμής αλήθειας.Διαφορετικά βρίσκουμε την αντίστοιχη - αν υπάρχει - πιθανή τιμή που θα ικανοποιούσε τον περιορισμό.Εάν είμαστε στην περίπτωση που έχουμε σαν όρισμα τα δύο τετράγωνα τότε απλά ελέγχουμε αν το αποτέλεσμα της αφαίρεσης ικανοποιεί τον περιορισμό.\\

$\bullet \ $ \bf * \normalfont:  Υπολογίζει το γινομενο όλων των τετραγώνων που είναι αποθηκευμένα στο βοηθητικό λεξικό, αν ικανοποιείται ο περιορισμός επιστρέφει $True$. \\

$\bullet \ $ \bf / \normalfont:  Παρόμοια λειτουργία με αυτή της αφαίρεσης , η διαφορά εδώ είναι η πράξη της διαίρεσης. \\

$\bullet \ $ \bf κανένα σύμβολο \normalfont:  Αν η τιμή του τετραγώνου προς εξέταση ταυτίζεται με τον αριθμό στόχου της κλίκας τότε επιστρέφει $True$.


\vspace{5mm}

\bf 3. \normalfont \\

\begin{figure}[H]
    \includegraphics[width=\linewidth, height=.25\textheight,keepaspectratio=true]{3_1.png}\\
    \caption{Πίνακας 3χ3}
\end{figure} 
\begin{figure}[H]
    \includegraphics[width=\linewidth, height=.25\textheight,keepaspectratio=true]{3_2.png}\\
    \caption{Πίνακας 5χ5}
\end{figure} 
\begin{figure}[H]
    \includegraphics[width=\linewidth, height=.25\textheight,keepaspectratio=true]{3_3.png}\\
    \caption{Πίνακας 7χ7}
\end{figure} 
\bf ΒΤ \normalfont:
\\ \\ Ο αλγόριθμός $BackTracking$ ακολουθεί μια απλή στρατηγική, η οποία είναι η εξής: Ο αλγόριθμος διασχίζει αναδρομικά κατα βάθος του δέντρου αναζήτησης του προβλήματος ικανοποίησης περιορισμών.Ωστόσο εάν μία επιλεγμένη τιμή οδηγήσει σε αποτυχία, ο ΒΤ υποχωρεί στην προηγούμενη μεταβλητή και προσπαθεί να της δοκιμάσει νέες νόμιμες τιμές. Η επιλογή της τιμής είναι αυθαίρετη, δεν υπάρχει κάποια σειρά ή προτίμηση στην επιλογή.Γι’ αυτό τον λόγο ο αλγόριθμος καλεί πολλές φορές την συνάρτηση περιορισμών σπαταλόντας χρόνο σε ελέγχους αλλά και αναθέτοντας πολλές τιμές σε κάθε μεταβλητή του προβλήματος.Αυτό φαίνεται και από τα αποτελέσματα των πειραματικών ελέγχων, τα οποία έχουν αποτυπωθεί στους παραπάνω πίνακες. \\ \\

\bf ΒΤ+$ \bf MRV$ \normalfont: \\ \\
Σε αυτόν τον αλγόριθμο τα πράγματα δείχνουν να χειροτερεύουν.Παρατηρούμε στους πίνακες ότι για πλέγμα μικρών διαστάσεων με την προσθήκη του ευρετικού μηχανισμού $MRV$ οι δύο αλγόριθμοι μαζί είναι αποδοτικοί, ο χρόνος είναι μικρότερος και οι αναθέσεις είναι λιγότερες.Ωστόσο όσο μεγαλώνει το πλέγμα, τόσο χειρότερη είναι η απόδοση, μάλιστα γίνεται ακόμη πιο χειρότερη από τον ΒΤ. Ο $MRV$ αυτό που κάνει είναι να επιλέγει σε αντίθεση με τον ΒΤ την μεταβλητή, με τις λιγότερες νόμιμες τιμές, συνήθως αυτό αποδίδει καλύτερα από μία τυχαία επιλογή. Παρ’ όλα αυτά ο $MRV$ δεν είναι έγκυρος 100/100, διότι δεν βοηθάει καθόλου στην επιλογή της πρώτης μεταβλητής του προβλήματος και δεν είναι σίγουρο πως θα ανταπεξέλθει σε διαφορετικά προβλήματα. \\ \\ \\ \\

\bf $ \bf FC$ \normalfont: \\ \\
Ο αλγόριθμος $Forward \ Checking$, αυτό που κάνει είναι να αναθέτει σε μια μεταβλητή Χ τιμή, όπου γι’ αυτήν την  μεταβλητή ο $FC$ διασφαλίζει συνέπεια τόξου. Ο $FC$ αν ανιχνεύσει μερική ανάθεση τιμών που είναι ασυνεπής, υπαναχωρεί, έτσι αν μια μεταβλητή συνδέεται με την Χ, δηλαδή είναι γείτονας της, και στην συγκεκριμένη δεν της έχει ανατεθεί τιμή, τότε αυτή διαγράφεται από το πεδίο της Χ.Ο $FC$ είναι ένας ισχυρός αλγόριθμος, ανιχνεύει πολλές ασυνέπειες και αυτό φαίνεται και στα αποτελέσματα. Ο $FC$ δεν μπαίνει στην διαδικασία περιττων ελέγχων, όταν εντοπίσει σφάλμα επιστρέφει αμέσως.\\ \\

\bf $ \bf FC+MRV$ \normalfont: \\ \\
Ο συνδυασμός του πρώιμου ελέγχου και του ευρετικού μηχανισμού $MRV$, δημιουργεί το πιο ισχυρό αλγόριθμο, απ’ ολους. Ο πρώιμος έλεγχος έχει την ικανότητα να βλέπει μακριά,  αξιοποιώντας αυτό ο  $MRV$ γίνεται ακόμα πιο αποδοτικότερος παρόλο που δεν εγγυάται 100/100 αποδοτικότητα , ανάλογα με το πρόβλημα.Σίγουρα υπάρχει το ενδεχόμενο σε μεγαλύτερο πλέγμα να μη φέρει τα επιθυμητά αποτελέσματα.Ωστόσο η ισχύ του αλγορίθμου φαίνεται στα αποτελέσματα.\\ \\

\bf $ \bf MAC$ \normalfont: \\ \\
Ο $MAC$ εντοπίζει ασυνέπειες που δεν μπορεί να βρει ο $FC$.Είναι πιο ισχυρός από τον   $FC$, καθώς ο  $FC$ κάνει το ίδιο με ο $MAC$, στα αρχικά τόξα στην ουρά του $MAC$,όμως σε αντίθεση με αυτόν δεν διαδίδει αναδρομικά τους περιορισμούς όταν γίνονται αλλαγές  στα πεδία των μεταβλητών. (βιβλίο σελ. 227)\\
Η σύγκριση αυτή, φαίνεται και στα πειραματικά αποτελέσματα. \\



\vspace{5mm}

\bf 4. \normalfont \\ 

\begin{figure}[H]
    \includegraphics[width=\linewidth, height=.25\textheight,keepaspectratio=true]{MINCONFLICT.png}\\
    \caption{Πίνακας $Min \ Conflict$}
\end{figure} 

Ο ευρετικός μηχανισμός ελάχιστων συγκρούσεων, ο οποίος είναι γνωστό από την θεωρία ( βιβλίο σελ.230) ξεκινάει επιλέγοντας την αρχική κατάσταση είτε τυχαία είτε με μια διαδικασία άπλησητης ανάθεσης τιμών που επιλέγει μία τιμή ελάχιστων συγκρούσεων για την κάθε μεταβλητή με τη σειρά. Για παράδειγμα όπως στο πρόβλημα των 8 βασιλισσων, ο μηχανισμός αυτός είναι εξαιρετικά αποτελεσματικός, μόνο εάν του έχουν ανατεθεί στην αρχή τιμές.Στο $kenken$ η αποδοτικότητα του $MinConflict$ δεν είναι καθόλου καλή.Πιο συγκεκριμένα όσο αυξάνεται το μέγεθος του πλέγματος τόσο ο αλγόριθμος δεν δίνει λύση.Αυτό συμβαίνει διότι ξεκινάει με μια τυχαία ανάθεση, θα καταφέρει να επιστρέψει μόνο όταν δεν έχει υπερβεί το απαιτούμενο πλήθος συγκρούσεων. επίσης ο αλγόριθμος δεν εξετάζει μεγάλο πλήθος μεταβλητών, γι΄αυτό και στο $kenken$, δεν λειτουργεί αν το πλέγμα μεγαλώσει. Συμπεραίνουμε ότι ο αλγόριθμος δεν είναι καθόλου αποδοτικός σε σχέση με τους αλγορίθμους του ερωτήματος 3. \\ \\ \\ \\ \\ \\ \\   


\section*{Πρόβλημα 2}
\vspace{5mm}

\hspace{5mm}Το φοιτητικό δωμάτιο που θα επιπλώσει ο διακοσμητής, θα περιλαμβάνει τέσσερα έπιπλα. Το ζητούμενο σε αυτό το πρόβλημα, είναι να τοποθετηθούν και τα τέσσερα έπιπλα σύμφωνα με τους περιορισμούς αισθητικής, που έχει ορίσει ο διακοσμητής. \\

Για την καλύτερη κατανόηση του χώρου και των αντικειμένων που θα τοποθετηθούν σε αυτόν, φανταζόμαστε τον χώρο και τα αντικείμενα ως επίπεδες επιφάνειες. Δηλαδή η επίπεδη επιφάνεια καταλαμβάνει τις δύο διαστάσεις του τρισδιάστατου χώρου, όπου τα αντικείμενα και ο χώρος έχουν μηδενικό όγκο, κοιτάμε τα πράγματα είτε οριζόντια (δηλαδή από πάνω) είτε κάθετα (δηλαδή όπως κοιτάμε τον τοίχο).Έτσι στο πρόβλημα μας, θα χειριζόμαστε είτε το ζευγάρι (μήκος, πλάτος), είτε το ζευγάρι (μήκος,ύψος). Η αρχή των αξόνων του σχήματος θεωρούμε ότι είναι η κάτω αριστερή γωνία (Σχήμα 3).

\vspace{5mm}

\begin{figure}[!htb]
   \begin{minipage}{0.48\textwidth}
     \centering
     \includegraphics[width=.8\linewidth]{Problem2_orizontia.png}
     \caption{Οριζόντια Οπτική}\label{Fig:Data1}
   \end{minipage}\hfill
   \begin{minipage}{0.48\textwidth}
     \centering
     \includegraphics[width=.8\linewidth]{Problem2_ka8eta.png}
     \caption{Κάθετη Οπτική}\label{Fig:Data2}
   \end{minipage}
\end{figure}

\vspace{7mm}

\begin{figure}[H]
    \includegraphics[width=\linewidth, height=.25\textheight,keepaspectratio=true]{Problem2_3D_xyz.png}\\
    \caption{Τρισδιάστατος Χώρος - Φοιτητικό Δωμάτιο}
\end{figure} 

Η επίπλωση του φοιτητικού δωματίου μπορεί να θεωρηθεί ως ένα πρόβλημα ικανοποίησης περιορισμών με τέσσερις μεταβλητές, όπου η κάθε μία αντιστοιχεί σε ένα έπιπλο.Πιο αναλυτικά: \\ \\

\bf Ε \normalfont ( το σύνολο των μεταβλητών (δηλ. τα έπιπλα) ) : \textit{ Ε\_{κρεβάτι}, \ Ε\_{γραφείο}, \ Ε\_{καρέκλα}, \ Ε\_{καναπές} } \\


$\bf D$ \normalfont ( το σύνολο των πεδίων \{  $D$\_κρεβάτι, $D$\_γραφείο, $D$\_καρέκλα, $D$\_καναπές \}) ένα για κάθε μεταβλητή.

\hspace{5mm}Όλες οι μεταβλητές έχουν τα ίδια πεδία, ίδιου μεγέθους. \\ \\

\hspace{20mm}\underline{Oριζόντια}:
\[  D\text{\_έπιπλο} =
\left[ 
\begin{array}{ll}
\\
     \ \  0 \ \leq \ x \ \leq \ 300 \ , \ \ \ 0 \ \leq \ y \ \leq \ 400 \ \
     \\ \\
     \end{array} 
\right ] \] 
\\ \\

\hspace{20mm}\underline{Κάθετα}:
\[  D\text{\_έπιπλο} =
\left[ 
\begin{array}{ll}
\\
     \ \  0 \ \leq \ y \ \leq \ 400 \ , \ \ \ 0 \ \leq \ z \ \leq \ 230 \ \
     \\ \\
     \end{array} 
\right ] \] 
\\ \\

$\bf C$ \normalfont ( το σύνολο των περιορισμών ).Κάθε περιορισμός $C\_$έπιπλο αποτελείται από ένα ζεύγος $<scope,rel>$

\hspace{5mm}(βιβλίο σελ. 210).
\\ \\

\hspace{10mm}$\bullet$ Τα έπιπλα δεν πρέπει να εφάπτονται ή να πατάνε το ένα πάνω στο άλλο. \\

\hspace{15mm}\underline{Oριζόντια}: \\ 

\hspace{15mm}Έστω δύο έπιπλα Ε\_1 και Ε\_2  και έστω Α το σύνολο όλων των συντεταγμένων που καταλαμβάνει η επιφάνεια 

\hspace{15mm}του πρώτου επίπλου  Ε\_1 και αντίστοιχα για το δεύτερο έπιπλο  Ε\_2, το σύνολο Β. Έτσι ο περιορισμός είναι 

\hspace{15mm}ότι όλες οι συντεταγμένες του συνόλου Α πρέπει να είναι διαφορετικές του συνόλου Β.Η τομή των οποίων 

\hspace{15mm}πρέπει να είναι ίση με το κενό σύνολο ( $\varnothing$ ).  Α Β =\{ ($x,y$): ($x,y) \in $ Α $ \ \land \ $  ($x,y) \in $ Β \} = $\varnothing$ \\

\hspace{15mm}Θα περιγράφαμε με την εξής αναπαράσταση τον περιορισμό του $CSP$ προβλήματος μας: \\

\[ 
\left<   (E\_1, E\_2) , 
\begin{array}{ll}
\\  \ \ E\_1 \neq E\_2 \\ \\
     
     \end{array} 
\right > \] 
\\ 

\hspace{15mm}(ακολουθείτε η λογική του βιβλίου ενότητα 6.1 σελ. 210, 211 )\\ \\

\hspace{15mm}\underline{Κάθετα}: \\

\hspace{15mm}Το ίδιο ισχύει και για την κάθετη περίπτωση, απλώς τώρα, λαμβάνονται υπόψη οι συντεταγμένες των δια-

\hspace{15mm}στάσεων $y,z$.\\ \\

\hspace{10mm}$\bullet$ Το γραφείο θα πρέπει να είναι δίπλα σε κάποια είσοδο φωτός στο δωμάτιο. \\

\hspace{15mm}Η φωτεινή πηγή του δωματίου θεωρούμε ότι είναι το παράθυρο, γι΄αυτό το γραφείο επιλέγεται να τοποθετηθεί  

\hspace{15mm}κοντά σε αυτό. Γενικά ο φωτισμός της επιφάνειας είναι αντιστρόφως ανάλογος του τετραγώνου της απόστα-

\hspace{15mm}σης της φωτεινής πηγής, δηλαδή όσο πιο μακριά είναι κάποια επιφάνεια, τόσο λιγότερο φως πέφτει πάνω της,

\hspace{15mm} έτσι  επιλέγεται το γραφείο να τοποθετηθεί όσο πιο κοντά γίνεται, περιορίζοντας το σύνολο των διαθέσιμων 

\hspace{15mm}συντεταγμένων του γραφείου. Έτσι έχουμε τον εξής μοναδιαίο περιορισμό:   \\

\[  K =
\left[  
\begin{array}{ll}
\\  \ \ 100 \leq x \leq 300\ , \ 200 \leq y \leq 400 \  \\ \\
     \end{array} 
\right ] \] 
\\ 

\[ 
\left<   (E\_\text{γραφείο}) , 
\begin{array}{ll}
\\  (E\_\text{γραφείο}) \in K \\ \\
     
     \end{array} 
\right > \] 
\\ 

\hspace{10mm}$\bullet$ Στην περιοχή όπου ανοιγοκλείνει η πόρτα δεν επιτρέπεται η τοποθέτηση επίπλων.
 \\

\hspace{15mm}Για αισθητικούς και πρακτικούς λόγους, εννοείται ότι στο σύνολο συντεταγμένων: \\ \\
\[  C =
\left[ 
\begin{array}{ll}
\\
     \ \  0 \ \leq \ x \ \leq \ 100 \ , \ \ \ 0 \ \leq \ y \ \leq \ 100 \ \
     \\ \\
     \end{array} 
\right ] \] 
\\ 

\hspace{15mm}δε θα τοποθετείται κανένα έπιπλο, καθώς η πόρτα θα μπλοκάρεται και η προσβασιμότητα στο δωμάτιο δεν 

\hspace{15mm}θα είναι εύκολη. Έτσι έχουμε τον εξής γενικό περιορισμό, (ισχύει για κάθε μεταβλητή του προβλήματος): \\

\[ 
\left<   (E\_\text{έπιπλο}) , 
\begin{array}{ll}
\\  \ \ (E\_\text{έπιπλο}) \notin C \\ \\
     
     \end{array} 
\right > \] 
\\ 


Εδώ εννοείται, ότι  οι συντεταγμένες της βάσης του επίπλου , θα είναι της μορφής $(y, z) = (y,0)$ , δηλαδή η $z$ διάσταση θα ισούται πάντα με 0, διότι κάθε έπιπλο δεν μπορεί να αιωρείται.Επίσης το ύψος του επίπλου, πρέπει να είναι μικρότερο ή ίσο με το ύψος του ταβανιού, αν και στο συγκεκριμένο πρόβλημα δεν θα έχουμε πότε τέτοιο θέμα διότι τα δεδομένα έπιπλα δεν είναι “ψηλά”,αλλά θα είχε τεθεί ζήτημα, μόνο αν είχαμε έπιπλα όπως π.χ. ντουλάπα , βιβλιοθήκη, κουκέτα, κ.λ.π. \\ \\  
Έχοντας αναπαραστήσει επιτυχώς το πρόβλημα ως ένα $CSP$ πρόβλημα, μία λύση που δίνει είναι: 
\begin{figure}[H]
  \centering
  \includegraphics[width=.7\textwidth]{con2.png}
  \caption{Φοιτητικό Δωμάτιο - Διακοσμημένο}
  \label{con2.png}
\end{figure} 

\section*{Πρόβλημα 3}
\vspace{5mm}

\hspace{5mm}$\bf 1. \normalfont$ \\ \\
Το πρόβλημα του χρονοπρογραμματισμού, μπορεί να επιλυθεί, με τεχνικές των προβλημάτων ικανοποίησης περιορισμών. \\ \\
- \  \underline{Μεταβλητές}:  \\

Κάθε μία από τις πέντε ενέργειες μπορεί να μοντελοποιηθεί ως μεταβλητή. 
\\ \\
\[  X =
\left\{ 
\begin{array}{ll}
\\
      A_{1} \ , \ A_{2} \ , \  A_{3} \ , \ A_{4} \ , \ A_{5}
     \\ \\
     \end{array} 
\right \} \]

Το \bf Χ \normalfont αναπαραστεί το σύνολο των μεταβλητών. \\ \\ \\
- \ \underline{Πεδίο}:  \\

Όλες οι μεταβλητές έχουν τα ίδια πεδία, ίδιου μεγέθους.Επομένως γενικά το πεδίο της κάθε ενέργειας είναι:\\ \\
\[  D\_\text{ενέργειας} =
\left [
\begin{array}{ll}
\\
\ \ 9: 00 \ , \ 10:00 \ , \ 11:00 \ \
     \\ \\
     \end{array} 
\right ] \] \\ \\
- \ \underline{Περιορισμοί}: \\

Αναπαράσταση Περιορισμών Προτεραιότητας:

\vspace{5mm}

\hspace{5mm}$\bullet$ \ Η $A_{1}$ πρέπει να αρχίσει μετά την $A_{3}$. \\
\[ 
\left<   (A_{1}, A_{3}) , 
\begin{array}{ll}
\\  \ \ A_{1} > A_{3}\\ \\
     
     \end{array} 
\right > \] 
\\ 

\vspace{5mm}

\hspace{5mm}$\bullet$ \ Η $A_{3}$ πρέπει να αρχίσει πριν την $A_{4}$ και μετά την $A_{5}$. \\
\[ 
\left<   (A_{3}, A_{4}, A_{5}) , 
\begin{array}{ll}
\\  \ \ A_{3} < A_{4} \ \text{και} \ \ A_{3} > A_{5} \\ \\
     
     \end{array} 
\right > \] 
\\ 

\vspace{5mm}

\hspace{5mm}$\bullet$ \ Η $A_{2}$ δεν μπορεί να εκτελείται την ίδια ώρα με την $A_{1}$ ή την $A_{4}$. \\
\[ 
\left<   (A_{1}, A_{2}, A_{4}) , 
\begin{array}{ll}
\\  \ \ (\  A_{1} < A_{2} \ \ \text{ή} \ \ A_{1} > A_{2} \ ) \ \text{και} \ ( \ A_{2} < A_{4} \ \ \text{ή} \ \ A_{2} > A_{4} \ ) \\ \\
     
     \end{array} 
\right > \] 
\\ 

\vspace{5mm}

\hspace{5mm}$\bullet$ \ Η $A_{4}$ δεν μπορεί να αρχίσει στις 10:00.\\
\[ 
\left<   (A_{4}) , 
\begin{array}{ll}
\\  \ \  A_{4} \ \neq  \ 10:00  \\ \\
     
     \end{array} 
\right > \] 

\vspace{5mm}

\hspace{5mm}$\bf 2. \normalfont$ \\ \\ 
Το πρόβλημα του χρονοπρογραμματισμού ως γράφος περιορισμών, αναπαρίσταται ως εξής: 

\vspace{7mm}

\begin{figure}[H]
    \includegraphics[width=\linewidth, height=.25\textheight,keepaspectratio=true]{Problem3_graph.png}\\
    \caption{Γράφος Περιορισμών - Πρόβλημα Χρονοπρογραμματισμού}

\end{figure}  

\vspace{10mm}

\hspace{5mm}$\bf 3. \normalfont$ \\ \\
(Εφαρμογή διαφανειών φροντιστηρίου) \\
Γνωρίζουμε ότι ο αλγόριθμος $AC-3$, για να πετύχει την συνέπεια τόξου όλων των μεταβλητών, χρησιμοποιεί μια ουρά, η οποία έχει τα τόξα προς εξέταση. Στο πρόβλημα του χρονοπρογραμματισμού, η ουρά αυτή, στην αρχή έχει τα εξής στοιχεία:\\ \\
\[ Queue =
\left | 
\begin{array}{l|l|l|l|l|l|l|l|l|l}
 \hline  & & & & & & & & & \\  (A_{1},A_{2}) &
 (A_{1},A_{3}) & 
 (A_{2},A_{1}) &
 (A_{2},A_{4}) &  
 (A_{3},A_{1}) &  
 (A_{3},A_{4}) & 
 (A_{3},A_{5}) & 
 (A_{4},A_{2}) &
 (A_{4},A_{3}) &
 (A_{5},A_{3}) \\
& & & & & & & & & \\  
\hline  
     \end{array} 
\right | \] \\ \\
$\circ \ $ Η σειρά με την οποία θα ανατεθούν στις μεταβλητές τιμές είναι η εξής: $A_{1},A_{2}, A_{3},A_{4},A_{5}.$ \\ \\
$\circ \ $ Η σειρά επιλογής των τιμών είναι η εξής: 9:00, 10:00, 11:00 \\ \\
Αρχικά όλες οι ακμές είναι συνεπείς. \\

$\circ \ A_{1} \ = \ 9:00 $ \\

Περιορισμοί που αφορούν το $A_{1} : \ \ \ \ \ A_{1} > A_{3}, \ \ (A_{1} > A_{2} \ \ $ και \ \ $ A_{1} < A_{2} \ \  \equiv \ \ A_{1} \neq A_{2} )$\\

Επομένως έχουμε: 

$ A_{1} \rightarrow \ 9:00:  \ \ \ \ \ \ \ \ 
D_{A_{2}} = \{ 10:00, 11:00\}$ \\

 
\hspace{32mm}$D_{A_{3}} = \{ \} $ \\ \\


Παρατηρούμε το πεδίο τιμών της $A_{3}$ μένει κενό, άρα ο αλγόριθμος $AC-3$ εντόπισε ασυνέπεια.Για τον λόγο αυτό

υπαναχωρούμε και δοκιμάζουμε άλλη τιμή. \\ \\

$\circ \ A_{1} \ = \ 10:00 $ \\

Περιορισμοί που αφορούν το $A_{1} : \ \ \ \ \ A_{1} > A_{3}, \ \ A_{1} \neq A_{2}$ \\

Επομένως έχουμε: 

$ A_{1} \rightarrow \ 10:00:  \ \ \ \ \ \ \ \ 
D_{A_{2}} = \{ 9:00, 11:00\}$ \\

 
\hspace{33mm}$D_{A_{3}} = \{9:00 \} $ \\ \\

Για τον λόγο ότι άλλαξαν τα πεδία της $A_{2}$ και της $A_{3}$, επηρεάζονται και τα πεδία των γειτόνων τους.Πιο συγκεκριμένα: \\

Ελέγχονται οι περιορισμοί: $A_{4} > A_{3}, \ \ A_{5} < A_{3}, \ \  A_{2} \neq A_{4}$ 
Επομένως έχουμε: 

\[
D_{A_{4}} = \{ 10:00, 11:00\}
\]
\[ D_{A_{5}} = \{ \} \]

Ο $AC-3$ εντόπισε ξανά ασυνέπεια, καθώς το πεδίο της $A_{5}$ είναι κενό.Έτσι εξετάζουμε $A_{1} \ = \ 11:00$ \\ \\

Περιορισμοί: \\ \\

$\circ \  A_{1}: \ \ \ \ \ A_{1} > A_{3}, \ \ A_{1} \neq A_{2}$ \\

$\circ \  A_{2}: \ \ \ \ \ A_{2} \neq A_{1}, \ \ A_{2} \neq A_{4}$ \\

$\circ \  A_{3}: \ \ \ \ \ A_{1} > A_{3}, \ \ A_{3} < A_{4}, \ \  A_{3} > A_{5}$ \\

$\circ \  A_{4}: \ \ \ \ \ A_{4} \neq 10:00, \ \ A_{2} \neq A_{4},  \ \ A_{3} < A_{4}$ \\

$\circ \  A_{5}: \ \ \ \ \ A_{3} > A_{5}$ \\\\ \\ \\ \\ 

Άρα: \\ \\

$D_{A_{1}} = \{ 11:00 \} $\\ \\

$ D_{A_{2}} = \{ 9:00, 10:00 \}$ \\ \\

$ D_{A_{3}} = \{ 10:00 \}$ \\ \\

$ D_{A_{4}} = \{ 11:00 \}$ \\ \\

$ D_{A_{5}} = \{ 9:00 \}$ \\ \\

Μέχρι στιγμής είναι συνεπής τα εξής τόξα: $(A_{4},A_{2}),\  (A_{3},A_{4}),\ (A_{4},A_{3}), (A_{5},A_{3}), \ (A_{3},A_{5}).$ \\ \\

$A_{2} \ = \ 9:00$ \\ \\

Περιορισμοί:   $ A_{2} \neq A_{4} $
\\ \\
Ο περιρισμός πάλι ικανοποιείται επομένως και το τόξο  $(A_{2},A_{4})$ είναι συνεπής. Επομένως η διαδικασία διακόπτεται διότι σε όλες τις μεταβλητές έχουν ανατεθεί τιμές. \\ \\
Τα αποτελέσματα: \\
$D_{A_{1}} = 11:00 $\\ \\
$ D_{A_{2}} = 9:00$ \\ \\
$ D_{A_{3}} = 10:00 $ \\ \\
$ D_{A_{4}} = 11:00 $ \\ \\
$ D_{A_{5}} = 9:00$ \\ \\


\section*{Πρόβλημα 4}

\vspace{5mm}

Οι πίνακες αλήθειας των προτάσεων του προβλήματος:\\ \\
Σύμφωνα με το βιβλίο σελ. 253 η προτεραιότητα των τελεστών είναι η εξής: $ \ \neg, \ \wedge, \ \vee, \ \Rightarrow, \ \Leftrightarrow$ 

\vspace{5mm}

\[
\bullet \ \ \  \neg \ ( \ A \ \wedge \ \neg B \ \wedge \  C \ \Rightarrow \ D \ )
\ \Leftrightarrow \ (\ \neg A \ \Rightarrow \ ( \ B \Rightarrow \ ( \ C \ \Rightarrow \ D \ ))) \] \\
Ο πίνακας είναι στην επόμενη σελίδα!!!

\[ 
\left | 
\begin{array}{l|l|l|l|l|l|l|l|l}
 \hline  & & & & & & & &  \\
  \bf  A \ & \bf  B \ & \bf  C \ & \bf  D \ & \bf \ \neg B \ & \bf  ( \ A \ \wedge \ \neg B \ )  & \bf  ( \ A \ \wedge \ \neg B \ \wedge \  C \ )  &  \bf  ( \ A \ \wedge \ \neg B \ \wedge \  C \ \Rightarrow \ D \ )  & \bf  \neg \ ( \ A \ \wedge \ \neg B \ \wedge \  C \ \Rightarrow \ D \ )  \\
  & & & & & & & &   \\  
  \hline \hline & & & & & & & & \\
  %-------------------------- 1 
  T & T & T & T & \ F & \hspace{8mm} F & \hspace{12mm}F & \hspace{18mm}T & \hspace{20mm} F\\ 
  & & & & & & & & \\
  \hline & & & & & & & & \\
  %-------------------------- 2 
  T & T & T & F & \ F & \hspace{8mm} F & \hspace{12mm}F & \hspace{18mm}T & \hspace{20mm} F\\ 
  & & & & & & & & \\
  \hline & & & & & & & & \\
  %-------------------------- 3
  T & T & F & T & \ F & \hspace{8mm} F & \hspace{12mm}F & \hspace{18mm}T & \hspace{20mm} F\\ 
  & & & & & & & & \\
  \hline & & & & & & & & \\
  %-------------------------- 4
  T & T & F & F & \ F & \hspace{8mm} F & \hspace{12mm}F & \hspace{18mm}T & \hspace{20mm} F\\ 
  & & & & & & & & \\
  \hline & & & & & & & & \\
  %-------------------------- 5
  T & F & T & T & \ T & \hspace{8mm} T & \hspace{12mm} T & \hspace{18mm}T & \hspace{20mm} F\\ 
  & & & & & & & & \\
  \hline & & & & & & & & \\
  %-------------------------- 6
  T & F & T & F & \ T & \hspace{8mm} T & \hspace{12mm}T & \hspace{18mm}F & \hspace{20mm} T\\ 
  & & & & & & & & \\
  \hline & & & & & & & & \\
  %-------------------------- 7
  T & F & F & T & \ T & \hspace{8mm} T & \hspace{12mm}F & \hspace{18mm}T & \hspace{20mm} F\\ 
  & & & & & & & & \\
  \hline & & & & & & & & \\
  %-------------------------- 8
  T & F & F & F & \ T & \hspace{8mm} T & \hspace{12mm}F & \hspace{18mm}T & \hspace{20mm} F \\ 
  & & & & & & & & \\
  \hline & & & & & & & & \\
  %-------------------------- 9
  F & T & T & T & \ F & \hspace{8mm} F & \hspace{12mm}F & \hspace{18mm}T & \hspace{20mm} F\\ 
  & & & & & & & & \\
  \hline & & & & & & & & \\
  %-------------------------- 10
  F & T & T & F & \ F & \hspace{8mm} F & \hspace{12mm}F & \hspace{18mm}T & \hspace{20mm} F\\ 
  & & & & & & & & \\
  \hline & & & & & & & & \\
  %-------------------------- 11
  F & T & F & T & \ F & \hspace{8mm} F & \hspace{12mm}F & \hspace{18mm}T & \hspace{20mm} F\\ 
  & & & & & & & & \\
  \hline & & & & & & & & \\
  %-------------------------- 12
  F & T & F & F & \ F & \hspace{8mm} F & \hspace{12mm}F & \hspace{18mm}T & \hspace{20mm} F\\ 
  & & & & & & & & \\
  \hline & & & & & & & & \\
  %-------------------------- 13
  F & F & T & T & \ T & \hspace{8mm} F & \hspace{12mm}F & \hspace{18mm}T & \hspace{20mm} F\\ 
  & & & & & & & & \\
  \hline & & & & & & & & \\
  %-------------------------- 14
  F & F & T & F & \ T & \hspace{8mm} F & \hspace{12mm}F & \hspace{18mm}T & \hspace{20mm} F \\ 
  & & & & & & & & \\
  \hline & & & & & & & & \\
  %-------------------------- 15
  F & F & F & T & \ T & \hspace{8mm} F & \hspace{12mm}F & \hspace{18mm}T & \hspace{20mm} F\\ 
  & & & & & & & & \\
  \hline & & & & & & & & \\
  %-------------------------- 16
  F & F & F & F & \ T & \hspace{8mm} F & \hspace{12mm}F & \hspace{18mm}T & \hspace{20mm} F\\ 
  & & & & & & & & \\
  \hline 
     \end{array} 
\right | \] 

\[ 
\left | 
\begin{array}{l|l|l|l}
 \hline  & & & \\
\bf \neg A \ & \bf ( \ C \ \Rightarrow \ D \ ) \ & \bf ( \ B \Rightarrow \ ( \ C \ \Rightarrow \ D \ )) & \bf \ (\ \neg A \ \Rightarrow \ ( \ B \Rightarrow \ ( \ C \ \Rightarrow \ D \ ))) \  \\
& & &  \\  
 \hline \hline & & & \\
  %-------------------------- 1 
  \ F & \hspace{8mm} T & \hspace{14mm} T & \hspace{24mm} T\\ 
  & & &  \\
  \hline & & &  \\ 
  %-------------------------- 2 
  \ F & \hspace{8mm} F & \hspace{14mm} F & \hspace{24mm} T\\ 
  & & &  \\
  \hline & & &  \\
  %-------------------------- 3   
  \ F & \hspace{8mm} T & \hspace{14mm} T & \hspace{24mm} T\\ 
  & & &  \\
  \hline & & &  \\
  %-------------------------- 4 
  \ F & \hspace{8mm} T & \hspace{14mm} T & \hspace{24mm} T\\ 
  & & &  \\
  \hline & & &  \\
  %-------------------------- 5 
  \ F & \hspace{8mm} T & \hspace{14mm} T & \hspace{24mm} T\\ 
  & & &  \\
  \hline & & &  \\
  %-------------------------- 6 
  \ F & \hspace{8mm} F & \hspace{14mm} T & \hspace{24mm} T\\ 
  & & &  \\
  \hline & & &  \\
  %-------------------------- 7 
  \ F & \hspace{8mm} T & \hspace{14mm} T & \hspace{24mm} T\\ 
  & & &  \\
  \hline & & &  \\
  %-------------------------- 8
  \ F & \hspace{8mm} T & \hspace{14mm} T & \hspace{24mm} T\\ 
  & & &  \\
  \hline & & &  \\
  %-------------------------- 9 
  \ T & \hspace{8mm} T & \hspace{14mm} T & \hspace{24mm} T\\ 
  & & &  \\
  \hline & & &  \\
  %-------------------------- 10 
  \ T & \hspace{8mm} F & \hspace{14mm} F & \hspace{24mm} F\\ 
  & & &  \\
  \hline & & &  \\
  %-------------------------- 11 
  \ T & \hspace{8mm} T & \hspace{14mm} T & \hspace{24mm} T\\ 
  & & &  \\
  \hline & & &  \\
  %-------------------------- 12 
  \ T & \hspace{8mm} T & \hspace{14mm} T & \hspace{24mm} T\\ 
  & & &  \\
  \hline & & &  \\
  %-------------------------- 13 
  \ T & \hspace{8mm} T & \hspace{14mm} T & \hspace{24mm} T\\ 
  & & &  \\
  \hline & & &  \\
  %-------------------------- 14 
  \ T & \hspace{8mm} F & \hspace{14mm} T & \hspace{24mm} T\\ 
  & & &  \\
  \hline & & &  \\
  %-------------------------- 15 
  \ T & \hspace{8mm} T & \hspace{14mm} T & \hspace{24mm} T\\ 
  & & &  \\
  \hline & & &  \\
  %-------------------------- 16 
  \ T & \hspace{8mm} T & \hspace{14mm} T & \hspace{24mm} T\\ 
  & & &  \\
  \hline
     \end{array} 
\right | \] 
\underline{\it Τελικός πίνακας αλήθειας:}
\[ 
\left | 
\begin{array}{l}
 \hline  \\
 \bf \neg \ ( \ A \ \wedge \ \neg B \ \wedge \  C \ \Rightarrow \ D \ )
\ \Leftrightarrow \ (\ \neg A \ \Rightarrow \ ( \ B \Rightarrow \ ( \ C \ \Rightarrow \ D \ )))
\\  \\ 
\hline \\
\hspace{48mm}F \\
\hline \\
\hspace{48mm}F \\
\hline \\
\hspace{48mm}F \\
\hline \\
\hspace{48mm}F \\
\hline \\
\hspace{48mm}F \\
\hline \\
\hspace{48mm}T \\
\hline \\
\hspace{48mm}F \\
\hline \\
\hspace{48mm}F \\
\hline \\
\hspace{48mm}F \\
\hline \\
\hspace{48mm}T \\
\hline \\
\hspace{48mm}F \\
\hline \\
\hspace{48mm}F \\
\hline \\
\hspace{48mm}F \\
\hline \\
\hspace{48mm}F \\
\hline \\
\hspace{48mm}F \\
\hline \\
\hspace{48mm}F \\
\hline
\end{array} 
\right | \] 
\\ \\
% ======================================================
% 
% ======================================================

\[
\bullet \ \ \  \neg A \ \wedge ( \ \neg A \ \Rightarrow \  B \ ) \wedge ( \ A \ \Rightarrow \  \neg B \ ) \]


\[ 
\left | 
\begin{array}{l|l|l|l|l}
 \hline  & & & &   \\
  \bf  A \ & \bf  B \ & \bf \neg A \ & \bf ( \ \neg A \ \Rightarrow \  B \ ) & \bf  \neg A \ \wedge ( \ \neg A \ \Rightarrow \  B \ )   \\
  & & & &    \\  
  \hline \hline & & & &  \\
  %-------------------------- 1 
  T & T & F & T & \ F \\ 
  & & & &  \\
  \hline & & & &\\
  %-------------------------- 2 
  T & F & F & T & \ F \\ 
  & & & &  \\
  \hline & & & &\\
  %-------------------------- 3
  F & T & T & T & \ T \\ 
  & & & &  \\
  \hline & & & &\\
  %-------------------------- 4
  F & F & T & F & \ F \\ 
  & & & &  \\
  \hline 
     \end{array} 
\right |  
\ \ \ \ \ \ \ \ \ 
\left | 
\begin{array}{l|l}
 \hline  &  \\
\neg B  & ( \ A \ \Rightarrow \  \neg B \ \ )\\
\\  
 \hline \hline &  \\
  %-------------------------- 1 
  \ F & \hspace{8mm} F \\ 
    &   \\
  \hline & \\
  %-------------------------- 2 
  \ T & \hspace{8mm} T \\ 
    &   \\
  \hline & \\ 
  %-------------------------- 3 
  \ F & \hspace{8mm} T \\ 
    &   \\
  \hline & \\
  %-------------------------- 4
  \ T & \hspace{8mm} T \\ 
    &   \\
  \hline
     \end{array} 
\right | \] 
\underline{\it Τελικός πίνακας αλήθειας:}
\[ 
\left | 
\begin{array}{l}
 \hline \\
 \bf \neg A \ \wedge ( \ \neg A \ \Rightarrow \  B \ ) \wedge ( \ A \ \Rightarrow \  \neg B \ )
\\  \\ 
\hline \hline  \\
\hspace{28mm}F \\
\hline \\
\hspace{28mm}F \\
\hline \\
\hspace{28mm}T \\
\hline \\
\hspace{28mm}F \\
\hline
\end{array} 
\right | \] 

% ==========================================


\vspace{5mm}

\[
\bullet \ \ \ ( \ A \ \vee \ \neg B \ ) \ \wedge \ ( A \ 
\vee \ \neg C \ ) \ \wedge \ \neg B \ \wedge \ \neg C \]


\[ 
\left | 
\begin{array}{l|l|l|l|l|l|l}
 \hline  & & & & & &  \\
  \bf  A \ & \bf  B \ & \bf  C \ & \bf \neg B & \bf ( \ A \ \vee \ \neg B \ ) & \bf \neg C & \bf ( A \ 
\vee \ \neg C \ ) \\
  & & & & & &   \\  
  \hline \hline & & & & & & \\
  %-------------------------- 1 
  T & T & T & \ F & \hspace{8mm} T & \ F & \hspace{8mm} T\\ 
  & & & & & & \\
  \hline & & & & & & \\
  %-------------------------- 2 
  T & T & F & \ F & \hspace{8mm} T & \ T & \hspace{8mm} T\\ 
  & & & & & & \\
  \hline & & & & & & \\
  %-------------------------- 3 
  T & F & T & \ T & \hspace{8mm} T & \ F & \hspace{8mm} T\\ 
  & & & & & & \\
  \hline & & & & & & \\
  %-------------------------- 4 
  T & F & F & \ T & \hspace{8mm} T & \ T & \hspace{8mm} T\\ 
  & & & & & & \\
  \hline & & & & & & \\
  %-------------------------- 5 
  F & T & T & \ F & \hspace{8mm} F & \ F & \hspace{8mm} F\\ 
  & & & & & & \\
  \hline & & & & & & \\
  %-------------------------- 6
  F & T & F & \ F & \hspace{8mm} F & \ T & \hspace{8mm} T\\ 
  & & & & & & \\
  \hline & & & & & & \\
  %-------------------------- 7 
  F & F & T &  \ T & \hspace{8mm} T & \ F & \hspace{8mm} F\\ 
  & & & & & & \\
  \hline & & & & & & \\
  %-------------------------- 8 
  F & F & F & \ T & \hspace{8mm} T & \ T & \hspace{8mm} T\\ 
  & & & & & & \\
  \hline 
     \end{array} 
\right | \] 
\\ \\ 
\[ 
\left | 
\begin{array}{l|l}
 \hline  & \\ \bf
( \ A \ \vee \ \neg B \ ) \ \wedge \ ( A \ 
\vee \ \neg C \ ) \ &  \bf ( \ A \ \vee \ \neg B \ ) \ \wedge \ ( A \ 
\vee \ \neg C \ ) \ \wedge \ \neg B \  \\
&  \\  
 \hline \hline &  \\
  %-------------------------- 1 
  \hspace{25mm} T & \hspace{32mm} F \\ 
  &  \\
  \hline &  \\ 
  %-------------------------- 2 
  \hspace{25mm} T & \hspace{32mm} F \\ 
  &  \\
  \hline &  \\ 
  %-------------------------- 3 
  \hspace{25mm} T & \hspace{32mm} T \\ 
  &  \\
  \hline &  \\ 
  %-------------------------- 4
  \hspace{25mm} T & \hspace{32mm} T \\ 
  &  \\
  \hline &  \\ 
  %-------------------------- 5 
  \hspace{25mm} F & \hspace{32mm} F \\ 
  &  \\
  \hline &  \\ 
  %-------------------------- 6 
  \hspace{25mm} F & \hspace{32mm} F \\ 
  &  \\
  \hline &  \\ 
  %-------------------------- 7 
  \hspace{25mm} F & \hspace{32mm} F \\ 
  &  \\
  \hline &  \\
  %-------------------------- 8 
  \hspace{25mm} T & \hspace{32mm} T \\ 
  &  \\
  \hline
     \end{array} 
\right | \] 
\underline{\it Τελικός πίνακας αλήθειας:}
\[ 
\left | 
\begin{array}{l}
 \hline  \\
 \bf  ( \ A \ \vee \ \neg B \ ) \ \wedge \ ( A \ 
\vee \ \neg C \ ) \ \wedge \ \neg B \ \wedge \ \neg C
\\  \\ 
\hline \\
\hspace{32mm}F \\
\hline \\
\hspace{32mm}F \\
\hline \\
\hspace{32mm}F \\
\hline \\
\hspace{32mm}T \\
\hline \\
\hspace{32mm}F \\
\hline \\
\hspace{32mm}F \\
\hline \\
\hspace{32mm}F \\
\hline \\
\hspace{32mm}T \\
\hline
\end{array} 
\right | \] 
\\ \\
% ======================================================
% 
% ======================================================
% ==========================================


\vspace{5mm}

\[
\bullet \ \ \ ( \ A \ \vee \  B \ ) \ \wedge \ ( \ \neg A \ 
\vee \ \neg C \ ) \ \wedge \ ( \ B \ \vee \ \neg C \ ) \]


\[ 
\left | 
\begin{array}{l|l|l|l|l|l|l|l|l}
 \hline  & & & & & & & &  \\
  \bf  A \ & \bf  B \ & \bf  C \ & \bf ( \ A \ \vee \ B \ ) & \bf \neg A & \bf \neg C & \bf ( \ \neg A \ \vee \ \neg C \ ) & \bf
( \ A \ \vee \ B \ ) \ \wedge \ ( \ \neg A \ 
\vee \ \neg C \ ) \ &  \bf ( \ B \ \vee \ \neg C \ ) \\
  & & & & & & & &\\  
  \hline \hline & & & & & & & & \\
  %-------------------------- 1 
  T & T & T & \hspace{8mm} T & \ F & \ F & \hspace{10mm} F & \hspace{25mm} F & \hspace{8mm} T\\ 
  & & & & & & & &\\
  \hline & &  & & & & & &  \\
  %-------------------------- 2 
  T & T & F & \hspace{8mm} T & \ F & \ T & \hspace{10mm} T & \hspace{25mm} T & \hspace{8mm} T \\ 
  & & & & & & & & \\
  \hline & & & & & & & &  \\
  %-------------------------- 3 
  T & F & T & \hspace{8mm} T & \ F & \ F & \hspace{10mm} F & \hspace{25mm} F & \hspace{8mm} F\\ 
  & & & & & & & & \\
  \hline & & & & & & & & \\
  %-------------------------- 4 
  T & F & F & \hspace{8mm} T & \ F & \ T & \hspace{10mm} T & \hspace{25mm} T & \hspace{8mm} T\\ 
  & & & & & & & & \\
  \hline & & & & & & & & \\
  %-------------------------- 5
  F & T & T & \hspace{8mm} T & \ T & \ F & \hspace{10mm} T & \hspace{25mm} T & \hspace{8mm} T\\ 
  & & & & & & & &\\
  \hline & & & & & & & & \\
  %-------------------------- 6
  F & T & F & \hspace{8mm} T & \ T & \ T & \hspace{10mm} T & \hspace{25mm} T & \hspace{8mm} T\\ 
  & & & & & & & &\\
  \hline & & & & & & & & \\
  %-------------------------- 7 
  F & F & T & \hspace{8mm} F & \ T & \ F & \hspace{10mm} T & \hspace{25mm} F & \hspace{8mm} F\\ 
  & & & & & & & & \\
  \hline & & & & & & & & \\
  %-------------------------- 8
  F & F & F & \hspace{8mm} F & \ T & \ T & \hspace{10mm} T & \hspace{25mm} F & \hspace{8mm} T\\ 
  & & & & & & & & \\
  \hline 
     \end{array} 
\right | \] 
\\ \\ 

\underline{\it Τελικός πίνακας αλήθειας:}
\[ 
\left | 
\begin{array}{l}
 \hline  \\
 \bf   ( \ A \ \vee \  B \ ) \ \wedge \ ( \ \neg A \ 
\vee \ \neg C \ ) \ \wedge \ ( \ B \ \vee \ \neg C \ )
\\  \\ 
\hline \\
\hspace{35mm}F \\
\hline \\
\hspace{35mm}T \\
\hline \\
\hspace{35mm}F \\
\hline \\
\hspace{35mm}T \\
\hline \\
\hspace{35mm}T \\
\hline \\
\hspace{35mm}T \\
\hline \\
\hspace{35mm}F \\
\hline \\
\hspace{35mm}F \\
\hline
\end{array} 
\right | \] 
\\ \\
% ======================================================
% 
% ======================================================

\[
\bullet \ \ y \ = \  \neg \ ( \ A \ \wedge \ \neg B \ \wedge \  C \ \Rightarrow \ D \ )
\ \Leftrightarrow \ (\ \neg A \ \Rightarrow \ ( \ B \Rightarrow \ ( \ C \ \Rightarrow \ D \ ))) \]

1. Η πρόταση $y$ δεν είναι έγκυρη, διότι υπάρχει τουλάχιστον μία αποτίμηση $V$ που είναι ψευδής, ισχύει $V(y$)=Ψευδές \\

2. Η πρόταση $y$ είναι ικανοποιήσιμη , διότι υπάρχει αποτίμηση  που την ικανοποιεί\\

3. Η πρόταση $y$ δεν είναι μη ικανοποιήσιμη, διότι είναι ικανοποιήσιμη. \\

4. Η πρόταση $y$ έχει τουλάχιστον ένα μοντέλο αφού είναι ικανοποιήσιμη. \\

5. Η πρόταση $y$ δεν είναι ταυτολογία, καθώς δεν είναι έγκυρη
\\ \\

\[
\bullet \ \ y \ = \  \neg A \ \wedge ( \ \neg A \ \Rightarrow \  B \ ) \wedge ( \ A \ \Rightarrow \  \neg B \ ) \]

1. Η πρόταση $y$ δεν είναι έγκυρη, διότι υπάρχει τουλάχιστον μία αποτίμηση $V$ που είναι ψευδής, ισχύει $V(y$)=Ψευδές \\

2. Η πρόταση $y$ είναι ικανοποιήσιμη , διότι υπάρχει αποτίμηση  που την ικανοποιεί\\

3. Η πρόταση $y$ δεν είναι μη ικανοποιήσιμη, διότι είναι ικανοποιήσιμη. \\

4. Η πρόταση $y$ έχει τουλάχιστον ένα μοντέλο αφού είναι ικανοποιήσιμη. \\

5. Η πρόταση $y$ δεν είναι ταυτολογία, καθώς δεν είναι έγκυρη
\\ \\

\[
\bullet \ \ y \ = \ ( \ A \ \vee \ \neg B \ ) \ \wedge \ ( A \ 
\vee \ \neg C \ ) \ \wedge \ \neg B \ \wedge \ \neg C \]

1. Η πρόταση $y$ δεν είναι έγκυρη, διότι υπάρχει τουλάχιστον μία αποτίμηση $V$ που είναι ψευδής, ισχύει $V(y$)=Ψευδές \\

2. Η πρόταση $y$ είναι ικανοποιήσιμη , διότι υπάρχει αποτίμηση  που την ικανοποιεί\\

3. Η πρόταση $y$ δεν είναι μη ικανοποιήσιμη, διότι είναι ικανοποιήσιμη. \\

4. Η πρόταση $y$ έχει τουλάχιστον ένα μοντέλο αφού είναι ικανοποιήσιμη. \\

5. Η πρόταση $y$ δεν είναι ταυτολογία, καθώς δεν είναι έγκυρη
\\ \\

\[
\bullet \ \ y \ = \ ( \ A \ \vee \  B \ ) \ \wedge \ ( \ \neg A \ 
\vee \ \neg C \ ) \ \wedge \ ( \ B \ \vee \ \neg C \ ) \]


1. Η πρόταση $y$ δεν είναι έγκυρη, διότι υπάρχει τουλάχιστον μία αποτίμηση $V$ που είναι ψευδής, ισχύει $V(y$)=Ψευδές \\

2. Η πρόταση $y$ είναι ικανοποιήσιμη , διότι υπάρχει αποτίμηση  που την ικανοποιεί\\

3. Η πρόταση $y$ δεν είναι μη ικανοποιήσιμη, διότι είναι ικανοποιήσιμη. \\

4. Η πρόταση $y$ έχει τουλάχιστον ένα μοντέλο αφού είναι ικανοποιήσιμη. \\

5. Η πρόταση $y$ δεν είναι ταυτολογία, καθώς δεν είναι έγκυρη
\\ \\

\section*{Πρόβλημα 5}

\vspace{5mm}

Έστω: \\

\hspace{20mm}φ: ο Σήφης είναι καλός μάγειρας \ \ \ \bf  και \normalfont \ \ \  ψ: είμαι αστροναύτης\\ 

Στην προτασιακή λογική η πρόταση θα μετατρεποταν σε μία πρόταση συνεπαγωγής, ως εξής:

\[ \text{φ} \ \Rightarrow \ \text{ψ} \] \\

Με την πρόταση  $\text{φ} \ \Rightarrow \ \text{ψ}$, λέμε ότι αν η φ, δηλαδή  αν ο Σήφης είναι καλός μάγειρας είναι αληθής, τότε ισχυριζόμαστε ότι και αυτός που λέει την πρόταση είναι αστροναύτης.Παρόλο αυτά από την εκφώνηση μας έχει δοθεί ως δεδομένο, ότι αυτός που λέει την πρόταση δεν είναι αστροναυτης.Από την θεωρία γνωρίζουμε ότι μία πρόταση συνεπαγωγής, της προτασιακής  λογικής  είναι ψευδής, μονο αν η φ είναι αληθές και η ψ ψευδές. Άρα με βάση αυτό η πρόταση είναι ψευδής και κατ’ επέκταση ο Σήφης δεν είναι καλός μάγειρας.


\section*{Πρόβλημα 6}

\vspace{5mm}

Για την επίλυση του προβλήματος θα χρησιμοποιήσουμε το συμπέρασμα του βιβλίου (σελ. 258):
\[
\text{α} \ |= ~\text{β αν και μόνο αν η πρόταση \ α} \wedge \neg \text{β \ είναι μη ικανοποιήσιμη.}
\]
Για την απόδειξη της μη ικανοποιησιμότητας της (  α $\wedge \neg $β) θα εφαρμόσουμε την τεχνική της απαγωγής σε άτοπο. \\ \\
Πρώτα απ’ όλα θα μετατρέψουμε τις προτάσεις του συνόλου Α και την πρόταση β, που είναι γραμμένες στην φυσική γλώσσα, σε προτάσεις της προτασιακής λογικής. \\ \\
Έχουμε: \\ 

$\bullet$ \ Το σχήμα α είναι κύβος ή δωδεκάεδρο ή τετράεδρο. \\

\hspace{20mm}$\bf P1: \ \ \normalfont$ K $\oplus$ Δ $\oplus$ Τ    \ \ \ \ \ \ \ \ \ \ (Κ: κύβος, \ Δ: δωδεκάεδρο, \ Τ: τετράεδρο)\\  \\

$\bullet$ \ Το σχήμα α είναι μικρό ή μεσαίο ή μεγάλο. \\ 

\hspace{20mm}$\bf P2: \ \ \normalfont$ Μικρ $\oplus$ Μεσ $\oplus$ Μεγ    \ \ \ \ \ \ \ \ \ \ (Μικρ: μικρό, \ Μεσ: μεσαίο, \ Μεγ: μεγάλο)\\  \\

$\bullet$ \ Το σχήμα α είναι μεσαίο αν και μόνο αν είναι δωδεκάεδρο. \\ 

\hspace{20mm}$\bf P3: \ \ \normalfont$ Μεσ $\Leftrightarrow$ Δ\\  \\

$\bullet$ \ Το σχήμα α είναι τετράεδρο αν και μόνο αν είναι μεγάλο.
 \\ 

\hspace{20mm}$\bf P4: \ \ \normalfont$ Τ $\Leftrightarrow$ Μεγ\\  \\

$\bullet$ \ Το σχήμα α είναι κύβος αν και μόνο αν είναι μικρό. \\ 

\hspace{20mm}$\bf P5: \ \ \normalfont$ Κ $\Leftrightarrow$ Μικρ 

\vspace{10mm}

Οι προτάσεις του προβλήματος θα εκφραστούν ως συζευκτική κανονική μορφή ($CNF$), ώστε να εφαρμόσουμε τον κανόνα της ανάλυσης. \\ \\

Έχουμε: \\ \\

$\rightharpoonup$ \ \bf K $\bigoplus$ Δ $\bigoplus$ Τ \normalfont \\

$\circ$ \ Γνωρίζουμε ότι ισχύει η εξής ισοδυναμία (έχει γραφεί και στο $piazza$):

\[
(p \oplus q ) \equiv (p \vee q) \wedge \neg (p \wedge q)
\] \\

\hspace{5mm}Οπότε με την εφαρμογή της ισοδυναμίας στην πρόταση Ρ1, έχουμε: \\ 

\[ \bf P6: \normalfont \ \ ( (K \vee \text{Δ}) \ \wedge \  \neg(K \wedge \text{Δ})) \ \oplus \ T \]
\\ \\
$\circ$ \ Εφαρμόζουμε άλλη μια φορά την ισοδυναμία στην πρόταση Ρ6:\\

\[ \bf P7: \normalfont \ \ (( (K \vee \text{Δ}) \ \wedge \  \neg(K \wedge \text{Δ})) \ \vee \ T ) \ \wedge \ \neg (((K \vee \text{Δ}) \ \wedge \  \neg(K \wedge \text{Δ})) \ \wedge \ T)\]\\ \\
$\circ$ \ Αντιμεταθετικότητα του $\vee$ και του $\wedge$ αντίστοιχα στην πρόταση Ρ7:\\ \\

\[ \bf P8: \normalfont \ \ (  T \ \vee \ ( (K \vee \text{Δ}) \ \wedge \  \neg(K \wedge \text{Δ}))) \ \wedge \ \neg ( T \ \wedge \ ((K \vee \text{Δ}) \ \wedge \  \neg(K \wedge \text{Δ})))\]\\ \\
$\circ$ \ Επιμεριστικότητα του $\vee$ ως προς το $\wedge$ στην πρόταση Ρ8:\\

\[ \bf P9: \normalfont \ \ ( \  ( T \ \vee \  (K \vee \text{Δ}) ) \ \wedge \ ( T \ \vee  \ (\neg(K \wedge \text{Δ}))) \ ) \ \wedge \ \neg ( T \ \wedge \ ((K \vee \text{Δ}) \ \wedge \  \neg(K \wedge \text{Δ})))\]\\ \\
$\circ$ \ Εφαρμόζουμε τον νόμο του $De \ Morgan$ στην Ρ9: \\

\[ \bf P10: \normalfont \ \ ( \  ( T \ \vee \  (K \vee \text{Δ}) ) \ \wedge \ ( T \ \vee  \ (\neg K \vee\neg \text{Δ}))) \ ) \ \wedge \ (\neg  T \ \vee \ \neg((K \vee \text{Δ}) \ \wedge \  \neg(K \wedge \text{Δ})))\]\\ \\
$\circ$ \ Εφαρμόζουμε ξανά τον νόμο του $De \ Morgan$ στην Ρ10: \\

\[ \bf P11: \normalfont \ \ ( \  ( T \ \vee \  (K \vee \text{Δ}) ) \ \wedge \ ( T \ \vee  \ (\neg K \vee\neg \text{Δ}))) \ ) \ \wedge \ (\neg  T \ \vee \ (\neg(K \vee \text{Δ}) \ \vee \  \neg\neg(K \wedge \text{Δ})))\]\\ \\
$\circ$ \ Εφαρμόζουμε ξανά τον νόμο του $De \ Morgan$ στην Ρ11: \\

\[ \bf P12: \normalfont \ \ ( \  ( T \ \vee \  (K \vee \text{Δ}) ) \ \wedge \ ( T \ \vee  \ (\neg K \vee\neg \text{Δ}))) \ ) \ \wedge \ (\neg  T \ \vee \ ( (\neg K \wedge \ \neg\text{Δ}) \ \vee \  (K \wedge \text{Δ})))\]\\ \\
$\circ$ \ Η λογική ισοδυναμία για την επιμεριστικότητα μας δίνει: \\

\[
\bf P13: \normalfont \ \ ( \  ( T \ \vee \  (K \vee \text{Δ}) ) \ \wedge \ ( T \ \vee  \ (\neg K \vee\neg \text{Δ})) \ )\ \bigwedge \ ( \ \neg T \ \vee \ ( \ (\neg K \vee K) \ \wedge \ (\neg K \vee \text{Δ}) \ \wedge \ (\neg \text{Δ} \vee K) \wedge ( \neg \text{Δ} \vee \text{Δ}) \  ) \ ) \]
\[ 
\equiv  
\]
\[
\bf P13: \normalfont \ \ ( \  ( T \ \vee \  (K \vee \text{Δ}) ) \ \wedge \ ( T \ \vee  \ (\neg K \vee\neg \text{Δ})) \ )\ \bigwedge \ ( \ \neg T \ \vee \ ( \   (\neg K \vee \text{Δ}) \ \wedge \ (\neg \text{Δ} \vee K) \  ) \ ) 
\] \\ \\
$\circ$ \ Εφαρμόζουμε ξανά την λογική ισοδυναμία της επιμεριστικότητας: \\

\[
\bf P14: \normalfont \ \ ( \  ( T \ \vee \  (K \vee \text{Δ}) ) \ \wedge \ ( T \ \vee  \ (\neg K \vee\neg \text{Δ})) \ )\ \bigwedge \ ( \ (\neg T \ \vee \ ( \   \neg K \vee \text{Δ}) \ ) \ \wedge \ (\neg T \ \vee \ (\neg \text{Δ} \vee K) \  ) \ ) 
\] \\ \\ \\ \\ \\ \\ \\ 
$\rightharpoonup$ \ \bf \text{Μικρ} $\bigoplus$ \text{Μεσ} $\bigoplus$ \text{Μεγ} \normalfont \\ \\
Ακολουθούμε τα ίδια βήματα όπως και προηγουμένως στην πρόταση \text{Κ} $\bigoplus$ \text{Δ} $\bigoplus$ \text{Τ}. Επομένως προκύπτει η εξής $CNF$:  \\

\[
\bf P15: \normalfont \ \ ( \  ( \text{Μεγ} \ \vee \  (\text{Μικρ} \vee \text{Μεσ}) ) \ \wedge \ ( \text{Μεγ} \ \vee  \ (\neg \text{Μικρ} \vee\neg \text{Μεσ})) \ )\ \bigwedge \ ( \ (\neg \text{Μεγ} \ \vee \ ( \   \neg \text{Μικρ} \vee \text{Μεσ}) \ ) \ \wedge \ (\neg \text{Μεγ} \ \vee \ (\neg \text{Μεσ} \vee \text{Μικρ}) \  ) \ ) 
\] \\ \\ \\ 
$\rightharpoonup$ \ \bf Μεσ $\Leftrightarrow$ Δ \normalfont \\ \\
$\circ$ \ Εφαρμόζουμε απαλοιφή αμφίδρομης υποθετικής στην πρόταση Ρ3:\\
\[ 
\bf P16: \normalfont \ \ ( \ ( \text{Μεσ}  \Rightarrow  \text{Δ} ) \ \bigwedge \  ( \text{Δ}  \Rightarrow \text{Μεσ} ) \ )
\] \\ \\
$\circ$ \ Εφαρμόζουμε απαλοιφή συνεπαγωγής στην πρόταση Ρ16:\\
\[ 
\bf P17: \normalfont \ \ ( \ ( \neg\text{Μεσ}  \vee  \text{Δ} ) \ \bigwedge \  ( \neg\text{Δ}  \vee \text{Μεσ} ) \ )
\] \\ \\ \\
$\rightharpoonup$ \ \bf Τ $\Leftrightarrow$ Μεγ \normalfont \\ \\
$\circ$ \ Εφαρμόζουμε απαλοιφή αμφίδρομης υποθετικής στην πρόταση Ρ4:\\
\[ 
\bf P18: \normalfont \ \ ( \ ( \text{Τ}  \Rightarrow  \text{Μεγ} ) \ \bigwedge \  ( \text{Μεγ}  \Rightarrow \text{Τ} ) \ )
\] \\ \\
$\circ$ \ Εφαρμόζουμε απαλοιφή συνεπαγωγής στην πρόταση Ρ18:\\
\[ 
\bf P19: \normalfont \ \ ( \ ( \neg\text{Τ}  \vee  \text{Μεγ} ) \ \bigwedge \  ( \neg\text{Μεγ}  \vee \text{Τ} ) \ )
\] \\ \\ \\ 
$\rightharpoonup$ \ \bf Κ $\Leftrightarrow$ Μικρ \normalfont \\ \\
$\circ$ \ Εφαρμόζουμε απαλοιφή αμφίδρομης υποθετικής στην πρόταση Ρ5:\\
\[ 
\bf P20: \normalfont \ \ ( \ ( \text{Κ}  \Rightarrow  \text{Μικρ} ) \ \bigwedge \  ( \text{Μικρ}  \Rightarrow \text{Κ} ) \ )
\] \\ \\
$\circ$ \ Εφαρμόζουμε απαλοιφή συνεπαγωγής στην πρόταση Ρ20:\\
\[ 
\bf P21: \normalfont \ \ ( \ ( \neg\text{Κ}  \vee  \text{Μικρ} ) \ \bigwedge \  ( \neg\text{Μικρ}  \vee \text{Κ} ) \ )
\] \\ \\ \\

Όπως αναφέραμε στην αρχή θα αποδείξουμε την πρόταση $(\text{α} \wedge \neg \text{β})$, επομένως πρέπει να βρούμε την πρόταση $\neg \text{β}$. Έχουμε: \\ \\

Η β είναι η Ρ21 που βρήκαμε προηγουμένως, επομένως:

\[
\neg \text{β} \equiv \ \neg ( \ ( \neg\text{Κ}  \vee  \text{Μικρ} ) \ \bigwedge \  ( \neg\text{Μικρ}  \vee \text{Κ} ) \ )
\]
\[
\hspace{36mm} \equiv \ \neg  ( \neg\text{Κ}  \vee  \text{Μικρ} ) \ \bigvee \  ( \neg  (\neg\text{Μικρ}  \vee \text{Κ} ) \ ) \ \ \ \ \ \ \ ( De \ Morgan)
\]
\[
\hspace{36mm} \equiv \  ( \text{Κ}  \wedge \neg \text{Μικρ} ) \ \bigvee \  (\text{Μικρ}  \wedge \neg \text{Κ}  ) \ \ \ \ \ \ \ \ \ \ \ \ \ \  ( De \ Morgan)
\]
\[
\hspace{36mm} \equiv \  ( \text{Κ}  \vee \text{Μικρ} ) \ \bigwedge \ ( \text{Κ}  \vee \neg \text{Κ}) \ \bigwedge \ (\neg \text{Μικρ} \vee \text{Μικρ})\ \bigwedge \  (\neg \text{Μικρ} \vee \neg \text{Κ})\ \ \ \ \ \ \ \ \ \ \ \ \ \  \text{(επιμεριστικότητα)}
\]
\[
\hspace{36mm} \equiv \  ( \text{Κ}  \vee \text{Μικρ} ) \ \bigwedge \  (\neg \text{Μικρ} \vee \neg \text{Κ})
\]

\vspace{10mm}

Έχουν παραχθεί η εξής φράσεις από το σύνολο Α και από την πρόταση β: \\ \\
$\bf 1. \normalfont \ \ \  ( T \ \vee \  (K \vee \text{Δ}) ) $ \\
$ \\
\bf 2. \normalfont \ \ \ ( T \ \vee  \ (\neg K \vee\neg \text{Δ})) $ \\
$  \\
\bf 3. \normalfont \ \ \  (\neg T \ \vee \ ( \   \neg K \vee \text{Δ}) \ )$ \\
$ \\
\bf 4. \normalfont \ \ \  (\neg T \ \vee \ (\neg \text{Δ} \vee K) \  ) $ \\
$\\ \bf 5. \normalfont \ \ \  ( \text{Μεγ} \ \vee \  (\text{Μικρ} \vee \text{Μεσ}) )$ \\
$ \\
\bf 6. \normalfont \ \ \  ( \text{Μεγ} \ \vee  \ (\neg \text{Μικρ} \vee\neg \text{Μεσ}))\\
$ \\
$
\bf 7. \normalfont \ \ \  (\neg \text{Μεγ} \ \vee \ ( \   \neg \text{Μικρ} \vee \text{Μεσ}) \ )$ \\
$ \\
\bf 8. \normalfont \ \ \ (\neg \text{Μεγ} \ \vee \ (\neg \text{Μεσ} \vee \text{Μικρ}) \  ) $ \\
$ \\ \bf 9. \normalfont \ \ \  ( \neg\text{Μεσ}  \vee  \text{Δ} )$ \\
$\\  \bf 10. \normalfont \ ( \neg\text{Δ}  \vee \text{Μεσ} ) $ \\
$ \\ \bf 11. \normalfont \  ( \neg\text{Τ}  \vee  \text{Μεγ} ) $ \\
$\\ \bf 12. \normalfont \ ( \neg\text{Μεγ}  \vee \text{Τ} ) $ \\
$\\  \bf 13. \normalfont \  ( \text{Κ}  \vee \text{Μικρ} ) $ \\ 
$\\  \bf 14. \normalfont \  (\neg \text{Μικρ} \vee \neg \text{Κ}) $ \\

\vspace{5mm}

Εφαρμόζουμε την αρχή της ανάλυσης στις σχέσεις: \\ \\ \\

$\bullet$ \ (1 \&  4) δύο φορές,  διότι έχουμε δύο συμπληρωματικά ζεύγη ($\neg T, T$) και ($\neg \text{Δ}, \text{Δ}$). \\

\hspace{5mm}Το αναλυθεν είναι: Κ \ (15)  \\ \\ 

$\bullet$ \ (2 \&  3) δύο φορές,  διότι έχουμε δύο συμπληρωματικά ζεύγη ($\neg T, T$) και ($\neg \text{Δ}, \text{Δ}$). \\

\hspace{5mm}Το αναλυθεν είναι: $\neg K$ \ (16) \\ \\

$\bullet$ \ (16 \&  13) έχουμε το συμπληρωματικό ζεύγος ($\neg K, K$). \\

\hspace{5mm}Το αναλυθεν είναι: Μικρ \ (17) \\ \\

$\bullet$ \ (15 \&  14) έχουμε το συμπληρωματικό ζεύγος ($\neg K, K$). \\

\hspace{5mm}Το αναλυθεν είναι: $\neg \text{Μικρ}$ \ (18) \\ \\ 

Εφαρμόζουμε την ανάλυση για μία τελευταία φορά στις  (17 \&  18). \\

\hspace{5mm}Το αναλυθέν είναι η κενή φράση, έτσι έχουμε αποδείξει ότι η $( \text{α} \wedge \neg \text{β} ) $ είναι μη ικανοποιήσιμη, άρα ισχύει  η α $|=$ β.


\vspace{5mm}

\section*{Πρόβλημα 7}

\vspace{5mm}

Για την επίλυση του προβλήματος θα βασιστούμε στα εξής γνωστά από την θεωρία: \\

\hspace{5mm} $\bf \ 1) \normalfont$ \ Το \bf θεώρημα της ισοδυναμίας \normalfont ( διαφάνεια 48 του αρχείου «Διαφάνειες 1» της ενότητας 8 των διαλέξεων):

\[ \text{φ} \equiv  \text{ψ αν και μόνο αν η πρόταση φ} \Leftrightarrow \text{ψ είναι έγκυρη.} \] \\

\hspace{5mm} $\bf \ 2) \normalfont$ \ Οι \bf έγκυρες προτάσεις \normalfont είναι γνωστές και ως \bf ταυτολογίες \normalfont (βιβλίο σελ.257). \\

\hspace{5mm} $\bf \ 3) \normalfont$ \ Το \bf θεώρημα παραγωγής \normalfont ( βιβλίο σελ. 257) :

\[ \text{Για οποιεσδήποτε προτάσεις α και β, α } |\text{= β αν και μόνο αν η πρόταση ( α} \Rightarrow   \text{β ) είναι έγκυρη.}\] \\

\hspace{5mm} $\bf \ 4) \normalfont$ \ Τον \bf ορισμό της ισοδυναμίας \normalfont (βιβλίο σελ. 257 και διαφάνεια 48 του αρχείου «Διαφάνειες 1» της ενότητας

\hspace{13mm}8 των διαλέξεων):

\[
\text{α} \equiv \text{β αν και μόνο αν α } |\text{= β και β } |\text{= α}
\] \\

\hspace{5mm} $\bf \ 5) \normalfont$ \ Το συμπέρασμα του βιβλίου σελ. 258:

\[
\text{α } | \text{= β αν και μόνο αν η πρόταση (α } \wedge \neg\text{β) είναι μη ικανοποιήσιμη. } \] \\
\\ \\
Για να είναι έγκυρη η πρόταση \ $\neg(A \Leftrightarrow B) \Leftrightarrow \neg((A \wedge B) \vee (\neg A \wedge \neg B)) $ \  πρέπει να είναι έγκυρη σε όλα τα μοντέλα.Για να δείξουμε αυτό, αρχικά θα αποδείξουμε ότι η \ $\neg(A \Leftrightarrow B) \Leftrightarrow \neg((A \wedge B) \vee (\neg A \wedge \neg B)) $ \  είναι μια ταυτολογία, το οποίο θα αποδειχθεί με την χρήση του κανόνα ανάλυσης, μέσω του οποίου θα δείξουμε ότι ισχύει: \\
\[ \neg(A \Leftrightarrow B) \ \ |= \ \neg((A \wedge B) \vee (\neg A \wedge \neg B)) \ \ \ \textbf{και} \ \ \ \neg((A \wedge B) \vee (\neg A \wedge \neg B)) \ \ |= \ \neg(A \Leftrightarrow B) \] \\ \\
Αρχικά θα μετατρέψουμε την πρόταση \ $\neg(A \Leftrightarrow B) \Leftrightarrow \neg((A \wedge B) \vee (\neg A \wedge \neg B)) $ \  σε μορφή $CNF$.\\ \\
Έστω οι προτάσεις: \\

\hspace{10mm} \bf α \normalfont : \ $\neg(A \Leftrightarrow B)$ \\

\hspace{25mm} και \\

\hspace{10mm} \bf β \normalfont : $\neg((A \wedge B) \vee (\neg A \wedge \neg B)) $ \\ \\ \\
$\bullet \ $ Για την πρόταση \bf α\normalfont , έχουμε: \\ \\

$\circ$ \ Εφαρμόζουμε απαλοιφή αμφίδρομης υποθετικής:

\[
\neg((A \Rightarrow B) \ \wedge \ (B \Rightarrow A))
\] \\

$\circ$ \ Εφαρμόζουμε απαλοιφή συνεπαγωγής:

\[
\neg((\neg A \vee B) \ \wedge \ (\neg B \vee A))
\] \\

$\circ \ De \ Morgan: $

\[
\neg(\neg A \vee B) \ \vee \ (\neg(\neg B \vee A))
\] \\

$\circ \ De \ Morgan: $

\[
( A \wedge  \neg B) \ \vee \ ( B \wedge \neg A)
\] \\

$\circ$ \ Επιμεριστικότητα:

\[
( A \vee B) \ \wedge \ ( A \vee  \neg A) \ \wedge \ ( \neg B \vee B) \ \wedge \ ( \neg B \vee \neg A) \ \equiv \ ( A \vee B) \ \wedge \ ( \neg B \vee \neg A)
\] \\ \\ \\
$\bullet \ $ Για την πρόταση \bf β\normalfont , έχουμε: \\ \\

$\circ$ \ Εφαρμόζουμε $De \ Morgan$:

\[
\neg(A \wedge B) \ \wedge \ (\neg(\neg A \wedge \neg B))
\] \\

$\circ$ \ Εφαρμόζουμε ξανά $De \ Morgan$:

\[
(\neg A \vee \neg B) \ \wedge \ ( A \vee  B)
\] \\ \\ \\
Έχουν παραχθεί η εξής φράσεις από τις προτάσεις α και β: \\ \\
$\bf 1. \normalfont \ \ \  ( A \ \vee \ B)
 $ \\
$ \\
\bf 2. \normalfont \ \ \ ( \neg B \ \vee \ \neg A)
 $ \\
 $ \\
 \bf 3. \normalfont \ \ \  (\neg A \ \vee \  \neg B)
 $ \\
$ \\
\bf 4. \normalfont \ \ \ ( A \ \vee \ B)
 $ \\ \\ \\

 
Εφαρμόζουμε την αρχή της ανάλυσης στις σχέσεις: \\ \\ \\

$\bullet$ \ (1 \&  3) έχουμε το συμπληρωματικό ζεύγος ($\neg A, A$). \\

\hspace{5mm}Το αναλυθεν είναι: $\neg B \vee B$ \ (5) \\ \\ 

$\bullet$ \ (2 \&  4) έχουμε το συμπληρωματικό ζεύγος ($\neg A, A$). \\

\hspace{5mm}Το αναλυθεν είναι: $\neg B \vee B$ \ (6) \\ \\ 
Εφαρμόζουμε την ανάλυση για μία τελευταία φορά στις  (5 \&  6). \\ \\
\hspace{5mm}Το αναλυθέν είναι η κενή φράση, έτσι έχουμε αποδείξει ότι η $( \text{α} \wedge \neg \text{β} ) $ είναι μη ικανοποιήσιμη, άρα ισχύει  η α $|$= β. \\ \\
Εφαρμόζουμε την ίδια διαδικασία για την απόδειξη β $|$= α.

Απο την χρήση των 5 γνώσεων που απαριθμήθηκαν και περιγράφηκαν στην αρχή του προβλήματος, έχουμε καταλήξει στην απόδειξη τ
 

\end{document}

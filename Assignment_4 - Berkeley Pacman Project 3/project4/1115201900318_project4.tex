\documentclass[10pt]{article}
\usepackage[utf8]{inputenc}
\usepackage{url}
\usepackage{hyperref}
\usepackage{amsmath}
\usepackage{amsfonts}
\usepackage{amssymb}
\usepackage{graphicx}
\usepackage{float}
\usepackage{lipsum}
\usepackage{multicol}
\usepackage{xcolor}
\usepackage{natbib}
\usepackage[font=small]{caption}
\usepackage{graphicx,float}
\addtolength{\abovecaptionskip}{-3mm}
\addtolength{\textfloatsep}{-5mm}
\setlength\columnsep{20pt}

\usepackage{wrapfig}

% \documentclass{article}
\usepackage{amsmath}
\usepackage{esint}
\usepackage[LGR, T1]{fontenc}
\usepackage[utf8]{inputenc}
\usepackage[greek]{babel}
\usepackage{alphabeta}
\usepackage{graphicx}
\usepackage{tikz}

\title{ \selectlanguage{english} Project 1 - \selectlanguage{greek} Τεχνητή Νοημοσύνη}

\usepackage[a4paper,left=1.50cm, right=1.50cm, top=1.50cm, bottom=1.50cm]{geometry}


\begin{document}

   
   \begin{center}
        {\Large \textbf{\selectlanguage{english} Project 4 \ - \ \selectlanguage{greek} Τεχνητή Νοημοσύνη}}\\
        \vspace{1em}
        {\large Παναγιώτα Γύφτου ,   A.M.: 1115201900318  } \\
        \vspace{1em}
        {\large Φεβρουάριος 2023}
    \end{center}
    
    
    \begin{center}
        \rule{150mm}{0.2mm}
    \end{center}

    \hspace{69mm}
    \textbf{Θέμα εργασίας} \\
    
\hspace{15mm}1. Ερμηνείες και Ικανοποίηση Προτάσεων \\ 

\hspace{15mm}2. Λογική Πρώτης Τάξης \\ 

\hspace{15mm}3. Συμπερασμός


    \begin{center}
        \rule{150mm}{0.2mm}
    \end{center}

    \vspace{5mm}
    

\section*{Πρόβλημα 2}
\vspace{5mm}
Για την επίλυση του προβλήματος θα χρησιμοποιήσουμε τις διαφάνειες του φροντιστηρίου ($fol-semantics.pdf$). \\ \\ \\
($\bf a$) \normalfont \ Ορισμός ερμηνείας Ι για το λεξιλόγιο των προτάσεων:
\[
φ_1 = JediMaster(Yoda) \ \ \]
\[
φ_2 = (\exists x) JediMaster(x) \ \
\]\[
φ_3 = (\forall x) (JediMaster(x))
\] \\

1. \ Αρχικά ορίζουμε το πεδίο της ερμηνείας Ι που περιέχει τα αντικείμενα του κόσμου της εικόνας: \\ 
\[
\lvert \ \text{Ι} \ \rvert = \{ \ \text{Γιόντα} \ \}
\]
\\

2. \ Για τα σύμβολα σταθερών, η Ι κάνει τις εξής αντιστοιχίσεις σε αντικείμενα:
\[
Yoda^{I} = \text{Γιόντα}
\] \\

3. \ Η Ι αντιστοιχίζει στο μοναδιαίο σύμβολο κατηγορήματος $JediMaster$ την ακόλουθη μοναδιαία σχέση:
\[
\{ \ \left< \ \text{Γιόντα} \ \right > \ \}
\]\\ \\ \\
($\bf b$) \normalfont \ Εύρεση προτάσεων που ικανοποιύνται από την Ι.  \\ \\ \\
$\hookrightarrow  \ \phi_1$: Για τον τύπο $\phi_1$, από τον ορισμό της ικανοποίησης έχουμε:
\\ 

Η ερμηνεία Ι ικανοποιεί τον τύπος της $\phi_1$ για κάποια αντικατάσταση μεταβλητών $s$ αν και μόνο αν το αποτέλεσμα της 

αντικατάστασης της σταθεράς $Yoda$ ανήκει στην σχέση που αντιστοιχίζει η Ι στο κατηγόρημα $JediMaster$.Δηλαδή: \\
\[
 |\text{=}_{I} \ JediMaster(Yoda)[s] \ \  \text{ανν} \ \ \left< \ \overline{s} (Yoda) \ \right > \
 \in JediMaster^{I}
\] \\ 

Όμως \\
\[
\overline{s} (Yoda) = Yoda^{I}= \text{Γιόντα} 
\] \\

και \\
\[
JediMaster^{I} = \{ \ \left< \ \text{Γιόντα} \ \right > \ \}
\] \\ \\

Αρα το παραπάνω ισχύει και συνεπώς ο $\phi_1$ ικανοποιείται από την Ι.  \\ \\ \\ \\
$\hookrightarrow  \ \phi_2$: Για τον τύπο $\phi_2$, έχουμε:
\\ 
\[
|\text{=}_{I} \ ((\exists x) JediMaster(x))[s]
\] \\

Ο τύπος $\phi_{2}$ ικανοποιείται απο την ερμηνεία Ι για κάποια αντικατάσταση μεταβλητών $s$, αν και μόνο αν υπάρχει $d_x \in \lvert \ \text{Ι} \ \rvert$,

δηλαδή:\\
\[
|\text{=}_{I} \ (JediMaster(x))[s(x|d_x)]
\] \\

Δεδομένου ότι για το πεδίο της
I είναι:
\[
\lvert \ \text{Ι} \ \rvert = \{\text{Γιόντα} \}
\] \\

Στην $x$ ανατίθεται η τιμή Γιόντα.\\

Άρα ισχύει ότι: \\
\[
|\text{=}_{I} \ [\ s(x|\text{Γιόντα})\ ]
\] \\

Αφού:\\
\[
s(x|\text{Γιόντα}) = \text{Γιόντα}
\] \\ \\

Επομένως υπάρχει τουλάχιστον μία τιμή του $\lvert \ \text{Ι} \ \rvert $, η οποία ικανοποιεί τον τύπο $\phi_2$.   \\ \\ \\ \\
$\hookrightarrow  \ \phi_3$: Για τον τύπο $\phi_3$, έχουμε:
\\ 
\[
|\text{=}_{I} \ ((\forall x) (JediMaster(x)))[s]
\] \\

Ο τύπος $\phi_{3}$ ικανοποιείται απο την ερμηνεία Ι για κάποια αντικατάσταση μεταβλητών $s$, αν και μόνο αν για κάθε

$d_x \in \lvert \ \text{Ι} \ \rvert$, δηλαδή:\\
\[
|\text{=}_{I} \ (JediMaster(x))[s(x|d_x)]
\] \\

Δεδομένου ότι για το πεδίο της
I είναι:
\[
\lvert \ \text{Ι} \ \rvert = \{\text{Γιόντα} \}
\] \\

Στην $x$ ανατίθεται η τιμή Γιόντα.\\

Άρα ισχύει ότι: \\
\[
|\text{=}_{I} \ [\ s(x|\text{Γιόντα})\ ]
\] \\

Αφού:\\
\[
s(x|\text{Γιόντα}) = \text{Γιόντα}
\] \\ \\

Επομένως κάθε τιμή του $\lvert \ \text{Ι} \ \rvert $, ικανοποιεί τον τύπο $\phi_3$.
\\ \\ \\
\section*{Πρόβλημα 3}
\vspace{5mm}
Για την επίλυση του προβλήματος έχει χρησιμοποιηθεί ο ψευδοκώδικας των διαλέξεων του φροντιστηρίου ($fol-inference.pdf$) διαφάνειες: 4 και 5 \\ \\
Για ευκολία στην καταγραφή της εκτέλεσης του αλγορίθμου $Unify$ για κάθε ενοποίηση αναγράφονται μόνο οι συνθήκες που ισχύουν κάθε φορά. \\ \\
\bf 1. \normalfont Ενοποίηση των τύπων $P(x,x) \  \text{και} \ P( \ G(F(\text{υ})), \  G(u) \ )$ \\ \\

\underline{1ος βρόχος ($i = 0$)}: \ \ \  $x = P, \ \   y = P, \ \ $ γ = \{\} (κενό)  \\ \\

\hspace{5mm}  $\bf if \ x = y \  then \  return \{\}$ \\

\hspace{5mm} Η παραπάνω συνθήκη ισχύει καθώς  $x = P ,\  y = P$ άρα  $x = y \ \Rightarrow 
 P = P$, επομένως επιστρέφεται το κενό. \\ \\

 \underline{2ος βρόχος ($i = 1$)}: \ \ \  $x = x ,\ y = G(F(\text{υ})), \ $ γ = \{\} (κενό) \\ \\
 
 \hspace{5mm} $\bf if \ Variable(x) \ then \ return \ Unify-Var(x, y)$\\

\hspace{5mm} Η συνθήκη αυτή ισχύει αφού η $x \ (=x)$ είναι μεταβλητή και επομένως καλείται η συνάρτηση $Unify-Var$ με

\hspace{5mm} ορίσματα (\ $x=x, \  y=G(F(\text{υ}))$ \ ) \\ 

\hspace{5mm} $\bf Unify-Var(x,y)$ , \ όπου  $x = x , \  y = G(F(\text{υ}))$ \\

\hspace{10mm} $ \bf return \{x/y\}$, \ όπου \  $\{x/ G(F(\text{υ}))\}$ \\ 

\hspace{10mm} Οι δύο τύποι δεν περιέχουν τις ίδιες μεταβλητές, δηλαδή η μεταβλητή $x$ δεν εντοπίζεται στον  $G(F(\text{υ}))$ και 

\hspace{10mm} επομένως επιστρέφεται η ανάθεση $\{x/ G(F(\text{υ}))\}$. \\ \\

\hspace{3mm} Άρα μετά την 2η επανάληψη, έχουμε: \\

\hspace{3mm} $\bullet$ \ γ = $\{x/ G(F(\text{υ}))\}$ \\

\hspace{3mm} $\bullet$ παραγώμενοι τύποι $P( \ G(F(\text{υ})), \ G(F(\text{υ})) \ )$ και $P( \ G(F(\text{υ})), \ G(u) \ )$ \\ \\

 \underline{3ος βρόχος ($i = 2$)}: \ \ \  $x = G(F(\text{υ})) ,\ y = G(u), \ $ γ =$\{x/ G(F(\text{υ}))\}$ \\ \\
 
 Σε αυτή την περίπτωση έχουμε και τα δύο ορίσματα να είναι σύμβολα συναρτήσεων.Άρα θα κληθεί η $UNIFY$ 
 
 αναδρομικά. \\

 \hspace{5mm} \underline{1ος βρόχος αναδρομής ($j = 0$)}: \ \ \  $x = G, \ \   y = G, \ \ $ γ' = \{\} (κενό)  \\ \\

 \hspace{10mm}  $\bf if \ x = y \  then \  return \{\}$ \\

\hspace{10mm} Η παραπάνω συνθήκη ισχύει καθώς  $x = G ,\  y = G$ άρα  $x = y \ \Rightarrow 
 G = G$, επομένως επιστρέφεται το κενό. \\ \\

\hspace{5mm} \underline{2ος βρόχος αναδρομής ($j = 1$)}: \ \ \  $x = F(\text{υ}), \ \   y = u, \ \ $ γ' = \{\} (κενό)  \\ \\

 \hspace{10mm} $\bf if \ Variable(y) \ then \ return \ Unify-Var(y,x)$\\

\hspace{10mm} Η συνθήκη αυτή ισχύει αφού η $y \ (=u)$ είναι μεταβλητή και επομένως καλείται η συνάρτηση $Unify-Var$ με

\hspace{10mm} ορίσματα (\ $x=u, \  y=F(\text{υ})$ \ ) \\ 

\hspace{10mm} $\bf Unify-Var(x,y)$ , \ όπου  $x = u , \  y = F(\text{υ})$ \\

\hspace{10mm} $ \bf return \{x/y\}$, \ όπου \  $\{u/ F(\text{υ})\}$ \\ 

\hspace{10mm} Οι δύο τύποι δεν περιέχουν τις ίδιες μεταβλητές, δηλαδή η μεταβλητή $u$ δεν εντοπίζεται στον  $F(\text{υ})$ και 

\hspace{10mm} επομένως επιστρέφεται η ανάθεση $\{u/ F(\text{υ})\}$. \\ \\

\hspace{7mm} Άρα μετά την 2η επανάληψη, έχουμε: \\

\hspace{7mm} $\bullet$ \ γ' = $\{u/ F(\text{υ})\}$ \\

\hspace{7mm} $\bullet$ παραγώμενοι τύποι $ G(F(\text{υ})) \ $ και $ \ G(F(\text{υ}))$ \\ \\

\hspace{3mm} Άρα μετά την 3η επανάληψη, έχουμε: \\

\hspace{3mm} $\bullet$ \ γ = $\{x/ G(F(\text{υ})) \ , \ u/ F(\text{υ})\}$ \\

\hspace{3mm} $\bullet$ παραγώμενοι τύποι $P( \ G(F(\text{υ})), \ G(F(\text{υ})) \ )$ και $P( \ G(F(\text{υ})), \ G(F(\text{υ})) \ )$ \\ \\

Άρα ο πιο γενικός ενοποιητής $mgu$ είναι:
\[
\{x/ G(F(\text{υ})) \ , \ u/ F(\text{υ})\}
\]
 \\ \\
\bf 2. \normalfont Ενοποίηση των τύπων 
$P(x_1, G(x_2, x_3), x_2, B) \  \text{και} \ P(G(H(A, x_5), x_2), x_1, H(A, x_4), x_4)$ \\ \\
Με παρόμοιο τρόπο η εκτέλεση του αλγορίθμου δίνει τον εξής $mgu$: \\

\[
\{x_1/G(H(A,B),H(A,B)), \ x_2/H(A,B),\ x_3/H(A,B), \ x_5/B, \ x_4/B\}  
\]
 \\ \\
\bf 3. \normalfont Ενοποίηση των τύπων  \\

$P(x_1, x_2, . . . , x_n, F(y_0, y_0), . . . , F(y_{n-1}, y_{n-1}), y_n) \  \ \text{και} \  \ P(F(x_0, x_0), F(x_1, x_1), . . . , F(x_{n-1}, x_{n-1}), y_1, . . . , y_n, x_n)$ \\ \\
Η εκτέλεση του αλγορίθμου δίνει τον εξής $mgu$: \\
\[
\{ \ x_1/F(x_0, x_0), \ x_2/F( \ F(x_0, x_0),F(x_0, x_0)\ ),..., \ x_n/F( \ F(F..._{(n-1)}F(x_0,x_0)),F(F..._{(n-1)}F(x_0,x_0)) \ ), \ y_1/F(x_0, x_0),\]
\[
\ y_2/F( \ F(x_0, x_0),F(x_0, x_0)\ ),..., \ y_n/F( \ F(F..._{(n-1)}F(x_0,x_0)),F(F..._{(n-1)}F(x_0,x_0)) \ ),\ y_0/x_0\}  
\]


\section*{Πρόβλημα 4}
\vspace{5mm}
($\bf a$) \normalfont \\

$\bullet \ $ Τα σύμβολα των σταθερών που αντιπροσωπεύουν αντικείμενα είναι τα εξής: \\ 

\hspace{25mm} \bf Κυριάκος, Αλέξης, Νίκος, Σοσιαλισμός, Καπιταλισμός \normalfont \\

$\bullet \ $ Τα σύμβολα κατηγορημάτων που αντιπροσωπεύουν σχέσεις μεταξύ αντικειμένων είναι τα εξής: \\

\hspace{25mm} \bf Αρέσει($ \bf x,y) \longrightarrow \ $  \normalfont στον $x$ αρέσει το $y$ \\

$\bullet \ $ Τα σύμβολα συναρτήσεων  που αντιπροσωπεύουν σχέσεις στις οποίες υπάρχει μια τιμή ως είσοδος, είναι τα εξής:\\

\hspace{25mm} \bf μέλοςΚόμματοςΚΟΡΩΝΑ($ \bf  x) \longrightarrow \ $ \normalfont  o $x$ είναι μέλος του κόμματος ΚΟΡΩΝΑ

\hspace{25mm} \bf Φιλελεύθερος($\bf x )  \longrightarrow \ $ \normalfont ο $x$ είναι φιλελεύθερος 

\hspace{25mm} \bf Δεξιός($\bf x )  \longrightarrow \ $ \normalfont  ο 
$x$ είναι δεξιός \\ \\ \\
Αφού έχουμε εξηγήσει με ακρίβεια τι παριστάνουν τα παραπάνω σύμβολα, θα μετατρέψουμε τις δοσμένες ελληνικές προτάσεις σε προτάσεις της λογικής πρώτης τάξης.Έχουμε: \\ 

\[ 
\bf i.\ \ \ \ \  \normalfont \text{μέλοςΚόμματοςΚΟΡΩΝΑ(Κυριάκος)} \ \ , \ \ \text{μέλοςΚόμματοςΚΟΡΩΝΑ(Αλέξης)} \ \ , \ \ \text{μέλοςΚόμματοςΚΟΡΩΝΑ(Νίκος)} 
\]

\[ 
\bf ii.\ \ \  \normalfont (\forall x)\left   ( \ \left (\text{μέλοςΚόμματοςΚΟΡΩΝΑ}(x) \ \ \bigwedge \ \ \neg \text{Δεξιός}(x)\right)  \ \  \Longrightarrow  \ \ \text{Φιλελεύθερος}(x) \ \right) \ \ \ \ \ \ \ \ \ \ \ \ \ \ \ \ \ \ \ \ \ \ \ \ \ \ \ \ \ \ \ \ \ \ \ \ \ \ \ \ \ \ \ \ \ \ \ \ \
\]

\[ 
\bf iii.\ \  \normalfont (\forall x)\left   ( \  \text{Δεξιός}(x) \ \ \Longrightarrow     \ \ \neg \text{Αρέσει($x$,Σοσιαλισμός)} \ \right)  \ \ \ \ \ \ \ \ \ \ \ \ \ \ \ \ \ \ \ \ \ \ \ \ \ \ \ \ \ \ \ \ \ \ \ \ \ \ \ \ \ \ \ \ \ \ \ \ \ \ \ \ \ \ \ \ \ \ \ \ \ \ \ \ \ \ \ \ \ \ \ \ \ \ \ \ \ \ \ \ \ \ \ \ \ \ \ \ \ \ \ \ \ \ \ \ \ \ \ \
\]

\[ 
\bf iv.\ \ \  \normalfont (\forall x)\left   ( \ \neg \text{Αρέσει($x$,Καπιταλισμός)} \ \ \Longrightarrow     \ \ \neg \text{Φιλελεύθερος($x$)} \ \right)  \ \ \ \ \ \ \ \ \ \ \ \ \ \ \ \ \ \ \ \ \ \ \ \ \ \ \ \ \ \ \ \ \ \ \ \ \ \ \ \ \ \ \ \ \ \ \ \ \ \ \ \ \ \ \ \ \ \ \ \ \ \ \ \ \ \ \ \ \ \ \ \ \ \ \ \ \ \ \ \ \ \ \ \ \ \ \ \ \ \ \ \ \ \ \ \ \ \ \ \
\]

\[ 
\bf v.\ \ \ \ \normalfont (\forall x)\left   ( \ \left   (  \text{Αρέσει(Αλέξης,$x$)} \ \ \Rightarrow  \ \ \neg \text{Αρέσει(Κυριάκος,$x$)}  \right) \ \ \bigwedge \ \ \left   (  \neg \text{Αρέσει(Αλέξης,$x$)} \ \ \Rightarrow  \ \ \text{Αρέσει(Κυριάκος,$x$)}  \right) \  \right) \ \ \ \ \ \ \ \ \ \ \ \ \ \ \ \ \ \ \ \ \ \ \ \ \ \ \ \ \ \ \ \ \ \ \ \ \ \ \ \ \ \ \ \ \ \ \ \ \ \ \ \ \ \ \ \ \ \ \ \ \ \ \ \ \ \ \ \ \ \ \ \ \ \ \ \ \ \ \ \ \ \ \ \ \ \ \ \ \ \ \ \ \ \ \ \ \ \ \ \
\]

\[ 
\bf vi.\ \ \ \normalfont \ \text{Αρέσει(Αλέξης,Σοσιαλισμός)} \ \ \bigwedge  \ \ \text{Αρέσει(Αλέξης,Καπιτελισμός)} \ \ \ \ \ \ \ \ \ \ \ \ \ \ \ \ \ \ \ \ \ \ \ \ \ \ \ \ \ \ \ \ \ \ \ \ \ \ \ \ \ \ \ \ \ \ \ \ \ \ \ \ \ \ \ \ \ \ \ \ \ \ \ \ \ \ \ \ \ \ \ \ \ \ \ \ \ \ \ \ \ \ \ \ \ \ \ \ \ \ \ \ \ \ \ \ \ \ \ \ \
\] \\
Επομένως η Βάση Γνώσης ΚΒ περιέχει τις παραπάνω προτάσεις. \\ \\ \\
Η πρόταση $vii$ μετατρέπεται στην εξής λογική πρόταση:

\[
\phi \ = \  (\exists x)\left   ( \text{μέλοςΚόμματοςΚΟΡΩΝΑ}(x) \ \ \bigwedge \ \ \text{Φιλελεύθερος}(x) \ \ \bigwedge \ \ \neg \text{Δεξιός}(x) \ \right) \ \ \ \ \ \ \ \ \ \ \ \ \ \ \ \ \ \ \ \ \ \ \ \ \ \ \ \ \ \ \ \ \ \ \ \ \ \ \ \ \ \ \ \ \ \ \ \ \
\]
\\ \\ \\ ($\bf b$) \normalfont \ Απόδειξη ΚΒ $|$= φ, μέσω της μεθόδου της ανάλυσης.  \\

Αρχικά θα πρέπει να μετατρέψουμε τις προτάσεις σε συζευκτική κανονική μορφή.

\[ 
\bf i.\ \ \ \ \  \normalfont \text{μέλοςΚόμματοςΚΟΡΩΝΑ(Κυριάκος)} \ \ \ \ \ \ \ \ \ \ \ \ \ \ \  \LARGE \bf (1) \ \ \ \ \ \ \ \ \ \ \ \ \ \ \ \ \ \ \ \ \ \ \ \ \ \ \ \ \ \ \ \ \ \ \ \ \ \ \ \ \ \ \ \ \ \ \ \ \ \]
\[ \text{μέλοςΚόμματοςΚΟΡΩΝΑ(Αλέξης)} \  \ \ \ \ \ \ \ \ \ \ \ \ \ \ \  \LARGE \bf \  (2) \ \ \ \ \ \ \ \ \ \ \ \ \ \ \ \ \ \ \ \ \ \ \ \ \ \ \ \ \ \ \ \ \ \ \ \ \ \ \ \ \ \ \ \ \ \]
\[
\ \text{μέλοςΚόμματοςΚΟΡΩΝΑ(Νίκος)} \  \ \ \ \ \ \ \ \ \ \ \ \ \ \ \  \LARGE \bf \ \  (3) \ \ \ \ \ \ \ \ \ \ \ \ \ \ \ \ \ \ \ \ \ \ \ \ \ \ \ \ \ \ \ \ \ \ \ \ \ \ \ \ \ \ \ \ \ \
\]
\\

\[ 
\bf ii.\ \ \  \normalfont (\forall x)\left   ( \ \left (\text{μέλοςΚόμματοςΚΟΡΩΝΑ}(x) \ \ \bigwedge \ \ \neg \text{Δεξιός}(x)\right)  \ \  \Longrightarrow  \ \ \text{Φιλελεύθερος}(x) \ \right) \ \ \ \ \ \ \ \ \ \ \ \ \ \ \ \ \ \ \ \ \ \ \ \ \ \ \ \ \ \ \ \ \ \ \ \ \ \ \ \ \ \ \ \ \ \ \ \ \
\]
 \\ 
 
$\bullet \ $ Απαλοιφή των συνεπαγωγών:

\[ 
(\forall x)\left   ( \ \neg \left (\text{μέλοςΚόμματοςΚΟΡΩΝΑ}(x) \ \ \bigwedge \ \ \neg \text{Δεξιός}(x)\right)  \ \  \bigvee  \ \ \text{Φιλελεύθερος}(x) \ \right) \ \ \ \ \ \ \ \ \ \ \ \ \ \ \ \ \ \ \ \ \ \ \ \ \ \ \ \ \ \ 
\] \\

$\bullet \ $ Μετακίνηση του $\neg$ προς τα μέσα:

\[ 
(\forall x)\left   ( \ \neg \text{μέλοςΚόμματοςΚΟΡΩΝΑ}(x) \ \ \bigvee \ \ \text{Δεξιός}(x)  \ \  \bigvee  \ \ \text{Φιλελεύθερος}(x) \ \right) \ \ \ \ \ \ \ \ \ \ \ \ \ \ \ \ \ \ \ \ \ \ \ \ \ \ \ \ \ \ \ \ \ \ \ 
\] \\

$\bullet \ $ Κατάργηση καθολικών ποσοδεικτών:

\[ 
 \ \neg \text{μέλοςΚόμματοςΚΟΡΩΝΑ}(x) \ \ \bigvee \ \ \text{Δεξιός}(x)  \ \  \bigvee  \ \ \text{Φιλελεύθερος}(x) \ \ \ \ \ \ \ \ \ \ \ \ \ \ \ \ \ \ \ \ \ \ \LARGE \bf (4)  \ \ \ \ \ \ \ \ \ \ \ \ \ \ 
\] 

\[ 
\bf iii.\ \  \normalfont (\forall x)\left   ( \  \text{Δεξιός}(x) \ \ \Longrightarrow     \ \ \neg \text{Αρέσει($x$,Σοσιαλισμός)} \ \right)  \ \ \ \ \ \ \ \ \ \ \ \ \ \ \ \ \ \ \ \ \ \ \ \ \ \ \ \ \ \ \ \ \ \ \ \ \ \ \ \ \ \ \ \ \ \ \ \ \ \ \ \ \ \ \ \ \ \ \ \ \ \ \ \ \ \ \ \ \ \ \ \ \ \ \ \ \ \ \ \ \ \ \ \ \ \ \ \ \ \ \ \ \ \ \ \ \ \ \ \
\] \\ 
 
$\bullet \ $ Απαλοιφή των συνεπαγωγών:

\[ 
 (\forall x)\left   ( \ \neg \text{Δεξιός}(x) \ \ \bigvee     \ \ \neg \text{Αρέσει($x$,Σοσιαλισμός)} \ \right)  \ \ \ \ \ \ \ \ \ \ \ \ \ \ \ \ \ \ \ \ \ \ \ \ \ \ \ \ \ \ \ \ \ \ \ \ \ \ \ \ \ \ \ \ \ \ \ \ \ \ \ \ \ \ \ \ \ \ \ \ \ \ \ \ \ \ \ \ \ \  
\] \\

$\bullet \ $ Κατάργηση καθολικών ποσοδεικτών:

\[ 
 \neg \text{Δεξιός}(x) \ \ \bigvee     \ \ \neg \text{Αρέσει($x$,Σοσιαλισμός)} \ \ \ \ \ \ \ \ \ \ \ \ \ \ \ \ \ \ \ \ \ \ \LARGE \bf (5)  \ \ \ \ \ \ \ \ \ \ \ \ \ \ \ \ \ \ \ \ \ \ \ \ \ \ \ \ 
\] \\

\[ 
\bf iv.\ \ \  \normalfont (\forall x)\left   ( \ \neg \text{Αρέσει($x$,Καπιταλισμός)} \ \ \Longrightarrow     \ \ \neg \text{Φιλελεύθερος($x$)} \ \right)  \ \ \ \ \ \ \ \ \ \ \ \ \ \ \ \ \ \ \ \ \ \ \ \ \ \ \ \ \ \ \ \ \ \ \ \ \ \ \ \ \ \ \ \ \ \ \ \ \ \ \ \ \ \ \ \ \ \ \ \ \ \ \ \ \ \ \ \ \ \ \ \ \ \ \ \ \ \ \ \ \ \ \ \ \ \ \ \ \ \ \ \ \ \ \ \ \ \ \ \
\]\ \\
 
$\bullet \ $ Απαλοιφή των συνεπαγωγών:

\[ 
 (\forall x)\left   ( \ \text{Αρέσει($x$,Καπιταλισμός)} \ \ \bigvee    \ \ \neg \text{Φιλελεύθερος($x$)} \ \right)  \ \ \ \ \ \ \ \ \ \ \ \ \ \ \ \ \ \ \ \ \ \ \ \ \ \ \ \ \ \ \ \ \ \ \ \ \ \ \ \ \ \ \ \ \ \ \ \ \ \ \ \ \ \ \ \ \ \ \ \ \ \ 
\]\\

$\bullet \ $ Κατάργηση καθολικών ποσοδεικτών:

\[ 
 \text{Αρέσει($x$,Καπιταλισμός)} \ \ \bigvee    \ \ \neg \text{Φιλελεύθερος($x$)}  \ \ \ \ \ \ \ \ \ \ \ \ \ \ \ \ \ \ \ \ \ \ \LARGE \bf (6)  \ \ \ \ \ \ \ \ \ \ \ \ \ \ \ \ \ \ \ \ \ \ \ \ \ \ \ \ \
\]\\

\[ 
\bf v.\ \ \ \ \normalfont (\forall x)\left   ( \ \left   (  \text{Αρέσει(Αλέξης,$x$)} \ \ \Rightarrow  \ \ \neg \text{Αρέσει(Κυριάκος,$x$)}  \right) \ \ \bigwedge \ \ \left   (  \neg \text{Αρέσει(Αλέξης,$x$)} \ \ \Rightarrow  \ \ \text{Αρέσει(Κυριάκος,$x$)}  \right) \  \right) \ \ \ \ \ \ \ \ \ \ \ \ \ \ \ \ \ \ \ \ \ \ \ \ \ \ \ \ \ \ \ \ \ \ \ \ \ \ \ \ \ \ \ \ \ \ \ \ \ \ \ \ \ \ \ \ \ \ \ \ \ \ \ \ \ \ \ \ \ \ \ \ \ \ \ \ \ \ \ \ \ \ \ \ \ \ \ \ \ \ \ \ \ \ \ \ \ \ \ \
\]\\
 
$\bullet \ $ Απαλοιφή των συνεπαγωγών:

\[ 
 (\forall x)\left   ( \ \left   ( \neg \text{Αρέσει(Αλέξης,$x$)} \ \ \bigvee  \ \ \neg \text{Αρέσει(Κυριάκος,$x$)}  \right) \ \ \bigwedge \ \ \left   (  \text{Αρέσει(Αλέξης,$x$)} \ \ \bigvee   \ \ \text{Αρέσει(Κυριάκος,$x$)}  \right) \  \right) \ \ \ \ \ \ \ \ \ \ \ \ \ \ \ \ \ \ \ \ \ \ \ \ \ \ \ \ \ \ \ \ \ \ \ \ \ \ \ \ \ \ \ \ \ \ \ \ \ \ \ \ \ \ \ \ \ \ \ \ \ \ \ \ \ \ \ \ \ \ \ \ \ \ \ \ \ \ \ \ \ \ \ \ \ \ \ \ \ \ \ \ \ \ \ \ \ \ \ \
\]\\

$\bullet \ $ Κατάργηση καθολικών ποσοδεικτών:

\[ 
  \ \left   ( \neg \text{Αρέσει(Αλέξης,$x$)} \ \ \bigvee  \ \ \neg \text{Αρέσει(Κυριάκος,$x$)}  \right) \ \ \bigwedge \ \ \left   (  \text{Αρέσει(Αλέξης,$x$)} \ \ \bigvee   \ \ \text{Αρέσει(Κυριάκος,$x$)}  \right) \ \ \ \ \ \ \ \ \ \ \ \ \ \ \ \ \ \ \ \ \ \ \ \ \ \ \LARGE \bf (7) \ \ \ \ \ \ \ \ \ \ \ \ \ \ \ \ \ \ \ \ \ \ \ \ \ \ \ \ \ \ \ \ \ \ \ \ \ \ \ \ \ \ \ \ \ \ \ \ \ \ \ \ \ \ \ \ \ \ \ \ \ \ \ \ \ \ \ \ \ \ \ \ \ \ \
\]\\

\[ 
\bf vi.\ \ \ \normalfont \ \text{Αρέσει(Αλέξης,Σοσιαλισμός)} \ \ \bigwedge  \ \ \text{Αρέσει(Αλέξης,Καπιτελισμός)} \ \ \ \ \ \ \ \ \ \ \ \ \ \ \ \ \ \ \ \ \ \ \ \ \ \ \LARGE \bf (8)\ \ \ \ \ \ \ \ \ \ \ \ \ \ \ \ \ \ \ \ \ \ \ \ \ \ \ \ \ \ \ \ \ \ \ \ \ \ \ \ \ \ \ \ \ \ \ \ \ \ \ \ \ \ \ \ \ \ \ \ \ \ \ \
\] \\ \\ \\
Για να αποδείξουμε μέσω της ανάλυσης ότι ΚΒ $|$= φ, αρκεί να αποδείξουμε ότι (ΚΒ$\ \wedge  \ \neg$φ) δεν είναι ικανοποιήσιμη δηλαδή παράγοντας την κενή πρόταση. \\ \\ \\
$\circ$ Η $\neg$φ είναι η εξής: 

\[
\neg \phi \ = \ \neg  (\exists x)\left   ( \text{μέλοςΚόμματοςΚΟΡΩΝΑ}(x) \ \ \bigwedge \ \ \text{Φιλελεύθερος}(x) \ \ \bigwedge \ \ \neg \text{Δεξιός}(x) \ \right) \ \ \ \ \ \ \ \ \ \ \ \ \ \ \ \ \ \ \ \ \ \ \
\]\\

$\bullet \ $ Μετακίνηση του $\neg$ προς τα μέσα:

\[
\neg \phi \ = \ ( \forall x) \neg \left   ( \text{μέλοςΚόμματοςΚΟΡΩΝΑ}(x) \ \ \bigwedge \ \ \text{Φιλελεύθερος}(x) \ \ \bigwedge \ \ \neg \text{Δεξιός}(x) \ \right) \ \ \ \ \ \ \ \ \ \ \ \ \ \ \ \ \ \ \ \ \ \ \
\]\\

$\bullet \ $ Μετακίνηση του $\neg$ προς τα μέσα:

\[
\neg \phi \ = \ ( \forall x) \left   (\neg  \text{μέλοςΚόμματοςΚΟΡΩΝΑ}(x) \ \ \bigvee \ \ \neg  \text{Φιλελεύθερος}(x) \ \ \bigvee \ \ \text{Δεξιός}(x) \ \right) \ \ \ \ \ \ \ \ \ \ \ \ \ \ \ \ \ \ \ \ \ \ \
\]\\

$\bullet \ $ Κατάργηση καθολικών ποσοδεικτών:

\[
\neg \phi \ = \neg  \text{μέλοςΚόμματοςΚΟΡΩΝΑ}(x) \ \ \bigvee \ \ \neg  \text{Φιλελεύθερος}(x) \ \ \bigvee \ \ \text{Δεξιός}(x) \ \ \ \ \ \ \ \ \ \ \ \ \ \ \ \ \ \ \ \ \ \ \ \
\]\\ \\ \\
Έχουν παραχθεί οι εξής φράσεις: \\ \\
$\bf 1. \normalfont \ \ \  \normalfont \text{μέλοςΚόμματοςΚΟΡΩΝΑ(Κυριάκος)} $  \\ \\
$\bf 2. \normalfont \ \ \  \normalfont \text{μέλοςΚόμματοςΚΟΡΩΝΑ(Αλέξης)} $ \\ \\
$\bf 3. \normalfont \ \ \  \normalfont \text{μέλοςΚόμματοςΚΟΡΩΝΑ(Νίκος)} $ \\ \\
$\bf 4. \normalfont \ \ \ \neg \text{μέλοςΚόμματοςΚΟΡΩΝΑ}(x) \ \ \bigvee \ \ \text{Δεξιός}(x)  \ \  \bigvee  \ \ \text{Φιλελεύθερος}(x) $\\ \\
$\bf 5. \normalfont \ \ \ \ \neg \text{Δεξιός}(x) \ \ \bigvee     \ \ \neg \text{Αρέσει($x$,Σοσιαλισμός)}$ \\ \\ 
$\bf 6. \normalfont \ \ \   \text{Αρέσει($x$,Καπιταλισμός)} \ \ \bigvee    \ \ \neg \text{Φιλελεύθερος($x$)} $ \\ \\
$\bf 7. \normalfont \ \ \ \  \neg \text{Αρέσει(Αλέξης,$x$)} \ \ \bigvee  \ \ \neg \text{Αρέσει(Κυριάκος,$x$)} $\\ \\
$\bf 8. \normalfont \ \ \ \   \text{Αρέσει(Αλέξης,$x$)} \ \ \bigvee  \ \ \text{Αρέσει(Κυριάκος,$x$)} \ $ \\ \\
$\bf 9. \normalfont \ \ \ \ \text{Αρέσει(Αλέξης,Σοσιαλισμός)} $ \\ \\
$\bf 10. \normalfont \ \ \ \text{Αρέσει(Αλέξης,Καπιτελισμός)}$ \\ \\
$\bf 11. \normalfont \ \ \  \neg  \text{μέλοςΚόμματοςΚΟΡΩΝΑ}(x) \ \ \bigvee \ \ \neg  \text{Φιλελεύθερος}(x) \ \ \bigvee \ \ \text{Δεξιός}(x)$\\ \\ \\

 
Εφαρμόζουμε την αρχή της ανάλυσης στις σχέσεις: \\ \\ \\

$\bullet$ \ (5 \&  9) απαλείφοντας τα συμπληρωματικά λεκτικά  \\
\[ \neg \text{Δεξιός}(x) \ \ \bigvee     \ \ \neg \text{Αρέσει($x$,Σοσιαλισμός) \ και \ } \text{Αρέσει(Αλέξης,Σοσιαλισμός)} \] 

\hspace{5mm} με τον ενοποιητή θ=\{$x/$Αλέξης\} έχουμε την αναλυθείσα πρόταση: \\

\hspace{60mm} $\neg$ Δεξιός(Αλέξης)\ \ \  (12) \\ \\  \\

$\bullet$ \ (4 \&  12) απαλείφοντας τα συμπληρωματικά λεκτικά  \\
\[ \neg \text{μέλοςΚόμματοςΚΟΡΩΝΑ}(x) \ \ \bigvee \ \ \text{Δεξιός}(x)  \ \  \bigvee  \ \ \text{Φιλελεύθερος}(x) \  \ \ \ \text{ και \ \ \ \ }  \neg \text{Δεξιός(Αλέξης)} \] 

\hspace{5mm} με τον ενοποιητή θ=\{$x/$Αλέξης\} έχουμε την αναλυθείσα πρόταση: \\

\hspace{40mm} $\neg \text{μέλοςΚόμματοςΚΟΡΩΝΑ(Αλέξης)} \ \  \bigvee  \ \ \text{Φιλελεύθερος(Αλέξης)}$\ \ \  (13) \\ \\  \\

$\bullet$ \ (2 \&  13) απαλείφοντας τα συμπληρωματικά λεκτικά  \\
\[  \text{μέλοςΚόμματοςΚΟΡΩΝΑ(Αλέξης)} \ \ \ \ \text{ και \ \ \ \ }  \neg \text{μέλοςΚόμματοςΚΟΡΩΝΑ(Αλέξης)} \ \  \bigvee  \ \ \text{Φιλελεύθερος(Αλέξης)} \] 

\hspace{5mm} έχουμε την αναλυθείσα πρόταση: \\

\hspace{40mm} $\text{Φιλελεύθερος(Αλέξης)}$\ \ \  (14) \\ \\  \\

$\bullet$ \ (11 \&  14) απαλείφοντας τα συμπληρωματικά λεκτικά  \\
\[ \neg \text{μέλοςΚόμματοςΚΟΡΩΝΑ}(x) \ \ \bigvee \ \ \neg \text{Φιλελεύθερος}(x)  \ \  \bigvee  \ \ \text{Δεξιός}(x) \  \ \ \ \text{ και \ \ \ \ }  \text{Φιλελεύθερος(Αλέξης)} \] 

\hspace{5mm} με τον ενοποιητή θ=\{$x/$Αλέξης\} έχουμε την αναλυθείσα πρόταση: \\

\hspace{40mm} $\neg \text{μέλοςΚόμματοςΚΟΡΩΝΑ(Αλέξης)} \ \  \bigvee  \ \ \text{Δεξιός(Αλέξης)}$\ \ \  (15)
\\ \\\ \\

$\bullet$ \ (2 \&  15) απαλείφοντας τα συμπληρωματικά λεκτικά  \\
\[ \text{μέλοςΚόμματοςΚΟΡΩΝΑ(Αλέξης)} \ \ \ \ \text{ και \ \ \ \ }  \neg \text{μέλοςΚόμματοςΚΟΡΩΝΑ(Αλέξης)} \ \  \bigvee  \ \ \text{Δεξιός(Αλέξης)} \] 

\hspace{5mm} με τον ενοποιητή θ=\{\} έχουμε την αναλυθείσα πρόταση: \\

\hspace{40mm} $\text{Δεξιός(Αλέξης)}$\ \ \  (16)
\\ \\\ \\

$\bullet$ \ (5 \&  16) απαλείφοντας τα συμπληρωματικά λεκτικά  \\
\[ \neg \text{Δεξιός}(x) \ \ \bigvee \ \ \neg \ \ \text{Αρέσει($x$,Σοσιαλισμός)} \ \ \text{ και \ \ \ \ }  \text{Δεξιός(Αλέξης)}  \] 

\hspace{5mm} με τον ενοποιητή θ=\{$x/$Αλέξης\} έχουμε την αναλυθείσα πρόταση: \\

\hspace{40mm} $\neg \text{Αρέσει(Αλέξης,Σοσιαλισμός)}$\ \ \  (17)
\\ \\\ \\

$\bullet$ \ (9 \&  17) απαλείφοντας τα συμπληρωματικά λεκτικά  \\
\[ \text{Αρέσει(Αλέξης,Σοσιαλισμός)} \ \ \text{ και \ \ \ \ }  \neg \text{Αρέσει(Αλέξης,Σοσιαλισμός)}  \] 

\hspace{5mm} με τον ενοποιητή θ=\{\} έχουμε ότι το αναλυθέν είναι η κενή φράση.\\

Έτσι έχουμε αποδείξει ότι η $( KB \wedge \neg \phi) $ είναι μη ικανοποιήσιμη, άρα ισχύει  η $KB$ $|$= $\phi$. \\ \\  \\
($\bf c$) \normalfont \\ \\
Σε αυτή την περίπτωση δεν θα προσθέσουμε την $\neg \ \phi$ στην ΚΒ αλλά την $( \ Ans(x) \ \vee \ \neg \phi \ )$.\\
Για την εύρεση του μέλους του ΚΟΡΩΝΑ, αρκεί η μέθοδος της ανάλυσης να μας δώσει την φράση φ' = $Ans(x)$ \\ \\
\underline{Παρατήρηση}: Για ευκολία επειδή η εκτέλεση της ανάλυσης είναι παρόμοια με αυτή του ερωτήματος β, οι διαφορές φαίνονται 

\hspace{15mm} με έντονα γράμματα.

\[ \underline{\underline{
\left ( Ans(x) \ \bigvee \ \neg \phi \right )}} = \ Ans(x) \ \  \bigvee \ \ \neg  \text{μέλοςΚόμματοςΚΟΡΩΝΑ}(x) \ \ \bigvee \ \ \neg  \text{Φιλελεύθερος}(x) \ \ \bigvee \ \ \text{Δεξιός}(x) 
\] \ \\ \\
Έχουν παραχθεί οι εξής φράσεις: \\ \\
$\bf 1. \normalfont \ \ \  \normalfont \text{μέλοςΚόμματοςΚΟΡΩΝΑ(Κυριάκος)} $  \\ \\
$\bf 2. \normalfont \ \ \  \normalfont \text{μέλοςΚόμματοςΚΟΡΩΝΑ(Αλέξης)} $ \\ \\
$\bf 3. \normalfont \ \ \  \normalfont \text{μέλοςΚόμματοςΚΟΡΩΝΑ(Νίκος)} $ \\ \\
$\bf 4. \normalfont \ \ \ \neg \text{μέλοςΚόμματοςΚΟΡΩΝΑ}(x) \ \ \bigvee \ \ \text{Δεξιός}(x)  \ \  \bigvee  \ \ \text{Φιλελεύθερος}(x) $\\ \\
$\bf 5. \normalfont \ \ \ \ \neg \text{Δεξιός}(x) \ \ \bigvee     \ \ \neg \text{Αρέσει($x$,Σοσιαλισμός)}$ \\ \\ 
$\bf 6. \normalfont \ \ \   \text{Αρέσει($x$,Καπιταλισμός)} \ \ \bigvee    \ \ \neg \text{Φιλελεύθερος($x$)} $ \\ \\
$\bf 7. \normalfont \ \ \ \  \neg \text{Αρέσει(Αλέξης,$x$)} \ \ \bigvee  \ \ \neg \text{Αρέσει(Κυριάκος,$x$)} $\\ \\
$\bf 8. \normalfont \ \ \ \   \text{Αρέσει(Αλέξης,$x$)} \ \ \bigvee  \ \ \text{Αρέσει(Κυριάκος,$x$)} \ $ \\ \\
$\bf 9. \normalfont \ \ \ \ \text{Αρέσει(Αλέξης,Σοσιαλισμός)} $ \\ \\
$\bf 10. \normalfont \ \ \ \text{Αρέσει(Αλέξης,Καπιτελισμός)}$ \\ \\
$\bf 11. \ \ \  Ans(x) \ \  \bigvee \ \ \neg  \text{\bf μέλοςΚόμματοςΚΟΡΩΝΑ}(x) \ \ \bigvee \ \ \neg  \text{ \bf Φιλελεύθερος}(x) \ \ \bigvee \ \ \text{ \bf Δεξιός}(x) $\\ \\ \\

 
Εφαρμόζουμε την αρχή της ανάλυσης στις σχέσεις: \\ \\ \\

$\bullet$ \ (5 \&  9) απαλείφοντας τα συμπληρωματικά λεκτικά  \\
\[ \neg \text{Δεξιός}(x) \ \ \bigvee     \ \ \neg \text{Αρέσει($x$,Σοσιαλισμός) \ και \ } \text{Αρέσει(Αλέξης,Σοσιαλισμός)} \] 

\hspace{5mm} με τον ενοποιητή θ=\{$x/$Αλέξης\} έχουμε την αναλυθείσα πρόταση: \\

\hspace{60mm} $\neg$ Δεξιός(Αλέξης)\ \ \  (12) \\ \\  \\

$\bullet$ \ (4 \&  12) απαλείφοντας τα συμπληρωματικά λεκτικά  \\
\[ \neg \text{μέλοςΚόμματοςΚΟΡΩΝΑ}(x) \ \ \bigvee \ \ \text{Δεξιός}(x)  \ \  \bigvee  \ \ \text{Φιλελεύθερος}(x) \  \ \ \ \text{ και \ \ \ \ }  \neg \text{Δεξιός(Αλέξης)} \] 

\hspace{5mm} με τον ενοποιητή θ=\{$x/$Αλέξης\} έχουμε την αναλυθείσα πρόταση: \\

\hspace{40mm} $\neg \text{μέλοςΚόμματοςΚΟΡΩΝΑ(Αλέξης)} \ \  \bigvee  \ \ \text{Φιλελεύθερος(Αλέξης)}$\ \ \  (13) \\ \\  \\

$\bullet$ \ (2 \&  13) απαλείφοντας τα συμπληρωματικά λεκτικά  \\
\[  \text{μέλοςΚόμματοςΚΟΡΩΝΑ(Αλέξης)} \ \ \ \ \text{ και \ \ \ \ }  \neg \text{μέλοςΚόμματοςΚΟΡΩΝΑ(Αλέξης)} \ \  \bigvee  \ \ \text{Φιλελεύθερος(Αλέξης)} \] 

\hspace{5mm} έχουμε την αναλυθείσα πρόταση: \\

\hspace{40mm} $\text{Φιλελεύθερος(Αλέξης)}$\ \ \  (14) \\ \\  \\

$\bullet$ \ (11 \&  14) απαλείφοντας τα συμπληρωματικά λεκτικά  \\
\[ \bf Ans(x) \ \ \bigvee \ \ \neg \text{ \bf μέλοςΚόμματοςΚΟΡΩΝΑ}(x) \ \ \bigvee \ \ \neg \text{\bf Φιλελεύθερος}(x)  \ \  \bigvee  \ \ \text{\bf Δεξιός}(x) \  \ \ \ \text{ και \ \ \ \ }  \text{Φιλελεύθερος(Αλέξης)} \] 

\hspace{5mm} με τον ενοποιητή θ=\{$x/$Αλέξης\} έχουμε την αναλυθείσα πρόταση: \\

\hspace{10mm} $ \bf Ans(\text{Αλέξης}) \ \ \bigvee \ \ \neg \text{\bf μέλοςΚόμματοςΚΟΡΩΝΑ(Αλέξης)} \ \  \bigvee  \ \ \text{ \bf Δεξιός(Αλέξης)}$\ \ \  (15)
\\ \\\ \\

$\bullet$ \ (2 \&  15) απαλείφοντας τα συμπληρωματικά λεκτικά  \\
\[ \text{μέλοςΚόμματοςΚΟΡΩΝΑ(Αλέξης)} \ \text{ και \  }  \bf Ans(\text{Αλέξης}) \ \bigvee \ \neg \text{\bf μέλοςΚόμματοςΚΟΡΩΝΑ(Αλέξης)} \ \bigvee \ \text{ \bf Δεξιός(Αλέξης)}  \] 

\hspace{5mm} με τον ενοποιητή θ=\{\} έχουμε την αναλυθείσα πρόταση: \\

\hspace{40mm} $\bf Ans(\text{\bf Αλέξης}) \ \bigvee \ \text{\bf Δεξιός(Αλέξης)}$\ \ \  (16)
\\ \\\ \\

$\bullet$ \ (5 \&  16) απαλείφοντας τα συμπληρωματικά λεκτικά  \\
\[ \neg \text{Δεξιός}(x) \ \ \bigvee \ \ \neg \ \ \text{Αρέσει($x$,Σοσιαλισμός)} \ \ \text{ και \ \ \ \ }  \bf Ans(\text{\bf Αλέξης}) \ \bigvee \ \text{\bf Δεξιός(Αλέξης)}  \] 

\hspace{5mm} με τον ενοποιητή θ=\{$x/$Αλέξης\} έχουμε την αναλυθείσα πρόταση: \\

\hspace{40mm} $\bf Ans(\text{\bf Αλέξης}) \ \bigvee \ \neg \text{Αρέσει(Αλέξης,Σοσιαλισμός)}$\ \ \  (17)
\\ \\\ \\

$\bullet$ \ (9 \&  17) απαλείφοντας τα συμπληρωματικά λεκτικά  \\
\[ \text{Αρέσει(Αλέξης,Σοσιαλισμός)} \ \ \text{ και \ \ \ \ }  \bf Ans(\text{\bf Αλέξης}) \ \bigvee \ \neg \text{\bf Αρέσει(Αλέξης,Σοσιαλισμός)}  \] 

\hspace{5mm} με τον ενοποιητή θ=\{\} έχουμε ότι το αναλυθέν είναι η φράση: $\bf Ans(\text{\bf Αλέξης})$.\\ \\ \\
Έτσι έχουμε ότι η απάντηση στο ζητούμενο, δηλαδή ποιό είναι το μέλος του ΚΟΡΩΝΑ που έχει την ιδιότητα που παριστάνει η φ, είναι ο \it  \underline{\underline{Αλέξης}}. \normalfont 
\\ \\ \\ 
\section*{Πρόβλημα 5}
\vspace{5mm}
($\bf a$) \normalfont Μετατροπή προτάσεων σε συζευκτική κανονική μορφή $(CNF)$ \\ \\
\[ \bf A. \ \ \ \ \ \ \ \ \underline{\underline{
(\forall x)(\forall s)(\forall t) \left( In(x,s) \ \ \bigwedge \ \ In(x,t) \ \ \Longleftrightarrow  \ \ In\left( x, Intersection(s,t)\right )  \right )}}
 \ \ \ \ \ \ \ \ \ \ \ \ \] \\
 
$\circ \ $ Απαλοιφή αμφίδρομης υποθετικής:
\[
(\forall x)(\forall s)(\forall t) \left(a \bigwedge b \right ) \]

όπου, \\
\[
a \ = \ \left(  In(x,s) \ \ \bigwedge \ \ In(x,t)  \right ) \ \ \Longrightarrow  \ \ In( x, Intersection(s,t)) \ \ \ \ \ \ \ \ \ \ \ \ \ \ \ \ \ \ \ \  \ \ \ \ \ \ \ \ \ \ \ \ \ \ \ \ \ \ \ \ \ \ \ \ \ \ \ \ 
\]

\[
b \ = \ In( x, Intersection(s,t)) \ \ \Longrightarrow  \ \ \left(  In(x,s) \ \ \bigwedge \ \ In(x,t)  \right ) \ \ \ \ \ \ \ \ \ \ \ \ \ \ \ \ \ \ \ \  \ \ \ \ \ \ \ \ \ \ \ \ \ \ \ \ \ \ \ \ \ \ \ \ \ \ \ \ 
\]\\ \\
 
$\circ \ $ Απαλοιφή συνεπαγωγών:
\[
(\forall x)(\forall s)(\forall t) \left(a \bigwedge b \right ) \]

όπου, \\
\[
a \ = \ \neg \left (  In(x,s) \ \ \bigwedge \ \ In(x,t)  \right ) \ \ \bigvee  \ \ In( x, Intersection(s,t)) \ \ \ \ \ \ \ \ \ \ \ \ \ \ \ \ \ \ \ \  \ \ \ \ \ \ \ \ \ \ \ \ \ \ \ \ \ \ \ \ \ \ \ \ \ \ \ \ 
\]

\[
b \ = \ \neg In( x, Intersection(s,t)) \ \ \bigvee  \ \ \left(  In(x,s) \ \ \bigwedge \ \ In(x,t)  \right ) \ \ \ \ \ \ \ \ \ \ \ \ \ \ \ \ \ \ \ \  \ \ \ \ \ \ \ \ \ \ \ \ \ \ \ \ \ \ \ \ \ \ \ \ \ \ \ \ 
\]\\ \\
 
$\circ \ $ Μετακίνηση του $\neg$ προς τα μέσα:
\[
(\forall x)(\forall s)(\forall t) \left(a \bigwedge b \right ) \]

όπου, \\
\[
a \ = \ \neg  In(x,s) \ \ \bigvee \ \  \neg In(x,t) \ \ \bigvee  \ \ In( x, Intersection(s,t)) \ \ \ \ \ \ \ \ \ \ \ \ \ \ \ \ \ \ \ \  \ \ \ \ \ \ \ \ \ \ \ \ \ \ \ \ \ \ \ \ \ \ \ \ \ \ \ \ 
\]

\[
b \ = \ \neg In( x, Intersection(s,t)) \ \ \bigvee  \ \ \left(  In(x,s) \ \ \bigwedge \ \ In(x,t)  \right ) \ \ \ \ \ \ \ \ \ \ \ \ \ \ \ \ \ \ \ \  \ \ \ \ \ \ \ \ \ \ \ \ \ \ \ \ \ \ \ \ \ \ \ \ \ \ \ \ 
\]\\ \\
 
$\circ \ $ Κατάρηση καθολικών ποσοδεικτών:
\[
a \bigwedge b  \]

όπου, \\
\[
a \ = \ \neg  In(x,s) \ \ \bigvee \ \  \neg In(x,t) \ \ \bigvee  \ \ In( x, Intersection(s,t)) \ \ \ \ \ \ \ \ \ \ \ \ \ \ \ \ \ \ \ \  \ \ \ \ \ \ \ \ \ \ \ \ \ \ \ \ \ \ \ \ \ \ \ \ \ \ \ \ 
\]

\[
b \ = \ \neg In( x, Intersection(s,t)) \ \ \bigvee  \ \ \left(  In(x,s) \ \ \bigwedge \ \ In(x,t)  \right ) \ \ \ \ \ \ \ \ \ \ \ \ \ \ \ \ \ \ \ \  \ \ \ \ \ \ \ \ \ \ \ \ \ \ \ \ \ \ \ \ \ \ \ \ \ \ \ \ 
\]\\ \\

$\circ \ $ Κατανομή του $\bigvee$ ως προς το $\bigwedge$:
\[
a \bigwedge b  \]

όπου, \\
\[
a \ = \ \neg  In(x,s) \ \ \bigvee \ \  \neg In(x,t) \ \ \bigvee  \ \ In( x, Intersection(s,t)) \ \ \ \ \ \ \ \ \ \ \ \ \ \ \ \ \ \ \ \  \ \ \ \ \ \ \ \ \ \ \ \ \ \ \ \ \ \ \ \ \ \ \ \ \ \ \ \ 
\]

\[
b \ = \ \left( \neg In( x, Intersection(s,t)) \ \ \bigvee \ \  In(x,s) \right) \ \ \bigwedge \ \ \left( \neg In( x, Intersection(s,t)) \ \ \bigvee \ \  In(x,t) \right)  \ \ \ \ \ \
\]\\ \\

επομένως έχουν παραχθεί οι εξής φράσεις: \\ \\

\[ \bullet \ \ \ \ 
 \neg  In(x,s) \ \ \bigvee \ \  \neg In(x,t) \ \ \bigvee  \ \ In( x, Intersection(s,t)) \ \ \ \ \ \ \ \ \ \ \ \ \ \ \ \ \ \ \ \  \ \ \ \ \ \ \ \ \ \ \ \ \ \ \ \ \ \ \ \ \ \ \ \ \ \ \ \ 
\]

\[ \bullet \ \ \ \ 
 \neg In( x, Intersection(s,t)) \ \ \bigvee \ \  In(x,s)\ \ \ \ \ \ \ \ \ \ \ \ \ \ \ \ \ \ \ \ \ \ \ \ \ \  \ \ \ \ \ \ \ \ \ \ \ \ \ \ \ \ \ \ \ \ \ \ \ \ \ \ \ \ \ \ \ \ \ \ \ \ \ \ \ \ \ \ 
\]

\[ \bullet \ \ \ \ 
\neg In( x, Intersection(s,t)) \ \ \bigvee \ \  In(x,t) \ \ \ \ \ \ \ \ \ \ \ \ \ \ \ \ \ \ \ \ \ \ \ \ \ \  \ \ \ \ \ \ \ \ \ \ \ \ \ \ \ \ \ \ \ \ \ \ \ \ \ \ \ \ \ \ \ \ \ \ \ \ \ \ \ \ \ \ 
\]\\ \\

\[ \bf B. \ \ \ \ \ \ \ \ \underline{\underline{
(\forall s)(\forall t)\left( \ (\forall x) \left( In(x,s) \ \ \Rightarrow \ \ In(x,t) \right ) \ \ \Longrightarrow  \ \ SubsetOf(s,t) \right)}}
 \ \ \ \ \ \ \ \ \ \ \ \ \] \\
 
$\circ \ $ Απαλοιφή συνεπαγωγών:

\[ 
(\forall s)(\forall t)\left( \ \neg \left( (\forall x) \left( In(x,s) \ \ \Rightarrow \ \ In(x,t) \right ) \ \right) \ \ \bigvee  \ \ SubsetOf(s,t) \right)
 \ \ \ \ \ \ \ \ \ \ \ \ \] \\ \\
 
$\circ \ $ Μετακίνηση του $\neg$ προς τα μέσα:

\[ 
(\forall s)(\forall t)\left( \  (\exists x) \ \neg \left( In(x,s) \ \ \Rightarrow \ \ In(x,t) \right ) \ \ \bigvee  \ \ SubsetOf(s,t) \right)
 \ \ \ \ \ \ \ \ \ \ \ \ \] \\ \\
 
$\circ \ $ Απαλοιφή συνεπαγωγών:

\[ 
(\forall s)(\forall t)\left( \  (\exists x) \ \neg \left( \ \neg In(x,s) \ \ \bigvee \ \ In(x,t) \right ) \ \ \bigvee  \ \ SubsetOf(s,t) \right)
 \ \ \ \ \ \ \ \ \ \ \ \ \] \\ \\
  
$\circ \ $ Μετακίνηση του $\neg$ προς τα μέσα:

\[ 
(\forall s)(\forall t)\left( \  (\exists x) \ \left(  In(x,s) \ \ \bigwedge \ \ \neg In(x,t) \right ) \ \ \bigvee  \ \ SubsetOf(s,t) \right)
 \ \ \ \ \ \ \ \ \ \ \ \ \] \\ \\
  
$\circ \ $ Μετατροπή κατά $Skolem$:

\[ 
(\forall s)(\forall t)\left( \  \left(  In(F(s,t),s) \ \ \bigwedge \ \ \neg In(F(s,t),t) \right ) \ \ \bigvee  \ \ SubsetOf(s,t) \right)
 \ \ \ \ \ \ \ \ \ \ \ \ \] \\ \\

 $\circ \ $ Κατάργηση καθολικών ποσοδεικτών:

\[ 
 \  \left(  In(F(s,t),s) \ \ \bigwedge \ \ \neg In(F(s,t),t) \right ) \ \ \bigvee  \ \ SubsetOf(s,t) 
 \ \ \ \ \ \ \ \ \ \ \ \ \] \\ \\

 $\circ \ $ Κατανομή του $\bigvee$ ως προς το $\bigwedge$:

 \[ 
 \  \left(  In(F(s,t),s) \ \ \bigvee \ \  SubsetOf(s,t) \right ) \ \ \bigwedge  \  \  \left( \neg In(F(s,t),t) \ \ \bigvee \ \  SubsetOf(s,t) \right )
 \ \ \ \ \ \ \ \ \ \ \ \ \] \\ \\

επομένως έχουν παραχθεί οι εξής φράσεις: \\ \\

 \[  \bullet \ \ \ \
 \   In(F(s,t),s) \ \ \bigvee \ \  SubsetOf(s,t)
 \ \ \ \ \ \ \ \ \ \ \ \ \] 
 \[ \bullet \ \ \ \
  \neg In(F(s,t),t) \ \ \bigvee \ \  SubsetOf(s,t)  \ \ \ \ \ \ \ \ \ \ \ \
 \]
\\ \\ \\
\[ \bf C. \ \ \ \ \ \ \ \ \underline{\underline{
(\forall s)(\forall t)SubsetOf(Intersection(s,t),s)}}
 \ \ \ \ \ \ \ \ \ \ \ \ \] \\

 Όμως θέλουμε την άρνηση της πρότασης ($\neg C$). Επομένως:
 
\[ \neg \ 
(\forall s)(\forall t)SubsetOf(Intersection(s,t),s)
 \ \ \ \ \ \ \ \ \ \ \ \ \] \\

  
$\circ \ $ Μετακίνηση του $\neg$ προς τα μέσα:

\[  \ 
(\exists s)(\exists t)\ \neg 
 SubsetOf(Intersection(s,t),s)
 \ \ \ \ \ \ \ \ \ \ \ \ \] \\

 $\circ \ $ Μετατροπή κατά $Skolem$:

 \[  \  \neg 
 SubsetOf(Intersection(S,T),S)
 \ \ \ \ \ \ \ \ \ \ \ \ \] \\ \\

επομένως έχει παραχθεί η εξής φράση:  
 \[  \bullet \ \ \ \ \neg 
 SubsetOf(Intersection(S,T),S)
 \ \ \ \ \ \ \ \ \ \ \ \ \] \\ \\ \\
 ($\bf b$) \normalfont Απόδειξη ότι η πρόταση $C$ είναι λογική συνέπεια των προτάσεων Α και Β. \\ \\ \\
 Έστω $L$ το σύνολο των προτάσεων Α και Β. Θέλουμε να αποδείξουμε μέσω της ανάλυσης ότι $L$ $|$= $C$, αρκεί να αποδείξουμε ότι ($L\ \wedge  \ \neg C$) δεν είναι ικανοποιήσιμη δηλαδή παράγοντας την κενή πρόταση. \\ \\ 
 Έχουν παραχθεί οι εξής φράσεις από $A, \ B \ \text{και} \ \neg C$:\\ \\

 $ 1. \ \ \ \ 
 \neg  In(x,s) \ \ \bigvee \ \  \neg In(x,t) \ \ \bigvee  \ \ In( x, Intersection(s,t))$ \\

$ 2. \ \ \ \ 
 \neg In( x, Intersection(s,t)) \ \ \bigvee \ \  In(x,s)$ \\

$ 3. \ \ \ \ 
\neg In( x, Intersection(s,t)) \ \ \bigvee \ \  In(x,t) $ \\

$ 4. \ \ \ \
 \   In(F(s,t),s) \ \ \bigvee \ \  SubsetOf(s,t)$ \\

$ 5. \ \ \ \
  \neg In(F(s,t),t) \ \ \bigvee \ \  SubsetOf(s,t)$\\

$ 6. \ \ \ \ \neg 
 SubsetOf(Intersection(S,T),S)$ \\ \\ \\

 Εφαρμόζουμε την αρχή της ανάλυσης στις σχέσεις: \\ \\ 

$\bullet$ \ (4 \&  6) απαλείφοντας τα συμπληρωματικά λεκτικά  \\
\[
\neg SubsetOf(Intersection(S,T),S) \ \ \text{και} \ \ \left( In(F(s,t),s) \ \ \bigvee \ \ SubsetOf(s,t) \right) 
\] 

\hspace{5mm} με τον ενοποιητή θ=\{$s/Intersection(S,T) \ , \ t/S$\} έχουμε την αναλυθείσα πρόταση: \\

\hspace{60mm} $ In(F(Intersection(S,T),S),Intersection(S,T))$\ \ \  (7) \\ \\  \\

$\bullet$ \ (5 \&  6) απαλείφοντας τα συμπληρωματικά λεκτικά  \\
\[
\neg SubsetOf(Intersection(S,T),S) \ \ \text{και} \ \ \left( \neg In(F(s,t),t) \ \ \bigvee \ \ SubsetOf(s,t) \right) 
\] 

\hspace{5mm} με τον ενοποιητή θ=\{$s/Intersection(S,T) \ , \ t/S$\} έχουμε την αναλυθείσα πρόταση: \\

\hspace{60mm} $\neg In(F(Intersection(S,T),S),S)$\ \ \  (8) \\ \\  \\

$\bullet$ \ (2 \&  7) απαλείφοντας τα συμπληρωματικά λεκτικά  \\
\[
\left( \neg In( x, Intersection(s,t)) \ \ \bigvee \ \  In(x,s) \right) \ \ \text{και} \ \ In(F(Intersection(S,T),S),Intersection(S,T)) 
\] 

\hspace{5mm} με τον ενοποιητή θ=\{$x/F(Intersection(S,T),S) \ , \ s/S \ , \ t/T$\} έχουμε την αναλυθείσα πρόταση: \\

\hspace{60mm} $\ In(F(Intersection(S,T),S),S)$\ \ \  (9) \\ \\  \\

$\bullet$ \ (8 \&  9) απαλείφοντας τα συμπληρωματικά λεκτικά  \\
\[
\neg In(F(Intersection(S,T),S),S) \ \ \text{και} \ \  In(F(Intersection(S,T),S),S) 
\] 

\hspace{1mm} Το αναλυθέν είναι η κενή φράση, έτσι έχουμε αποδείξει ότι η $( L \wedge \neg C ) $ είναι μη ικανοποιήσιμη, άρα ισχύει  η $L$ $|$= $C$. \\ \\  \\
Συνεπώς αποδείξαμε ότι η πρόταση $C$ είναι λογική συνέπεια των προτάσεων Α και Β. \\ \\ \\
\section*{Πρόβλημα 6}
\vspace{5mm}
Αρχικά θα εκφράσουμε τις ελληνικές προτάσεις σε φράσεις $Horn$.Για την μετατροπή θα χρησιμοποιήσουμε τα εξής σύμβολα συναρτήσεων και κατηγορημάτων: \\ \\

\bf \underline{Σύμβολα συναρτήσεων}: \normalfont \\

\hspace{5mm}Όμορφο($x$) \ \ \ \ \ \ \ \ $\mapsto$ \ \ \ \ \ \ \ \ \ Ο/Η $x$ είναι όμορφος/η 


\hspace{5mm}Πλούσιος($x$)\ \ \ \ \ \ \ $\mapsto$\ \ \ \ \ \ \ \ \ \ \ Ο/Η $x$ είναι πλούσιος/ια 

\hspace{5mm}Μυώδης($x$)\ \ \ \ \ \ \ \ \ $\mapsto$ \ \ \ \ \ \ \ \ \ \ Ο/Η $x$ είναι μυώδης

\hspace{5mm}Ευγενικός($x$)\ \ \ \ \ \ $\mapsto$   \ \ \ \ \ \ \ \ \ \
Ο/Η $x$ είναι ευγενικός/η

\hspace{5mm}Ευτυχής($x$)\ \ \ \ \ \ \ \ $\mapsto$   \ \ \ \ \ \ \ \ \ \ Ο/Η $x$ είναι ευτυχισμένος/η

\hspace{5mm}Γυναίκα($x$) \ \ \ \ \ \ 
 \ \ $\mapsto$   \ \ \ \ \ \ \ \ \ \ Η $x$ είναι γυναίκα
 
\hspace{5mm}Άνδρας($x$) \ \ \ \ \ \ \ \ $\mapsto$   \ \ \ \ \ \ \ \ \ \ Ο $x$ είναι άνδρας \\ \\

\bf \underline{Σύμβολα κατηγορημάτων}:
\normalfont \\

\hspace{5mm}Αρέσει($x,y$) \ \ \ \ \ \ \ \ $\mapsto$   \ \ \ \ \ \ \ \ \ \ Στον/ην $x$ αρέσει ο/η $y$\\ \\ \\
Μετατροπή προτάσεων σε φράσεις $Horn$: \\

$\bullet \ $ Η Ελένη είναι όμορφη. 
\[
\text{Όμορφο( \ Ελένη \ )}
\]
\\

$\bullet \ $ Ο Γιάννης είναι όμορφος και πλούσιος.
\[
\text{Όμορφο( \ Γιάννης \ )} \]
\[
\text{Πλούσιος( \ Γιάννης \ )}
\] \\

$\bullet \ $ Ο Πέτρος είναι μυώδης και πλούσιος.
\[
\text{Μυώδης( \ Πέτρος \ )} \]
\[
\text{Πλούσιος( \ Πέτρος \ )}
\] \\

$\bullet \ $ Ο Τίμος είναι μυώδης και ευγενικός.
\[
\text{Μυώδης( \ Τίμος\ )} \]
\[
\text{Ευγενικός( \ Τίμος \ )}
\] \\

$\bullet \ $ Σε όλους τους άνδρες αρέσουν οι όμορφες γυναίκες. 
\[
\left( \text{Άνδρας($x$)} \ \ \bigwedge \ \ \text{Γυναίκα($y$)} \ \ \bigwedge \ \ \text{Όμορφο($y$)} \right) \ \ \Rightarrow \ \ \text{Αρέσει($x,y$)}
\]\\

$\bullet \ $ Όλοι οι πλούσιοι είναι ευτυχισμένοι. 
\[
\text{Πλούσιος($x$)} \ \ \Rightarrow \ \ \text{Ευτυχής($x$)}
\]\\

$\bullet \ $ Όλοι οι άνδρες που τους αρέσει μια γυναίκα, στην οποία αρέσουν, είναι ευτυχισμένοι.
\[
\left( \text{Άνδρας($x$)} \ \ \bigwedge \ \ \text{Γυναίκα($y$)} \ \ \bigwedge \ \  \text{Αρέσει($x,y$)} \ \ \bigwedge \ \  \text{Αρέσει($y,x$)} \right) \ \ \Rightarrow \ \ \text{Ευτυχής($x$)}
\]\\

$\bullet \ $ Όλες οι γυναίκες που τους αρέσει ένας άνδρας, στον οποίο αρέσουν, είναι ευτυχισμένες. 
\[
\left( \text{Γυναίκα($x$)} \ \ \bigwedge \ \ \text{Άνδρας($y$)} \ \ \bigwedge \ \  \text{Αρέσει($x,y$)} \ \ \bigwedge \ \  \text{Αρέσει($y,x$)} \right) \ \ \Rightarrow \ \ \text{Ευτυχής($x$)}
\]\\

$\bullet \ $ Στην Κατερίνα αρέσουν όλοι οι άνδρες, στους οποίους η ίδια αρέσει.
\[
\left( \text{Άνδρας($x$)} \ \ \bigwedge \ \  \text{Αρέσει($x,$Κατερίνα)} \right) \ \ \Rightarrow \ \ \text{Αρέσει(Κατερίνα,$x$)} 
\]\\

$\bullet \ $ Στην Ελένη αρέσουν όλοι οι άνδρες που είναι ευγενικοί και πλούσιοι ή μυώδεις και όμορφοι. 
\[
\left( \text{Άνδρας($x$)} \ \ \bigwedge \ \  \text{Ευγενικός($x$)} \ \ \bigwedge \ \  \text{Πλούσιος($x$)} \right) \ \ \Rightarrow \ \ \text{Αρέσει(Ελένη,$x$)} 
\]
\[
\left( \text{Άνδρας($x$)} \ \ \bigwedge \ \  \text{Μυώδης($x$)} \ \ \bigwedge \ \  \text{Όμορφο($x$)} \right) \ \ \Rightarrow \ \ \text{Αρέσει(Ελένη,$x$)} 
\]\\ \\ \\
Τα γνωστά γεγονότα στην ΚΒ είναι τα εξής: \\

$ 1. \ \ \text{Όμορφο( \ Ελένη \ )} $

$ 2. \ \ \text{Όμορφο( \ Γιάννης \ )} $

$ 3. \ \ \text{Πλούσιος( \ Γιάννης \ )} $

$ 4. \ \ \text{Μυώδης( \ Πέτρος \ )} $

$ 5. \ \ \text{Πλούσιος( \ Πέτρος \ )}$

$ 6. \ \ \text{Μυώδης( \ Τίμος\ )} $

$ 7. \ \ \text{Ευγενικός( \ Τίμος \ )}$ \\ \\ \\
Επιπλέον, έχουμε και τα εξής παραγώμενα γνωστά: \\

$ 1. \ \ \text{Γυναίκα( \ Ελένη \ )} $

$ 2. \ \ \text{Γυναίκα( \ Κατερίνα \ )} $

$ 3. \ \ \text{Άνδρας( \ Γιάννης \ )} $

$ 4. \ \ \text{Άνδρας( \ Πέτρος \ )} $

$ 5. \ \ \text{Άνδρας( \ Τίμος \ )} $ \\ \\ \\ \\ Έχει επιλεγεί για την εύρεση των συμπερασμάτων από τις προτάσεις $Horn$ ο αλγόριθμος $ forward \  chaining $. \\ \\
$\circ \ $  Ποιός αρέσει σε ποιόν; \\

Αυτό που θέλουμε να απαντήσουμε είναι το εξής: \ \ \  \underline{Αρέσει($x,y$)} \\ \\
Φράση της ΚΒ που οι υποθέσεις ταιριάζουν με τα υπάρχοντα γνωστά γεγονότα και το συμπέρασμα ταυτίζεται με το ζητούμενο είναι: \\

$\bullet \ $ Σε όλους τους άνδρες αρέσουν οι όμορφες γυναίκες. 
\[
\left( \text{Άνδρας($x$)} \ \ \bigwedge \ \ \text{Γυναίκα($y$)} \ \ \bigwedge \ \ \text{Όμορφο($y$)} \right) \ \ \Rightarrow \ \ \text{Αρέσει($x,y$)}
\]\\ \\ \\

\underline{1η περίπτωση}:  \{$x=$Γιάννης, $y=$Ελένη\}

\begin{figure}[H]
    \includegraphics[width=\linewidth, height=.25\textheight,keepaspectratio=true]{4_611.png}\\
    \caption{$\left( \text{Άνδρας($x$)} \ \ \bigwedge \ \ \text{Γυναίκα($y$)} \ \ \bigwedge \ \ \text{Όμορφο($y$)} \right) \ \ \Rightarrow \ \ \text{Αρέσει($x,y$)}$  \ \bf με \{$x=$Γιάννης, $y=$Ελένη\}}
\end{figure}  

\underline{2η περίπτωση}:  \{$x=$ Πέτρος, $y=$Ελένη\}

\begin{figure}[H]
    \includegraphics[width=\linewidth, height=.25\textheight,keepaspectratio=true]{4_612.png}\\
    \caption{$\left( \text{Άνδρας($x$)} \ \ \bigwedge \ \ \text{Γυναίκα($y$)} \ \ \bigwedge \ \ \text{Όμορφο($y$)} \right) \ \ \Rightarrow \ \ \text{Αρέσει($x,y$)}$  \ \bf με \{$x=$Πέτρος, $y=$Ελένη\}}
\end{figure} 

\underline{3η περίπτωση}:  \{$x=$ Τίμος, $y=$Ελένη\}

\begin{figure}[H]
    \includegraphics[width=\linewidth, height=.25\textheight,keepaspectratio=true]{4_613.png}\\
    \caption{Ο γράφος $AND-OR$ της περίπτωσης: \ $\bf \left( \text{Άνδρας($x$)} \ \ \bigwedge \ \ \text{Γυναίκα($y$)} \ \ \bigwedge \ \ \text{Όμορφο($y$)} \right) \ \ \Rightarrow \ \ \text{Αρέσει($x,y$)}$  \\ \bf με \{$x=$Τίμος, $y=$Ελένη\}}
\end{figure} 

Επομένως η απάντηση στο ερώτημα \it Ποιός αρέσει σε ποιόν; \normalfont είναι η εξής: \\

Αρέσει(Γίαννης,Ελένη)

Αρέσει(Πέτρος,Ελένη)

Αρέσει(Τίμος,Ελένη) \\ \\ \\
$\circ \ $  Ποιός είναι ευτυχισμένος; \\

Αυτό που θέλουμε να απαντήσουμε είναι το εξής: \ \ \  \underline{Ευτυχής($x$)} \\ \\
Φράση της ΚΒ που οι υποθέσεις ταιριάζουν με τα υπάρχοντα γνωστά γεγονότα και το συμπέρασμα ταυτίζεται με το ζητούμενο είναι: \\

$\bullet \ $ Όλοι οι πλούσιοι είναι ευτυχισμένοι. 
\[
\text{Πλούσιος($x$)} \ \ \Rightarrow \ \ \text{Ευτυχής($x$)}
\]\\

\underline{1η περίπτωση}:  \{$x=$ Γιάννης\}

\begin{figure}[H]
    \includegraphics[width=\linewidth, height=.25\textheight,keepaspectratio=true]{4_621.png}\\
    \caption{Ο γράφος $AND-OR$ της περίπτωσης: \ $\bf \text{Πλούσιος($x$)} \ \ \Rightarrow \ \ \text{Ευτυχής($x$)}$  \ \bf με \{$x=$Γιάννης\} }
\end{figure} 

\underline{1η περίπτωση}:  \{$x=$ Πέτρος\}

\begin{figure}[H]
    \includegraphics[width=\linewidth, height=.25\textheight,keepaspectratio=true]{4_622.png}\\
    \caption{Ο γράφος $AND-OR$ της περίπτωσης: \ $\bf \text{Πλούσιος($x$)} \ \ \Rightarrow \ \ \text{Ευτυχής($x$)}$  \ \bf με \{$x=$Πέτρος\} }
\end{figure} 

Επομένως η απάντηση στο ερώτημα \it Ποιός είναι ευτυχισμένος; \normalfont είναι η εξής: \\

Ευτυχής(Γιάννης)

Ευτυχής(Πέτρος)
\\ \\ \\
\section*{Πρόβλημα 7}
\vspace{5mm}
\underline{\underline{Για το πρόβλημα 4 έχουμε}}: \\ \\ \\
$\bf 4.b$ \\

\underline{1ο βήμα}: \ Δημιουργία βάσης γνώσης στο $Prover9$. Προσθέτουμε τις προτάσεις της ΚΒ στα $Assumptions$ του $Prover$. \\ 

\underline{2ο βήμα}: \ Για να ελέγξουμε ότι η ΚΒ καλύπτει την πρόταση φ προσθέτουμε στα $Goals$ την πρόταση φ. \\ \\ 
Πιο συγκεκριμένα:
\begin{figure}[H]
    \includegraphics[width=\linewidth, height=.35\textheight,keepaspectratio=true]{7_11.png}\\
    \caption{αρχείο: $4\_b\_input$ }
\end{figure} 
Το αποτέλεσμα της εκτέλεσης είναι το εξής:
\begin{figure}[H]
    \includegraphics[width=\linewidth, height=.35\textheight,keepaspectratio=true]{7_12.png}\\
    \caption{αρχείο: $4\_b.proof$ }
\end{figure} 
Η απόδειξη μας επιβεβαιώθηκε αφού αποδεικνύεται.
 \\ \\ \\
$\bf 4.c$ \\

\underline{1ο βήμα}: \ Δημιουργία βάσης γνώσης στο $Prover9$. Προσθέτουμε τις προτάσεις της ΚΒ στα $Assumptions$ του $Prover$. \\ 

\underline{2ο βήμα}: \ Για την εύρεση του μέλους του ΚΟΡΩΝΑ προσθέτουμε στα $Assumptions$ την πρόταση $\neg$ φ και το $Prover$ 

\hspace{15mm}θα μας δώσει μία απάντηση μέσω της $answer$. \\ \\ 
Πιο συγκεκριμένα:
\begin{figure}[H]
    \includegraphics[width=\linewidth, height=.35\textheight,keepaspectratio=true]{7_21.png}\\
    \caption{αρχείο: $4\_c\_input$ }
\end{figure} 
Το αποτέλεσμα της εκτέλεσης είναι το εξής:
\begin{figure}[H]
    \includegraphics[width=\linewidth, height=.35\textheight,keepaspectratio=true]{7_22.png}\\
    \caption{αρχείο: $4\_c.proof$ }
\end{figure} Η απόδειξη μας επιβεβαιώθηκε αφού η επιστρεφόμενη απάντηση είναι ο Αλέξης, όπως είχαμε βρεί προηγουμένως με την μέθοδο της ανάλυσης. \\ \\
\underline{\underline{Για το πρόβλημα 5 έχουμε}}: \\ \\ \\
$\bf 5.b$ \\

\underline{1ο βήμα}: \ Δημιουργία βάσης γνώσης στο $Prover9$. Προσθέτουμε τις προτάσεις Α και Β στα $Assumptions$ του $Prover$. \\ 

\underline{2ο βήμα}: \ Για να ελέγξουμε ότι η ΚΒ καλύπτει την πρόταση $C$ προσθέτουμε στα $Goals$ την πρόταση $C$. \\ \\ 
Πιο συγκεκριμένα:
\begin{figure}[H]
    \includegraphics[width=\linewidth, height=.35\textheight,keepaspectratio=true]{7_31.png}\\
    \caption{αρχείο: $5\_b\_input$ }
\end{figure} 
Το αποτέλεσμα της εκτέλεσης είναι το εξής:
\begin{figure}[H]
    \includegraphics[width=\linewidth, height=.35\textheight,keepaspectratio=true]{7_32.png}\\
    \caption{αρχείο: $5\_b.proof$ }
\end{figure} 
Η απόδειξη μας επιβεβαιώθηκε αφού αποδεικνύεται.
 \\ \\ \\
\section*{Πρόβλημα 8}
\vspace{5mm}
($\bf a$) \normalfont \\ \\
Τα δωθέντα κατηγορήματα $subClassOf$ και $belongsTo$ ορίζονται ως εξής:
\\
\[
subClassOf(x,y)\ \ \ \ \ \ \ \ \ \mapsto \ \ \ \ \ \ \ \ \ \text{για να υποδηλώσουμε ότι το $x$ είναι υποκλάση της κατηγορίας $y$.} 
\]

\[
belongsTo(x,y)\ \ \ \ \ \ \ \ \ \mapsto \ \ \ \ \ \ \ \ \ \text{για να δηλώσουμε ότι το $x$  είναι μέλος μιας κατηγορίας $y$.}  \ \ \ \ \ \
\]
\\ \\ Τα σύμβολα σταθερών που θα ανατεθούν ως τιμές των $x,y$ των παραπάνω κατηγορημάτων είναι τα εξής: \\ \\
$Country \\
Decentralized Administration \\
Region \\
RegionalUnit \\
Municipality \\
MunicipalityUnit \\
MunicipalCommunity \\
LocalCommunity \\
AdministrativeUnit$ \\ \\ \\
Ισχύουν οι εξής ιδιότητες:\\

Οι $subClassOf$ και $belongsTo$ είναι αντανακλαστικές και μεταβατικές, δηλαδή: \\

\hspace{5mm}  $subClassOf$ :
\[
(\forall x)(\neg subClassOf(x,x))
\]

\hspace{80mm} και
\[
(\forall x,y,z)\left( subClassOf(x,y) \ \ \bigwedge \ \ subClassOf(y,z) \ \ \bigwedge \ \ subClassOf(x,z) \right )
\]
\\

\hspace{5mm}  $belongsTo$ :
\[
(\forall x)(\neg belongsTo(x,x))
\]

\hspace{80mm} και
\[
(\forall x,y,z)\left( belongsTo(x,y) \ \ \bigwedge \ \ belongsTo(y,z) \ \ \bigwedge \ \ belongsTo(x,z) \right )
\]
\\ \\ \\
Η οντολογική μηχανή που αναπαριστά το σχήμα του προβλήματος αποτελείται από τους παρακάτω τύπους:  \\ \\

\underline{Μια κατηγορία είναι υποκατηγορία κάποιας άλλης}: \\ 

$subClassOf(Country,AdministrativeUnit)$

$subClassOf(Decentralized Administration,AdministrativeUnit)$

$subClassOf(Region,AdministrativeUnit)$

$subClassOf(RegionalUnit,AdministrativeUnit)$

$subClassOf(Municipality,AdministrativeUnit)$

$subClassOf(MunicipalityUnit,AdministrativeUnit)$

$subClassOf(MunicipalCommunity,AdministrativeUnit)$

$subClassOf(LocalCommunity,AdministrativeUnit)$ \\ \\

\underline{Μια διοικιτική υποδιέρεση ανήκει σε κάποια άλλη διοικιτική υποδιέρεση}: \\

$belongsTo(Decentralized Administration, Country)$

$belongsTo(Region,Decentralized Administration)$

$belongsTo(RegionalUnit,Region)$

$belongsTo(Municipality ,RegionalUnit)$

$belongsTo(MunicipalityUnit, Municipality)$

$belongsTo(MunicipalCommunity, MunicipalityUnit)$

$belongsTo(LocalCommunity , MunicipalCommunity)$
\\ \\ \\ 
($\bf b$) \normalfont \\ \\
Για την αναπαράσταση της επιπλέον κλάσης στην οντολογική μηχανή, που προστέθηκε στην οντολογία θα την εκφράσουμε στους εξής τύπους: \\

$MemberOf(Country, Class)$

$MemberOf(Decentralized Administration, Class)$

$MemberOf(Region, Class)$

$MemberOf(RegionalUnit, Class)$

$MemberOf(Municipality, Class)$

$MemberOf(MunicipalityUnit, Class)$

$MemberOf(MunicipalCommunity, Class)$

$MemberOf(LocalCommunity, Class)$

$MemberOf(AdministrativeUnit, Class)$ \\ \\
Και εδώ ισχύουν οι ιδιότητες της αντανακλαστικότητας και μεταβατικότητας, δηλαδή: \\

\hspace{5mm}  $MemberOf$ :
\[
(\forall x)(\neg MemberOf(x,x))
\]

\hspace{80mm} και
\[
(\forall x,y,z)\left( MemberOf(x,y) \ \ \bigwedge \ \ MemberOf(y,z) \ \ \bigwedge \ \ MemberOf(x,z) \right )
\]
\\ \\ \\
($\bf c$) \normalfont \\ \\
Για την αναπαράσταση του επιπλέον αντικειμένου $Municipality of Athens$, το οποίο είναι στοιχείο της κλάσης $Municipality$ στην οντολογική μηχανή, που προστέθηκε στην οντολογία θα χρησιμοποιήσουμε το δυαδικό σύμβολο κατηγορήματος που δημιουργήσαμε στο ερώτημα ($b$), ώστε να εκφράσουμε την προσθήκη σε πρόταση λογικής πρώτης τάξης.Έχουμε: \\

$MemberOf(Municipality of Athens, Municipality)$
\\ \\ \ Το αρχεία βρίσκονται στο $zip \ prover9$ 

\section*{Πρόβλημα 10}
\vspace{5mm}
($\bf a$) \normalfont \\

$\bullet \ $ Τα σύμβολα των σταθερών που αντιπροσωπεύουν αντικείμενα είναι τα εξής: \\ 

\hspace{25mm} \bf Γιάννης, Μαρία, Γιώργος, Ελένη \normalfont \\

$\bullet \ $ Τα σύμβολα κατηγορημάτων που αντιπροσωπεύουν σχέσεις μεταξύ αντικειμένων είναι τα εξής: \\

\hspace{25mm} \bf Παντρεμένοι($ \bf x,y) \longrightarrow \ $  \normalfont οι $x,y$ είναι παντρεμένοι

\hspace{25mm} \bf Αδέρφια($ \bf x,y) \longrightarrow \ $  \normalfont οι $x,y$ είναι αδέρφια\\

$\bullet \ $ Τα σύμβολα συναρτήσεων  που αντιπροσωπεύουν σχέσεις στις οποίες υπάρχει μια τιμή ως είσοδος, είναι τα εξής:\\

\hspace{25mm} \bf μέλοςΣυνδέσμου($ \bf  x) \longrightarrow \ $ \normalfont  o $x$ είναι μέλος του Συνδέσμου \\ \\ \\
Αφού έχουμε εξηγήσει με ακρίβεια τι παριστάνουν τα παραπάνω σύμβολα, θα μετατρέψουμε τις δοσμένες ελληνικές προτάσεις σε προτάσεις της λογικής πρώτης τάξης.Έχουμε: \\ \\ 
\[
\bf 1.  \ \ \  \normalfont \text{μέλοςΣυνδέσμου(Γιάννης)} \ \ \bigwedge \ \ \text{μέλοςΣυνδέσμου(Μαρία)} \ \ \bigwedge \ \ \text{μέλοςΣυνδέσμου(Γιώργος)} \ \ \bigwedge \ \ \text{μέλοςΣυνδέσμου(Ελένη)}
\] 

\[
\bf 2.  \ \ \  \normalfont \text{Παντρεμένοι(Γιάννης,Μαρία)}  \ \ \ \ \ \ \ \ \ \ \ \ \ \ \ \ \ \ \ \ \ \ \  \  \ \ \ \ \ \ \ \ \ \ \ \ \ \ \ \ \ \ \ \ \ \ \  \  \ \ \ \ \ \ \ \ \ \ \ \ \ \ \ \ \ \ \ \ \ \ \  \  \ \ \ \ \ \ \ \ \ \ \ \ \ \ \ \ \ \ \ \ \ \ \  \  \ \ \ \ \ \ \ \ \ \ \ \ \ \ \ \ \ \ \ \ \ \ \  \ 
\]

\[
\bf 3.  \ \ \  \normalfont \text{Αδέρφια(Γιώργος,Ελένη)}  \ \ \ \ \ \ \ \ \ \ \ \ \ \ \ \ \ \ \ \ \ \ \  \  \ \ \ \ \ \ \ \ \ \ \ \ \ \ \ \ \ \ \ \ \ \ \  \  \ \ \ \ \ \ \ \ \ \ \ \ \ \ \ \ \ \ \ \ \ \ \  \  \ \ \ \ \ \ \ \ \ \ \ \ \ \ \ \ \ \ \ \ \ \ \  \  \ \ \ \ \ \ \ \ \ \ \ \ \ \ \ \ \ \ \ \ \ \ \  \ 
\]

\[
\bf 4.  \ \ \  \normalfont (\forall x)(\forall y)\left ( \ \left( \text{μέλοςΣυνδέσμου}(x) \ \ \bigwedge \ \ \text{Παντρεμένοι}(x,y) \ \ \Rightarrow \ \  \text{μέλοςΣυνδέσμου}(y) \right) \ \ \bigvee \ \ \right . \ \ \ \ \ \ \ \ \ \ \ \ \ \ \ \ \ \ \ \ \ \ \ \ \ \ \ \ \ \ \ \ \ \ \ \ \ \ \ \ \ \ \ \ \ \ \ \ \ \ \ \ \ \ \ \ 
\]

\[
\left . \left( \text{μέλοςΣυνδέσμου}(y) \ \ \bigwedge \ \  \text{Παντρεμένοι}(y,x) \ \ \Rightarrow \ \  \text{μέλοςΣυνδέσμου}(x) \right) \right) \ \ \ \ \ \ \ \ \ \ \ \ \ \ \ \ \ \ \ \ \ \ \ \ \ 
\]\\ 
 
$\bullet \ $ Απαλοιφή των συνεπαγωγών:

\[(\forall x)(\forall y)\left ( \ \left( \ \neg \left( \text{μέλοςΣυνδέσμου}(x) \ \ \bigwedge \ \ \text{Παντρεμένοι}(x,y) \right) \ \ \bigvee \ \  \text{μέλοςΣυνδέσμου}(y) \right) \ \ \bigvee \ \ \right . \ \ \ \ \ \ \ \ 
\]

\[
\left . \left( \ \neg \left( \text{μέλοςΣυνδέσμου}(y) \ \ \bigwedge \ \  \text{Παντρεμένοι}(y,x) \right) \ \ \bigvee \ \  \text{μέλοςΣυνδέσμου}(x) \right) \right)
\]\\ 

$\bullet \ $ Μετακίνηση του $\neg$ προς τα μέσα:

\[(\forall x)(\forall y)\left ( \ \left( \ \neg  \text{μέλοςΣυνδέσμου}(x) \ \ \bigvee \ \ \neg \text{Παντρεμένοι}(x,y)  \ \ \bigvee \ \  \text{μέλοςΣυνδέσμου}(y) \right) \ \ \bigvee \ \ \right . \ \ \ \ \ \ \ \ 
\]

\[
\left . \left( \ \neg \text{μέλοςΣυνδέσμου}(y) \ \ \bigvee \ \  \neg \text{Παντρεμένοι}(y,x)  \ \ \bigvee \ \  \text{μέλοςΣυνδέσμου}(x) \right) \right)
\]\\ 

$\bullet \ $ Κατάργηση καθολικών ποσοδεικτών:

\[ \ \left( \ \neg  \text{μέλοςΣυνδέσμου}(x) \ \ \bigvee \ \ \neg \text{Παντρεμένοι}(x,y)  \ \ \bigvee \ \  \text{μέλοςΣυνδέσμου}(y) \right) \ \ \bigvee \ \ \ \ \ \ \ \ \ \ \ 
\]

\[
 \left( \ \neg \text{μέλοςΣυνδέσμου}(y) \ \ \bigvee \ \  \neg \text{Παντρεμένοι}(y,x)  \ \ \bigvee \ \  \text{μέλοςΣυνδέσμου}(x) \right) 
\]\\ \\ \\
Έχουν παραχθεί οι εξής φράσεις: \\ \\
$\bf 1. \normalfont \ \ \  \normalfont \text{μέλοςΣυνδέσμου(Γιάννης)} $  \\ \\
$\bf 2. \normalfont \ \ \  \normalfont \text{μέλοςΣυνδέσμου(Μαρία)} $ \\ \\
$\bf 3. \normalfont \ \ \  \normalfont \text{μέλοςΣυνδέσμου(Γιώργος)} $  \\ \\
$\bf 4. \normalfont \ \ \  \normalfont \text{μέλοςΣυνδέσμου(Ελένη)} $ \\ \\
$\bf 5. \normalfont \ \ \  \normalfont \text{Παντρεμένοι(Γιάννης,Μαρία)} $  \\ \\
$\bf 6. \normalfont \ \ \  \normalfont \text{Αδέρφια(Γιώργος,Ελένη)} $ \\ \\
$\bf 7. \normalfont \ \ \  \ \neg  \text{μέλοςΣυνδέσμου}(x) \ \ \bigvee \ \ \neg \text{Παντρεμένοι}(x,y)  \ \ \bigvee \ \  \text{μέλοςΣυνδέσμου}(y)  $  \\ \\
$\bf 8. \normalfont \ \ \ \ \neg \text{μέλοςΣυνδέσμου}(y) \ \ \bigvee \ \  \neg \text{Παντρεμένοι}(y,x)  \ \ \bigvee \ \  \text{μέλοςΣυνδέσμου}(x) $ \\ \\ \\
$\circ \ \ $ Να μετατρέψετε την πρόταση “Η Ελένη δεν είναι παντρεμένη” σε λογική πρώτης τάξης και να ονομάσετε την πρόταση 

που προκύπτει φ. \\ \\ 
\[
\phi \ = \ \neg \text{Παντρεμένοι(Ελένη,Χ)}
\] \\ \\
($\bf b$) \normalfont \\ \\
Θέλουμε να αποδείξουμε ότι η πρόταση $\neg$ φ ικανοποιείται από την ΚΒ.Αν όντως ισχύει αυτό τότε έχουμε αποδείξει ότι η από τη βάση γνώσης KB δεν έπεται λογικά η πρόταση φ. \\ \\
Λαμβάνοντας υπόψιν τις εξής έννοιες της σημασιολογίας της λογικής πρώτης τάξης: \\ \\
$ 1. \ \ \neg \phi \ = 
 \text{Παντρεμένοι(Ελένη,$x$)} $  \\ \\
$2. \normalfont \ \ \  \normalfont \text{μέλοςΣυνδέσμου(Ελένη)} $ \\ \\
$3. \ \ \neg \text{μέλοςΣυνδέσμου(Ελένη)} \ \ \bigvee \ \  \neg \text{Παντρεμένοι($x$,Ελένη)}  \ \ \bigvee \ \  \text{μέλοςΣυνδέσμου}(x)$ \\ \\
$ 4. \normalfont \ \ \  \normalfont \text{μέλοςΣυνδέσμου(}x) $ \\ \\
Για οποιαδήποτε ανάθεση του $x$ ($x$/Γιάννης/Μαρία/Γιώργος/Ελένη) \ έχουμε ότι προκύπτει ότι ΚΒ  $ | \neq$ φ. \\ \\ \\
($\bf c$) \normalfont Οι προτάσεις της λογικής πρώτης τάξης που πρέπει να προστεθούν στην KB ώστε να ισχύει
ότι KB $|$= φ είναι

οι εξής: \\ \\

1. \ Γυναίκα(Μαρία) \\

2. \ Γυναίκα(Ελένη) \\

3. \ Άντρας(Γιάννης) \\

4. \  Άντρας(Γιώργος) \\

5. \ $(\forall x)(\forall y)$ \ Γυναίκα($x$) \ $\wedge$ \ Γυναίκα($y$) \ $ \Rightarrow \ \neg $Παντρεμένοι($x,y$) \\

6. \ $(\forall x)(\forall y)$ \ Άντρας($x$) \ $\wedge$ \ Άντρας($y$) \ $ \Rightarrow \ \neg $Παντρεμένοι($x,y$) \\

7. \ $(\forall x)(\forall y)$ \ Αδέρφια($x,y$) $ \Rightarrow \ \neg $Παντρεμένοι($x,y$) \\

8. \  $(\forall x, y, z)$ Παντρεμένοι($x,y$) $ \Rightarrow \ \neg $Παντρεμένοι($x,z$) \\

9. \  $(\forall x)$ ($\neg$ Παντρεμένοι($x,x))$ \\

($\bf d$) \normalfont  \ Το αρχεία βρίσκονται στο $zip \ prover9$ 




\end{document}
